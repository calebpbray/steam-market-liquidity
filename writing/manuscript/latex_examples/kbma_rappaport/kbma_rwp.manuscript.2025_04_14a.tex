\documentclass[12pt]{article}  % I typically use 11

\usepackage{amsmath}            % -- improves latex's math output
\usepackage{amsfonts}           % -- extra math symbols and fonts
\usepackage{amssymb}            % -- extra math symbols
\usepackage{authblk}
\usepackage{comment}
\usepackage{enumerate}
\usepackage[nolists, nomarkers]{endfloat} % -- allows for the placement of tables and figures at the end of the document

\usepackage[nohead]{geometry}   %sets page dimensions
\geometry{left=1.10in,right=1.10in,top=1.00in,bottom=1.00in}

\usepackage{graphicx}                  % -- to add graphics

\usepackage[implicit=false]{hyperref}  %implicit = false, doesn't automatically think each reference has a link
\hypersetup{colorlinks=true,urlcolor=blue}

\usepackage{indentfirst}  % indents first paragraph of new sections (overrides default)
\usepackage{natbib}
\usepackage[singlespacing]{setspace}

\usepackage[labelfont=bf,font={stretch=0.93}]{caption}  % stretch=1 is singlespacing
\usepackage{subcaption}
\usepackage{floatrow} %% ! has to do with placement of floats: !
\floatsetup[figure]{style=plaintop}
\floatsetup[table]{style=plaintop}

\DeclareFloatVCode{somespace}{\vspace{0.667\baselineskip}}  %Jordan decreased value from 1.667
\floatsetup{rowpostcode =somespace, margins = centering}
%\DeclareNewFloatType{online_table}{name=Online Table,placement=tbp}


\begin{comment} Disabled preamble stuff (accumulated over many years)
\makeatother
%\usepackage[bottom]{footmisc}
%\usepackage{caption} % font=singlespacing
%\usepackage{color,soul}
%\usepackage{endnotes}
%\usepackage{etoolbox}
%\usepackage[inline]{enumitem} %maeve commented out, messing with my lists
%\usepackage{placeins} %(http://texdoc.net/texmf-dist/doc/latex/endfloat/endfloat.pdf)
%\usepackage{relsize}
%\usepackage{rotating} %\DeclareDelayedFloatFlavor{sidewaystable}{table}
%\usepackage{subcaption,booktabs}
%\usepackage{booktabs}
%\usepackage{subfiles}
%\usepackage{threeparttable}
%\usepackage{trackchanges}
%\usepackage{tikz}  % to draw graphics
%\usepackage{xspace}
%\usepackage{ragged2e}  Overleaf package for ragged and flush alignment

%\usepackage{afterpage}

%\renewcommand{\efloatseparator}{\mbox{}}
%\newcounter{simplenote}
%\newcounter{mcnote}
% *** alt syntax to save fn #: {\par \edef\savedfootnotenumber{\number\value{footnote}}}


%\setcounter{MaxMatrixCols}{30}


%\definecolor{beamer@blendedblue}{RGB}{32,93,121}
%\definecolor{raspberry}{RGB}{145,35,90}
%\definecolor{deptgreen}{RGB}{135,170,69}


% ! if you want to have, e.g., eqn 1a eqan 1b
%\newcommand{\eqnletters}{\setcounter{saveeqn}{\value{equation}}%
%\stepcounter{saveeqn}\setcounter{equation}{0}%
%\renewcommand{\theequation}{\mbox{\arabic{saveeqn}\alph{equation}}}}%
%
%\newcommand{\eqnnumber}{\setcounter{saveeqn}{\value{equation}}%
%\stepcounter{saveeqn}\setcounter{equation}{0}%
%\renewcommand{\theequation}{\mbox{\arabic{saveeqn}.\arabic{equation}}}}%
%
%\newcommand{\reseteqn}{\setcounter{equation}{\value{saveeqn}}%
%\renewcommand{\theequation}{\arabic{equation}}}%
%
%\newcommand{\overbar}[1]{\mkern 1.5mu\overline{\mkern-1.5mu#1\mkern-1.5mu}\mkern 1.5mu}



% ***!!! FOR TWO ADJACENT FOOTNOTES, use $^{,}$ between them !!!***


%\renewcommand\@makefntext[1]{%
%	\noindent\makebox[0.5em][l]{\@makefnmark}#1}
%\DeclareNewFloatType{map}{name=Map,placement=tbp}

% Note/Source/Text after Tables
%\makeatother
\usepackage[flushleft]{threeparttable}
\newcommand{\Figtext}[1]{%
	\begin{tablenotes}[para,flushleft]
		\hspace{6pt}
		\hangindent=1.75em
		#1
	\end{tablenotes}
}

%\usepackage[hidelinks=true]{hyperref}
%\usepackage{hyperref}
%linkcolor=black
%citecolor=black
%}
% colorlinks: false=boxes, true=color
%implicit=false: does not create links to bibliography, figures, etc
%colorlinks: use font color rather than a surrounding box to identify links


\end{comment}

\makeatletter
\setlength{\@fptop}{0pt}


\hyphenation{KBMA}
\hyphenation{KBMAs}
\hyphenation{CBSA}
\hyphenation{CBSAs}

\title{\vspace{-10mm}A Better Delineation of U.S.\ Metropolitan Areas}
\author{\vspace{5mm}Jordan Rappaport and McKenzie Humann \\ \vspace{-0.8mm}
Federal Reserve Bank of Kansas City\thanks{The views expressed herein are those of the authors and do not necessarily reflect the views of the Federal Reserve Bank of Kansas City or the Federal Reserve System. Thank you to Chris Cunningham, Giles Duranton, Xavier Gabaix, Todd Gardner, Sanghoon Lee, Albert Saiz, Elisabet Viladecans-Marsal, and several anonymous referees for helpful feedback. Thank you to Caleb Bray, Francis Dillon, and Sai Avinash Sattiraju for excellent research assistance.  Supplemental maps, enumerations, and detailed tables are available from the paper's \href{https://www.kansascityfed.org/research/research-working-papers/a-better-delineation-of-us-metropolitan-areas/}
      {\underline{webpage}}. }}
%\author[1]{McKenzie Humann}
%\author[1]{Jordan Rappaport}
%\affil[1]{Federal Reserve Bank of Kansas City}
\date{April 2025}


\begin{document}




\maketitle
\thispagestyle{empty}  % suppresses printing page number on title page

\sloppy %ok avoids the breakage of words at the end of lines, by adjusting spaces between words inside the lines

\onehalfspacing % reset below just before starting main text

\begin{abstract} %Metropolitan areas are a fundamental unit ....
\singlespacing
\noindent    Metropolitan areas are a fundamental unit of economic analysis. Broadly defined, they are unions of built-up locations near each other among which people travel between places of residence, employment, and consumption. Despite the importance of metropolitan areas, metropolitan Core-Based Statistical Areas and other official U.S.\ delineations considerably stray from this broad definition. We develop a simple algorithm to better match this definition, using commuting flows among U.S. census tracts in 2000. Three judgmental parameters govern the minimum strength of commuting ties between locations to include them in the same metropolitan area, the maximum separating distance between locations, and the minimum density of outlying settlement. A parameterization that balances encompassing commuting flows and excluding sparsely settled land delineates 361 Kernel-Based Metropolitan Areas (KBMAs), in aggregate capturing almost all the population and employment of metropolitan CBSAs in a small fraction of their land area. Additionally, we benchmark KBMAs against two alternative parameterizations, one that prioritizes encompassing commuting flows and one that prioritizes excluding sparsely settled land. % March 01, 2025: 158 words

\end{abstract}
\strut


\noindent \textbf{Keywords:} delineating metropolitan areas, Core-Based Statistical Areas, commuting flows, city size
\strut

\vspace{3mm}

\noindent \textbf{JEL Classification Numbers:}  R12, R14, R23
% \pagebreak       %breaks to the next page
\clearpage
\spacing{1.5}
\pagenumbering{arabic}
\setcounter{page}{1}

\section{Introduction}

Metropolitan areas are a fundamental unit of economic analysis. We define them broadly as unions of built-up locations near each other with combined population of at least moderate scale and among which a significant share of residents and workers travel on a day-to-day basis between places of residence, places of employment, and places of consumption. By this definition, metropolitan areas correspond to distinct markets for labor and for non-traded goods and services. They are also likely to correspond to the geographic area determining many agglomerative externalities and for sharing many production and consumption amenities. In addition, national populations sort themselves across and within metropolitan areas with respect to numerous characteristics. Conversely, metropolitan areas may foster a sense of shared identity among diverse residents.

Despite this fundamental importance, official delineations considerably stray from our broad definition of metropolitan areas. Metropolitan Core-Based Statistical Areas (CBSAs), delineated by the U.S.\ Office of Management and Budget, vastly overbound metropolitan land area. For example, the Phoenix CBSA spans more than 14,000 square miles, mostly empty desert, and the Honolulu CBSA includes a Pacific atoll more than 900 miles from its downtown. Even so, many commonly recognized labor markets are split into multiple metropolitan CBSAs, including Raleigh and Durham, Los Angeles and Riverside--San Bernadino, and San Francisco and San Jose. Conversely, other arguably separate metropolitan areas are encompassed by a single metropolitan CBSA. For example, 20 miles of empty desert in 2000 separated the settled portion of the Coachella Valley---Palm Springs and neighboring municipalities in the central portion of the Riverside--San Bernadino CBSA---from the settled western portion of the CBSA.

Two other official delineations of U.S.\ metropolitan areas in 2000 egregiously stray from our definition. Urbanized Areas (UAs), combinations of census blocks with contiguous population density above a specified threshold, fragment unambiguous metropolitan areas. For example, a slight gap in residential settlement causes a portion of one of Kansas City's main suburban municipalities to be delineated as its own UA, notwithstanding that more than half of its employed residents work in the Kansas City UA and more than one third of its employment is made up of workers living in the Kansas City UA. Commuting Zones (CZs), combinations of U.S. counties delineated to identify rural labor markets, both carve up some unambiguous metropolitan areas and arguably span multiple others. For example, six commuting zones in the vicinity of New York City are connected to Manhattan by commuter rail. Conversely, a commuting zone in Pennsylvania fully encompasses four metropolitan CBSAs (Harrisburg-Carlisle, Lancaster, Lebanon, and York-Hanover).


The failures of official delineations impede decision making, policy implementation, and fundamental understanding. Most immediately, metropolitan delineations affect a wide range of choices,  including on infrastructure investment, transportation planning, corporate location, and government-sponsored development. In addition, implementing a slew of legislation depends closely on metropolitan delineations \citep{CRS_2014}. For example, the income ceiling in 2024 for a neighborhood to qualify as low income under the Community Reinvestment Act was 13 percent lower in the Durham CBSA than in the Raleigh CBSA.
%https://www.ffiec.gov/Medianincome.htm
%C:\local\Fed Reserve KC Dropbox\Jordan Rappaport\KBMA\workbooks\ffiec_msa_2024_median_incomes_2025_01_31b.xlsx


From a research perspective, \cite{duranton_2021} argues that ``meaningful and appropriate'' delineations of metropolitan areas are required ``to understand anything about fundamental urban questions.'' Similarly, better delineations are likely to improve our understanding of many other economic processes for which variation across locations underpins empirical study. Such questions include labor supply and demand (e.g., \citeauthor{kennan_walker_2011}, \citeyear{kennan_walker_2011}; \citeauthor{autor_dorn_hanson_aer_2013}, \citeyear{autor_dorn_hanson_aer_2013}; \citeauthor{monte_redding_rossi-hansberg_2018}, \citeyear{monte_redding_rossi-hansberg_2018}),
housing supply and demand (e.g., \citeauthor{green_malpezzi_mayo_2005}, \citeyear{green_malpezzi_mayo_2005}; \citeauthor{saiz_qje_2010}, \citeyear{saiz_qje_2010}; \citeauthor{van-nieuwerburgh_weill_2010}, \citeyear{van-nieuwerburgh_weill_2010}, \citeauthor{landvoigt_piazzesi_schneider_2015}, \citeyear{landvoigt_piazzesi_schneider_2015}),
non-housing consumption demand
(\citeauthor{mian_sufi_2012}, \citeyear{mian_sufi_2012}, \citeyear{mian_sufi_2014}; \citeauthor{guren_mckay_nakamura_steinson_2021}, \citeyear{guren_mckay_nakamura_steinson_2021}),
productivity spillovers (\citeauthor{glaeser_mare_2001}, \citeyear{glaeser_mare_2001}; \citeauthor{moretti_aer_2004}, \citeyear{moretti_aer_2004}; \citeauthor{combes_duranton_gobillon_2008}, \citeyear{combes_duranton_gobillon_2008}; \citeauthor{greenstone_hornbeck_moretti_2010}, \citeyear{greenstone_hornbeck_moretti_2010}),
and social mobility
\citep{chetty_hendren_kline_saez_qje_2014}.


%local multipliers (e.g., \citeauthor{}, \citeyear{} ),
%the relationship between size and growth (\citeauthor{gabaix_qje_1999} , \citeyear{gabaix_qje_1999}; \citeauthor{eeckhout_2004} , \citeyear{eeckhout_2004}; \citeauthor{desmet_rappaport_jue_2017}, \citeyear{desmet_rappaport_jue_2017}),  [I'm going to let this fall under ``fundamental urban questions]
% other labor supply: D Yagan, JPE, 2019; ``Employment Hysteresis from the Great Recession'';

To address these and other questions and purposes, we develop a simple algorithm that delineates metropolitan areas that more closely match our broad definition. It combines elements from the methodologies used to delineate CBSAs, UAs, CZs, and several other alternatives. Similar to the construction of metropolitan CBSAS, UAs serve as cores, anchoring our delineations. The algorithm iteratively joins these cores together to form kernels based on the strength and distance of commuting flows among them and then builds out from the kernels to form Kernel-Based Metropolitan Areas (KBMAs).

Our delineation algorithm, like all others, requires judgment to set the values of parameters. Two such parameters set the minimum strength of commuting ties and the maximum separating distance for locations to be combined in the same metropolitan area; a third parameter sets the minimum density for outlying locations to be included. The algorithm is fully transparent and replicable; unlike CBSAs and UAs, it does not depend on the opinions of local residents.


We think of KBMAs as a baseline parameterization that is likely to be appropriate for most questions and purposes; it balances encompassing commuting flows and excluding locations that arguably are not sufficiently nearby or built up. As alternative benchmarks for other questions and purposes, we delineate more expansive kernel-based metropolitan regions and more compact kernel-based urban areas: The former parameterization more heavily weights encompassing commuting flows and so may better match studying extended connections such as long-distance commuting and regional supply chains. The latter parameterization more heavily weights excluding less built-up and less near locations and so may better match studying narrow geographic spillovers and urban land usage. Our delineation algorithm can also be flexibly parameterized to match numerous other questions and purposes.


The 361 KBMAs we delineate encompass 88 percent of the aggregate population of metropolitan CBSAs and 94 percent of their aggregate employment in just 15 percent of their aggregate land area. Their population ranges from 50,000 to 19 million and their land area from 27 to 6,100 square miles.  The population distributions of the kernel-based metropolitan regions and kernel-based urban areas closely match that of KBMAs. Under all three parameterizations, land area expands less than proportionately with population, consistent with centripetal forces, such as centralized employment and amenities, constraining metropolitan expansion.


\section{Existing Metropolitan Delineations}


Delineating metropolitan areas requires making a number of judgments, either explicit or implicit. What geographic units serve as a building blocks?  What makes building blocks sufficiently integrated to join them together in a cluster? Should clusters be anchored by a core? What qualifies a cluster as metropolitan?


% Possibilities for the United States include counties, the main subdivision of states; tracts, the main subdivision of counties; and blocks, the most granular statistical unit.

\subsection{Building Blocks}

%\cite{briant_combes_lafourcade_2010} describe how different sized geographies can lead to different conclusions of classic econ geography questions such as spatial concentration, agglomeration economies, and trade determinants. But their discussion of the Modifiable Areal Unit Problem (MAUP) is more about the constructed geographies rather than the building blocks used in the construction.

% see excellent Frankie summary: 2000_census_summary.2022_05_11a.docx
Census blocks are the most granular geographic unit for which the Census Bureau collects data. Blocks are typically quite small, for example the rectangle bounded by one city block on each side, but can also span many square miles in sparsely settled areas.  For the 2000 decennial census, the Census Bureau delineated approximately 8.2 million blocks with land area ranging from 0 (all water) to 8,072 square miles (unsettled), with a median of 0.01 square miles. Their population ranged from 0 (more than a third of blocks) to a bit over 23,000. Across census blocks with residents, the median population was 25. Only limited block data is publicly available.


%shows building blocks as reported in the 2000 decennial census for the 50 states and the District of Columbia. CBSAs and UAs/UCs are constructed using published criteria for the 2000 decennial census \citep{OMB_2000,Census_2002}.

Census block groups combine up to 1,000 blocks and serve mainly for statistical purposes. Specifically, they are the smallest geographic unit for which the Census Bureau reports tabulated data from decennial census sample questions and from the American Community Survey.

\begin{figure}[tb] %diagram of building blocks
\caption{\label{geo-hierarchy} \textbf{Metropolitan Building Blocks Using the 2000 Decennial Census}}
\includegraphics[scale = 0.60, trim = 45mm 43mm 0mm 70mm, clip]{
kbma_figure_census2000_geography_2025_03_12a.pdf}
\vspace{0mm}\begin{flushleft}
\footnotesize{Note: excludes U.S.\ territories.}
\end{flushleft}
\end{figure}

Census tracts combine census block groups with several goals, including maintaining relatively stable boundaries over time \citep{census_1997_tracts}. Even so, numerous  tracts are re-delineated in preparation for each decennial census. For the 2000 census, the Census Bureau targeted tracts to encompass between 1,500 and 8,000 residents. Nevertheless, 2 percent had population in 2000 below 500 and a handful had population above 20,000. Tracts' land area in 2000 ranged from 0.06 square miles at the 1st percentile to 800 at the 99th percentile (median = 1.96 square miles). The endogenous delineation of census tracts to meet the population targets induces a tight negative correlation between land area and population density. %(the correlation coefficient of their logarithms is -0.96).
%The extreme right tail includes several tracts in New York City, a correctional facility, and an airport.
% other goals: identifying geographic areas that represent ``meaningful'' geographic divisions of a county based on economic and social interactions, demographic differences, and topographic differences.

Counties in 2000 combined between 1 and 2,100 tracts, with population ranging from 1,000 at the 1st percentile to 1.1 million at the 99th percentile (median = 25,000) and land area ranging from 10 square miles at the 1st percentile to 8,100 square miles at the 99th percentile (median = 616). Importantly, their average land area considerably differs across U.S.\ regions, ranging from a median of 425 square miles in the South Atlantic census division to a median of 2,275 in the Mountain census division; this variation suggests that metropolitan delineations constructed using counties as building blocks are not comparable across different parts of the U.S. On the other hand, county borders have remained relatively stable since 1920, making counties an ideal building block for delineating metropolitan areas with unchanged borders since then. In addition, a wealth of county data is available from government agencies and private organizations.

\begin{comment}  endogenous delineations of counties and source of data
% median pop and land area documented in sf3_2000.county_sumstats_by_state.2024_10_08a.log
%County delineations partly reflect the path of historical development, with counties subdividing and borders shifting as settlement expanded west throughout the 19th century \citep{thorndale_dollarhide_1987}.
%  as population increased and new land was settled by non-Native Americans.
%For example, the number of counties and county equivalents delineated in Texas increased from 81 in 1850 to 225 in 1880 to 254 since 1921.
% I compiled the county numbers in Texasfrom ICPSR 02896 (Michael Haines) and from Dollar and Thorndale (1987).
\end{comment}

\subsection{Urbanized Areas and Urban Clusters}

% helpful website: https://www.census.gov/programs-surveys/geography/guidance/geo-areas/urban-rural.html

\begin{comment}  original introductory paragraph
%Urbanized Areas were first delineated by the Census Bureau following the 1950 decennial census, partly in response to rapidly growing population just outside the boundaries of large incorporated municipalities \citep{ratcliffe_2015}.Initially, UAs were constructed as the combination of incorporated municipalities with population of at least 50,000 together with contiguous municipalities with population of at least 2,5000 and contiguous territory with at least 500 housing units per square mile or that was used for commercial, industrial, transportation, or recreational purposes. The criteria evolved over time, notably with the relaxing of the population thresholds and an increasing emphasis on residential density.
\end{comment}

Urbanized Areas were first delineated by the Census Bureau following the 1950 decennial census, partly in response to rapidly growing population just outside the boundaries of large incorporated municipalities \citep{ratcliffe_2015}. Beginning with the 2000 decennial census, the Census Bureau used a mostly granular algorithm to delineate UAs and a smaller counterpart, Urban Clusters (UCs), as combinations of census blocks \citep{Census_2002}.\footnote{The separate designations were dropped following the 2020 decennial census, in favor of ``Urban Area'', and the threshold population for inclusion was increased from 2,500 to 5,000 \citep{Census_2022}.}

The granular portion of the UA/UC algorithm focuses on residential density and proximity. It starts by identifying clusters of two or more contiguous block groups or blocks with population density of at least 1,000 per square mile. Expanding outward, adjacent block groups and blocks with population density of at least 500 are attached. Each resulting cluster is then joined with any similarly constructed clusters separated from it by no more than 0.5 miles along a connecting road segment, forming \emph{interim cores} of chained clusters. Next, interim cores separated by no more than 2.5 miles along a connecting road segment are joined together, forming chained cores.  Additional criteria add in blocks mostly surrounded by the chained cores as well as adjacent blocks containing a major airport. The resulting unions of census blocks constitute \emph{urban area agglomerations}.\footnote{The term ``urban area agglomeration'' was introduced following the 2020 decennial census \citep{Census_2022}.}

% \footnote{\cite{rozenfeld_etal_2008, rozenfeld_etal_2011} formalize a granular methodology for delineating population clusters that approximately matches this first stage.}

All urban area agglomerations with population between 2,500 and 50,000 qualify as UCs under the 2000 algorithm, and most with population above this qualify as UAs. However, some of the larger agglomerations are split into two or more UAs to backwardly align them with metropolitan statistical areas (MSAs) delineated during the 1990s. For example, the Los Angeles UA is contiguous with two other UAs, Riverside--San Bernadino and Thousand Oaks, and so the three constitute a single urban area agglomeration. Splitting it aligns each of the resulting UAs with a legacy MSA. This backward alignment to the 1990s is perpetuated in the algorithms constructing UAs following the 2010 and 2020 decennial census, which split urban area agglomerations to avoid combining UAs previously delineated as separate \citep{Census_2011, Census_2022}.\footnote{Backwardly aligning UAs has required more extensive splits of urban area agglomerations as urban settlement has spread. For UAs delineated following the 2000 decennial census, we identify 32 pairs with contiguous borders, ranging from a single point to a maximum of 9 miles, implying that splits were made at each. Eight of these contiguous borders exceeded 3 miles in length, the maximum prescribed by the published rule for the 2000 delineations \citep{Census_2002}. For the UAs delineated using the 2020 decennial census, we identify 121 pairs with contiguous borders, ranging in length up to 47 miles.} %A separate backward exception splits some urban area agglomerations located entirely within a 1990s MSA or entirely within a county not included in any MSA.
% \footnote{Using criteria established for the 1990 decennial census, OMB delineated a Los Angeles--Riverside--Orange County Consolidated Metropolitan Statistical Area, divided into four Primary Metropolitan Statistical Areas: Los Angeles--Long Beach, Riverside--San Bernardino, Oxnard--Ventura, and Anaheim--Santa Ana. Most splits were restricted to cases where the resulting UAs had a contiguous border of no more than 3 miles. In consequence, the Los Angeles UA was not further divided to form a separate UA aligned with the Anaheim--Santa Ana PMSA.}

Backwardly aligning the UAs in 2000 with legacy MSAs breaks an otherwise granular algorithm, contaminating using UAs for some purposes. In addition to the explicit history dependence, doing so introduces ad hoc subjectiveness that was applied idiosyncratically across locations. Specifically, metropolitan delineations during the 1990s relied heavily on local opinion---the views of a wide range of public groups including business and other leaders, chambers of commerce, planning commissions, and local officials---to partition larger MSAs, designated as \emph{consolidated}, into two or more \emph{primary} MSAs \citep{OMB_1990}.

%Illustrating the problem for research, the backward-looking splits significantly reshape the population distribution across large UAs, contaminating inferring stochastic processes governing metropolitan population growth \citep{gabaix_qje_1999,eeckhout_2004}.

From a pragmatic perspective, a key flaw of UAs and UCs is that they fragment many locations that are arguably integrated. For example, as illustrated in Figure \ref{uas_in_sc}, five separate UAs are located along a 25-mile ray in the southern portion of the Oxnard--Thousand Oaks CBSA. Large commuting flows among three of these---Los Angeles, Simi Valley, and Thousand Oaks---far exceed our threshold metric for combining them in a single metropolitan area as do the commuting flows between the Oxnard and Camarillo UAs. Similarly, as described in the introduction, a slight gap in residential settlement causes a portion of one of Kansas City's main suburban municipalities to be delineated as its own UA, notwithstanding that more than half of its residents work in the Kansas City UA and more than one third of its employment is made up of workers living in the Kansas City UA. Analogous suburban UAs, from which at least half of employed residents commute to a much larger UA, were delineated adjacent to the Phoenix, Seattle, Chicago, and New York UAs.\footnote{The Lee's Summit, MO UA is separated from the Kansas City UA by 0.4 miles (implicitly not along a connecting road segment as otherwise they would not be separate). The Avondale, AZ UA is separated from the Phoenix--Mesa UA by 0.5 miles; the Marysville, WA, UA is separated from the Seattle UA by 0.4 miles; the Round Lake Beach--McHenrey--Grayslake, IL-WI UA is separated from  the Chicago UA by 0.4 miles; and the Hightstown, NJ UA is separated from the New York--Newark  UA by 0.1 miles.}

\begin{comment} documentation of commuting among identified suburban UAs and separating distance based on nearest centroids of tract-based approximations

from pairwise_core_flows.2025_03_14a.xlsx

% Avondale, AZ 		    Phoenix--Mesa, AZ 	        1.78	1.033	0.572	0.445   0.004	0.011

% Kansas City, MO--KS   Lee's Summit, MO 	        3.08	1.014	0.014	0.023	0.605	0.372

% Marysville, WA 	    Seattle, WA 	            3.41	0.871	0.600	0.244	0.006	0.020

% Chicago, IL--IN 	    Round Lake Beach, IL--WI 	3.03	0.833	0.003	0.015	0.576	0.239	

% Los Angeles, CA 	    Santa Clarita, CA 	        3.18	0.780	0.002	0.008	0.533	0.237

% Hightstown, NJ 	    New York                	1.43	0.738	0.513	0.223	0.000	0.002



xx percent of workers in the Avondale, AZ UA commute to the Phoenix--Mesa UA; yy percent of the workers in the Round Lake Beach, IL--WI UA commute to the Chicago UA; and zz percent of workers in the Denton--Lewisville, TX UA  commute to work in the Dallas--Fort Worth--Arlington UA; (Marysville WAS; The Woodlands Texas)
\textbf{STOPPED HERE}


For example, one such gap splits the area in the vicinity of the Kansas City, MO municipality into two UAs. The smaller is an unambiguous suburb of the larger, where more than half its employed residents work and more than one third of its employees live.

\end{comment}

\begin{figure}[tb] %UAs in Southern CA
\caption{\label{uas_in_sc} \hspace{-3mm} \textbf{Urbanized Areas in Southern California}}.
%\begin{center}
\hspace{0mm}
\includegraphics[scale = 0.74, trim = 32mm 33mm 0mm 33mm, clip]{kbma_map_socal_uas_2024_12_19.png} %png is 600 dpi
%\end{center}

\vspace{-4mm}
\begin{flushleft}
\footnotesize{Notes: Borders represent delineations following the 2000 decennial census. Blue lines demarcate metropolitan CBSAs. The Los Angeles UA includes a narrow corridor of census blocks along its northwest coast, some of which are not visible.}
\end{flushleft}
\end{figure}

Conversely, the relatively low density threshold for inclusion in urban area agglomerations contributes to some chains that arguably span multiple metropolitan areas. For example, a single urban area agglomeration stretches from northern New Jersey up along the Atlantic coast almost to New London, including a branch extending from New Haven north to Springfield, MA.


\begin{comment} Some details on UAs and UCs

 Census blocks---the smallest geographic unit tabluated by the census, normally bounded be visible physical features together with non-visible administrative boundaries---

    \item delineation criteria: The Census Bureau uses automated software program to apply the delineation rules in a sequence until there is no longer any surrounding territory that meets the inclusion criteria. The Census Bureau first identifies an "Initial Core" which consists of one or more contiguous block groups spanning less than 2 square miles and with a population density of greater than 1,000 people per square mile.\footnote{Land area and density calculations exclude areas of water in the block groups.} If there are no block groups that meet this criteria, the Census Bureau will assign block groups or even blocks that meet less restrictive criteria on population density or land area to be the "Initial Core." In the next step, the Census Bureau adds any block groups with greater than 1000ppsm and is less than 0.5 miles away from the edge of the "Initial Core" via a road connection. This is called a 'hop' connection. Once territory is added, it becomes a part of the 'Initial Core' and the Census Bureau once again looks for outside terriroty that can be added via a 'hop' connection, until there is no more qualifying territory. At this point the inital core and the territory that was added are considered to be the 'interim core'. To continue the delineation process, the Census Bureau identifies territory outside of the 'interim core' that could be added via a 'jump' connection. A 'jump' connection occurs when a block group is less than 2.5 miles away from the edge of the 'interim core' and has a population density greater than 500ppsm. Though the Census Bureau does not add an additional round of 'jump' connections, it adds one last round of 'hop' connections. Finally, any territory that is surrounded by territory assigned to the Urbanized Area is added so that the resulting Urbanized Area covers a contiguous section of land. % This probably does not need to be this detailed.

    \item drawbacks: By using blocks and block groups, urbanized areas provide a more granular coverage of metropolitan areas and potentially offer a more accurate measurement of metropolitan land area. However, the complex and rigid delineation rules result can create metropolitan areas that seem too small or that separate two urbanized areas belonging to one conceptual metropolitan area. For instance, the strict distance cut-offs resulted in two urbanized areas in the Kansas City metropolitan area, Kansas City, MO and Lee's Summit, MO even though Lee's Summit is strongly economically and socially integrated within the greater Kansas City metropolitan area. Moreover, the delineation criteria rely solely on population and population density without considering the connectedness of the block groups via commuting data or other measurements of how people interact or move within an urbanized area. As a result, urbanized areas cannot fully capture the basic concept of a metropolitan area as the land across where people live, work, and conduct the majority of their activities.


     Title Rules:  Federal Register 67, No. 51, March 15, 2002
     Domain: incorporated places with pop 2,500+
     1st Name:  incorporated place with most population
     Up to two additional names IF the additional incorporated place(s) has urban area population exceeding 250,000 OR has population exceeding 2,500 and it's population is at least two-thirds of the population in the most populous incorporated place in the urban area.
     State Postal Abbreviations: of all states with area in the UA/UC. Order corresponds, first, to the order of any incorporated places in title, and then any other states in descending order of their population in the UA/UC

\end{comment}

\begin{comment}

Dijkstra, Lewis and Hugo Poelman, 2014. ``A Harmonised Definition of Cities and Rural Areas: the New Degree of Urbanisation.''  European Commission Working Paper


\end{comment}


\subsection{Metropolitan Core-Based Statistical Areas}

The predecessor to the Office of Management and Budget (OMB) began delineating metropolitan statistical areas in the late 1940s, motivated by the desire to have geographic units for which government agencies could collect, tabulate, and publish data. The specific criteria and name of the delineated units evolved over time \citep{Berry_1969, OMB_1998, Gardner_1999, Gardner_2021}.

In preparation for the 2000 decennial census, OMB significantly revised its criteria to delineate metropolitan and micropolitan Core-Based Statistical Areas \citep{OMB_2000}. Both are constructed using counties as geographic building blocks, with UAs serving as cores anchoring metropolitan CBSAs and UCs with population of at least 10,000 serving as cores anchoring  micropolitan CBSAs. Each of these UAs and UCs is associated with one or more \emph{central counties}, in which they are substantially encompassed. Nearby counties are designated as \emph{outlying counties} of the CBSA if at least 25 percent of their employed residents work in the central counties or if at least 25 percent of their employment is made up of workers who live in the central counties. The published criteria also allow OMB to combine adjacent groups of counties that are algorithmically delineated as separate CBSAs if local opinion favors doing so.

As summarized in Table \ref{metro-comparisons}, OMB delineated 362 metropolitan CBSAs in 2000 with population ranging from 52,500 to 18.3 million and 560 micropolitan CBSAs with population ranging from 13,000 to 182,000. The considerable overlap in population between the two types emphasizes that they are distinguished from each other based on the population of their cores. OMB retained essentially the same criteria for the 2010 and 2020 decennial censuses \citep{OMB_2010,OMB_2021}. It also periodically updates delineations based on intercensal population estimates.%These updates typically take the form of adding new micropolitan CBSAs, promoting CBSAs from micropolitan to metropolitan status, and retitling CBSAs based on changes in the population ordering of their largest municipalities.

%Outlying counties were also required to be contiguously connected to a central county, either directly or via another outlying county.

\begin{table}[tb] % Official Delineations of U.S. Metropolitan Areas
%\begin{center}
\caption{\label{metro-comparisons} \textbf{Official Delineations of U.S.\ Metropolitan Areas}}
\includegraphics[scale = 1.2, trim = 18mm 198mm 10mm 19mm, clip]{kbma_table_official_delineations_2025_03_12a.pdf}
%\end{center}
\vspace{-7mm}\begin{flushleft}
\footnotesize{Notes: Delineations are based on the 2000 decennial census and exclude U.S.\ territories. The CBSA delineations are the version promulgated on June 6, 2003 \citep{OMB_2003}; the Commuting Zone delineations are disseminated by the Economic Research Service \citep{ers_2012}.}
\end{flushleft}
\end{table}

\begin{comment} formal description of commuting criterion
% More formally, let ${w}_o$ denote workers (i.e., employed residents) who reside in a candidate outlying county; ${e}_o$ denote employment in that county; ${f}_{o,c}$ denote the flow of workers commuting from the candidate outlying county to central counties; and ${f}_{c,o}$ denote the flow of workers from the central counties to the outlying county. The strength between the candidate outlying county and the core counties, $s_{o,c}$, is calculated as $\max({f}_{o,c}/{w}_o, \, {f}_{c,o}/{e}_o)$. Candidate outlying counties in the CBSA of the central county if $s_{o,c} \, \ge 0.25$
\end{comment}

Metropolitan CBSAs deviate from our metropolitan definition in several ways. First, many vastly overbound metropolitan land area, reflecting the use of counties as building blocks. As described in the introduction, the vast majority of the Phoenix CBSA is empty desert. Similarly, the Riverside--San Bernadino and Anchorage CBSAs each span land area more than three times the size of Massachusetts. Even in the New York City--Newark--Edison CBSA, half of the land area in 2000 was accounted for by census tracts with both population and employment density below 500, the threshold for a census block to be included in a UA or UC.

Second, some CBSAs fully encompass what are arguably separate metropolitan areas. For example, the settled portion of the Coachella Valley constitutes its own television broadcasting market, contributing to our judgment that it belongs to a different metropolitan area than the Riverside--San Bernadino UA, which is a part of the Los Angeles television market.

%\footnote{The Honolulu CBSA includes a Pacific atoll, Laysan's Island, more than 900 miles to the northwest of its downtown.}%The lack of a threshold for either the population or population density of outlying counties also contributes to the expansiveness. For example, Golden Valley County, part of the Billings, MT metropolitan CBSA, had just over 1,000 residents in 2000.

Third, many CBSAs underbound metropolitan areas. For example, the rule splitting urban area agglomerations into multiple UAs causes Los Angeles and Riverside--San Bernadino to be delineated as separate CBSAs as it also does for San Francisco and San Jose. For Raleigh and Durham NC, a slight gap in residential settlement between them in 2000 causes them to be delineated as separate UAs, a necessary condition for them to be included in separate CBSAs. An appendix table enumerates other pairs of metropolitan CBSAs with large cross commuting. Especially egregious, 39 percent of workers residing in the Hinesville--Fort Stewart, GA metropolitan CBSA in 2000 commuted to places of employment in the Savannah, GA metropolitan CBSA.\footnote{The high commuting outflow rate from Hinesville--Fort Stewart to Savannah does not meet the 25 percent threshold for merging because a significant portion of it originates in an outlying rather than central county. In recognition of overbounding and underbounding, OMB also delineates subsets of CBSAs, Metropolitan Divisions, and supersets of them, Combined Statistical Areas. It also constructs analogs for the six New England states using county subdivisions as geographic building blocks. }

For research purposes, the CBSA delineations suffer from several methodological flaws. One is that they are history dependent. As described previously,  metropolitan CBSA cores, UAs, are split to backwardly align with MSAs delineated during the 1990s. This history dependence deepened with the 2010 and 2020 delineations, reflecting that suburban expansion filled in unsettled gaps between UAs, thereafter requiring more extensive splits. Worse, the backward alignment is to delineations that relied heavily on local opinion, contaminating the CBSA algorithm.\footnote{Illustrating this heavy dependence on local opinion, the 1990s algorithm delineated a Consolidated MSA chaining from southern New Jersey up through New York City and then further north into Connecticut and eastern Pennsylvania. OMB relied on local opinion to divide it into 15 Primary MSAs. Among the more idiosyncratic, the Jersey City Primary MSA is made up of a 14-mile strip of land tightly wedged between Newark, NJ and the Hudson River.} Another flaw arises from the considerable variation in counties' average land area across U.S.\ regions, which implicitly delineates CBSAs differently across regions and so limits comparisons. %In addition, as we argue below, delineating CBSAs based on the population of their cores rather than their total population distorts research related population distributions.


\begin{comment}   TABLE WITH LARGE FLOWS BETWEEN CBSAS
\begin{table}[tb] % pairwise commuting flows between metro CBSAs
\vspace{0mm}  %
\includegraphics[scale = 1.05, trim = 18mm 194mm 0mm 19mm, clip]{kbma_table_metrocbsa_pairwise_flows_2024_09_13a.pdf}
\caption{\label{table_append_flows_pairs_p1} \textbf{Pairwise Commuting Flows between Metropolitan CBSAs}. {\normalfont \footnotesize \hspace{0.5mm} Outflow rates are the share of employed residents that commute to the paired CBSA. Inflow rates are the share of employees that commute from the paired CBSA. Displayed pairs are those for which the sum of the flow rates exceeds 0.40. Appendix C reports all pairs for which the sum of the flow rates exceeds 0.10.}}
\end{table}
\end{comment}


%Finally, CBSA delineations---like the other existing delineations---fail to take account of the location of consumption away from home.
%\footnote{Counties are designated as central if they have at least 50 percent of their population in UAs or UCs of at least 10,000 population or if they have within their boundaries a population of at least 5,000 located in a single UA or UC of at least 10,000 population.}

\begin{comment}  motivation for 1910 delineation; Supplemental metropolitan delineations (NECTSs, Metropolitan divisions, etc.)

\footnote{OMB supplemented its CBSA delineations with several alternative metropolitan proxies. For New England, OMB delineated metropolitan and micropolitan New England City and Town Areas (NECTAs), which also used UAs and UCs as cores but were constructed with MCDs rather than counties as building blocks. A number of larger metropolitan CBSAs were partitioned into multiple county-based Metropolitan Divisions. Likewise, the Boston-Cambridge-Quincy MA-NH NECTA was partitioned into multiple MCD-based NECTA Divisions. Conversely, adjacent CBSAs with strong commuting ties and adjacent NECTAs with strong commuting ties were respectively merged to form Combined Statistical Areas (CSAs) and Combined NECTAs.}

%To better capture the changing nature of U.S. settlement patterns, the Census Bureau added Metropolitan Districts as a new geographic area classification in 1910. The primary purpose of Metropolitan Districts was to capture 'the greater city' as the economic influence of many cities had expanded beyond their municipal boundaries by the early 20th century (U.S. Census Bureau, 1932).% “If we are to have a correct picture of the massing or concentration of population in extensive urban areas… it is necessary to establish metropolitan districts which will show the magnitude of each of the principal population centers.” Metropolitan Districts were available through the first half of the 20th century, however their usefulness was limited by the small amount of economic data that were published for minor civil divisions, the geographic building blocks of metropolitan districts.

\end{comment}

%Apparently during 1950s, OMB relied on state and local employment agency surveys to glean information about commuting. 1960 census first to ask about place of work, allowing commuting criteria to be explicitly include;d}

\begin{comment} (includes sources)


% \footnote{A tiering scheme was also introduced following the 1980 decennial census. Metropolitan areas with population of at least 1 million were eligible to be split into two or more county-based {Primary Metropolitan Statistical Areas} (PMSAs), in which case the encompassing metropolitan area was labeled a {Consolidated Metropolitan Statistical Areas} (CMSA). Non-divided metros were labeled {Metropolitan Statistical Areas} (MSAs). }

%Several of the CMSAs arguably far exceeded conceptual metropolitan ares. For example, the New York-Northern New Jersey-Long Island CMSA spanned 7,800 square miles across 32 counties in 4 states, stretching \textbf{xxx} miles north to south  and \textbf{yyy} miles east-west. Conversely, many PMSAs fell far short of conceptual metropolitan areas. For example, the Jersey City PMSA constituted a sliver of land measuring 32 square miles wedged between Manhattan and Newark. (for measuring: from Duchess County NY down to Ocean County NJ AND Old Saybrook CT to Pike County PA)


% *** see Q:\x\identification\metros\cbsa_definitions.June2003.2018_02_20.xlsx ***
%Office of Management and Budget. 2000. ``Standards for Defining Metropolitan and Micropolitan Statistical Areas.'' \emph{Federal Register} 65, 249, pp 82228-822237. Office of Management and Budget. 2003.  Bulletin No. 03-04, June 6, 2003. https://www.whitehouse.gov/wp-content/uploads/2017/11/bulletins_b03-04.pdf, accessed May 2 2020


% the baseline delineations I typically use were released in June 2003 and were based exclusively on Census 2000 data. There was a subsequent revision in December 2003, which incorporated Census Bureau population estimates for 2001 and 2002.

% old New York, Northern New Jersey, Long Island CMSA
%  --> 20191.47 km2  * 0.386102= 7,796 mi2    (Q:\x\data_files_stata\primary\census\stf\1990\stf3c\stf3c1990_v10_idplus_01d.dta)
%  --> North north east to Middletown Connecticut; North through Duchess County; Northwest through Pike County PA, southwest  through Ocean County NJ;  (https://donsnotes.com/nyc-nj/ny-nj-ct-cmsa.html)

% Jersey City, sliver of across the Hudson river from the lower half of Manhattan and New York harbor.  along half of Manhattana
%   79km2 * 0.386102 = 30.5 mi2.

%Annual postcensal updates of statistical areas since 2003 have been extensive and have included: (1) Qualification of new micropolitan statistical areas; (2) qualification of new metropolitan statistical areas; (3) qualification of new and expanded combined statistical areas, (4) qualification of new principal cities; (5) deletion of principal cities; and (6) changes in the titles of metropolitan statistical areas, micropolitan statistical areas, and metropolitan divisions, based on the addition and/or deletion of principal cities as well as changes in the relative population size rankings of principal cities. (Office of Management and Budget (2010). ``2010 Standards for Delineating Metropolitan and Micropolitan Areas''. \emph{Federal Register} 75, 123, pp. 37246-37252.

Title Criteria: (Federal Register vol 65 No 249, December 27 2000)
Up to three principal cities in descending order of population (there is an exception if the largest principal city is a Census Designated Place)
Principal City (or Cities):
(a). the largest incorporated place with population of at least 10,000 in the CBSA (or in absence of that, largest incorporated place or Census Designated place in CBSA)
(b) any additional incorporated place or CDP with population of at least 250,000 or in which 100,000 or more persons work
(c) Any additional incorporated place or CDP with population of at least 50,000 and in which the number of jobs weakly exceeds the number of employed residents



\end{comment}


\begin{comment} additional CBSA descriptive characteristics

% In addition, let $\sigma$ designate the threshold strength of commuting ties required to join an outlying county to central counties.

%As suggested by these ranges, the population of numerous large micropolitan CBSAs exceeded the population of numerous small metropolitan CBSAs, reflecting that the differentiating criterion depended on the population of the core rather than the population of the whole.

% \footnote{A county is deemed to be central if it wholly contains a UA or UC with population of at least 10,000 or if at least 5,000 people in the county live in a UA or UC with population of at least 10,000. Two contiguous CBSAs are merged into one if the central county or counties of one qualify as outlying counties of the other.}

%The CBSA commuting criterion can be expressed more formally. Let $\mathbf{w}_i$ denote residents of location $i$ who work (somewhere), $\mathbf{e}_i$ denote employment in location $i$, and $\mathbf{f}_{i,j}$ denote the commuting flow of workers from $i$ to $j$. A potential outlying county, $i$, will be combined in a CBSA with a union of central counties $j$ if either $\mathbf{f}_{i,j}/\mathbf{w}_i >  0.25$ or $\mathbf{f}_{j,i}/\mathbf{e}_i > 0.25$.

\end{comment}


\begin{comment} Extra material on Riverside--San Bernadino, Greenwhich CT, Boston

%Census tracts with at least 500 persons per square mile \emph{or} at least 500 workers per square mile---a benchmark we use for metropolitan character---accounted for just 0.3 percent of its land area but 70 percent of its population and 77 percent of its employment. The even-larger is similarly composed almost entirely of unsettled desert. Even in the crowded New York City-Newark-Edison CBSA, census tracts with density above our metropolitan benchmark accounted for only 51 percent of land area (but 96 percent of population and 97 percent of employment).%Across the 88 metropolitan CBSAs with population of at least 500,000 in 2000, the mean share of land area with density above this metropolitan benchmark was just 19 percent. The corresponding mean shares of  population and employment were 80 percent and 88 percent. %.
% shares reported in dropbox\fma\workbooks\fma.cbsa2003_density.2020_04_20a.duplicate.xlsx


%Third, CBSA delineations underbound portions of some conceptual metropolitan areas. For example the New York City-Newark-Edison CBSA as delineated in 2000 excluded the Greenwich and Stamford CT municipalities, which are connected to midtown Manhattan by commuter railroad and are widely thought of as New York City suburbs. Instead, Greenwhich and Stamford are included in the neighboring Bridgeport-Stamford-Norwalk CT CBSA, which is composed of Fairfield County, CT. Similarly, census blocks and block groups in the Boston UA spill out of the Boston CBSA into the Worcester, Manchester-Nashua, and Providence CBSAs.

\end{comment}



\subsection{Additional Methodologies }

A third set of possible proxies for U.S.\ metropolitan areas, Commuting Zones (CZs), were delineated by the U.S. Department of Agriculture's Economic Research Service following each of the 1980, 1990, and 2000 decennial censuses with the goal of better understanding rural labor markets (\citeauthor{Tolbert_Sizer_1987}, \citeyear{Tolbert_Sizer_1987}, \citeyear{Tolbert_Sizer_1996}; \citeauthor{Foote_etal}, \citeyear{Foote_etal}). Like CBSAs, CZs are constructed using counties as geographic building blocks. The commuting strength between two locations is calculated as the sum of the commuting flows in both directions normalized by the level of employment in the location where it is smaller. The delineation algorithm begins by constructing an N-by-N symmetric matrix of commuting strength between all possible pairs of counties; the pair with the strongest tie are joined, constituting a first cluster. The process is iteratively repeated, calculating a new (N-1)-by-(N-1) symmetric matrix and joining the pair of observations with the strongest tie until it falls below a judgmental threshold. The resulting clusters fully partition the United States.

%The CZ clustering algorithm starts by calculating an  (\citeauthor{Tolbert_Sizer_1987},\citeyear{Tolbert_Sizer_1987}, \citeyear{Tolbert_Sizer_1996}; \citeauthor{Foote_etal},\citeyear{Foote_etal}). Strength between counties $i$ and $j$, $s_{i,j}$ is calculated by the sum of the worker flows in both directions divided by employment in the smaller location: $({f}_{i,j} + {f}_{j,i})/\min({e}_i,{e}_j)$. The county pair with the strongest commuting tie constitutes a first cluster and then the process is repeated, calculating a new (N-1)-by-(N-1) symmetric matrix and joining the pair of observations (single counties or clusters of previously joined counties) with the strongest tie. The iteration continues until the remaining strongest commuting tie falls below a judgmental threshold.

CZs share many of the same flaws as CBSAs, including over-bounding and under-bounding. A large share of CZs encompass one, two, or three CBSAs together with a handful of adjacent rural counties. Even so, many medium and large CBSAs are split across multiple CZs, with groups of unambiguous suburbs separated from central business districts. As illustrated in an appendix, six CZs in the vicinity of New York City are connected to Manhattan by commuter rail.
% Applying the algorithm to data from the 2000 decennial census delineates 709 CZs with population ranging from 14,000 to 16.4 million and land area extending up to a maximum of 166,000 square miles.

\begin{comment} % Old Commuting Zone Subsection and additional details
\subsection{Commuting Zones}

The Economic Research Service (ERS), a division of the U.S. Department of Agriculture, delineated Commuting Zones following each of the 1980, 1990, and 2000 decennial censuses. Like CBSAs, CZs are constructed using counties as geographic building blocks. But in contrast to CBSAs, CZs are conceived as a full partition of the United States into areas within which people live and work, a key goal being to understand labor market characteristics of rural areas.

The CZ clustering algorithm starts by calculating an N-by-N symmetric matrix ranking the strength of commuting ties between all possible pairs of counties. The county pair, $i$ and $j$, with the strongest tie as measured by $({f}_{i,j} + {f}_{j,i})/\min({w}_i,{w}_j)$, is joined. The process is then repeated, calculating a new (N-1)-by-(N-1) symmetric matrix and joining the pair of counties or joined clusters of counties with the strongest commuting tie. This continues iteratively until the remaining strongest commuting tie falls below a judgmental threshold, $\sigma$  \citep{Tolbert_Sizer, Foote_etal}.\footnote{\cite{Tolbert_Sizer} used a stopping threshold, $\sigma$, of 0.02 to construct the 1980 and 1990 commuting zones. Because of computing limitations at the time, they implemented their algorithm separately for six overlapping sets of U.S. states and then used judgment to reconcile results. \cite{Foote_etal} argue that a stopping threshold of 0.0635 most closely replicates the 1990 delineation when the algorithm is run for all U.S. states simultaneously and so use it for the 2000 delineation.}

Applying this algorithm to data from the 2000 decennial census delineates 709 CZs with population ranging from 14,000 to 16.4 million and land area extending up to a maximum of 166,000 square miles, six times that of the geographically largest CBSA. A large share of CZs encompass one, two, or three CBSAs together with a handful of adjacent rural counties. Even so, many medium and large CBSAs are split across multiple CZs, with groups of unambiguous suburbs separated from central business districts. For example, four of the five CZs that split the New York--Newark--Edison CBSA are connected to Manhattan by commuter rail. %Contributing to this disparity in land area, the construction algorithm for CZs lacks a criterion to prevent chaining.

\end{comment}


\begin{comment} Some Details on Commuting Zones

% C:\jordan\X\identification\commuting_zones\cz00_eqv_v1.downloaded_2020_03_24.2020_09_13a.xlsx has mapping

These are 31 metropolitan CBSAs split across multiple CZs: some micropolitan ones also split into multiple CZs
Largest CBSAs, with multiple Metropolitan Divisions split into mutliple CZs
Boston CBSA into two CZs (corresponding to metro divisions)
Chicago CBSA into two CZs (corresponding to metro divisions)
Dallas CBSA into three  (split across two metro divisions)
NY CBSA into five CZs
Phillie CBSA into three CZs
Washington D.C. CBSA into four CZs
Allentown CBSA into 2 CZs
Atlanta CBSA into 7 CZs
Baltimore into 2 CZs
Charlotte NC-SC CBSA into 2 CZs
Cincinatti into 2
Columbia SC into 2
Columbus GA-AL into 2
Columbus OH into 3
Evansville IN-KY into 2
Fort-Smith AR-OK into 2
Houston into 2
Kansas city into 4
Lexington KY into 2
Louisville KY into 4
Memphis into 2
Minneapolis into 2
Nashville-Davidson into 2
Peoria, IL  into 2
Portland ME into 2
Richmond VA into 4
Sacramento into 2
Sioux-City IA-NE-SD into 2
St. Joseph MO into 2
Toledo OH into 2
Tulsa into 2
Virginia Beach-Norfolk-Newport News into 2


%  our workbook: ...\Dropbox\FMA\workbooks\fma.commuting_zones.2020_05_10a.xlsx
%  *  -->  constructed from: Q:\x\identification\commuting_zones\workbooks\cz00_eqv_v1.downloaded_2020_03_24.2020_05_01a.xlsx
%  *  -->  "Commuting Zones and Labor Market Areas" downoaded from https://www.ers.usda.gov/data-products/commuting-zones-and-labor-market-areas/ on 2020-03-24
%  *  -->  https://www.ers.usda.gov/webdocs/DataFiles/48457/cz00_eqv_v1.xls?v=0; file is identified as last updated on 2/22/2012


% Foote, Andrew, Mark J. Kutzbach, and Lars Vilhuber (2017). ``Recalculating... : How Uncertainty in Local Labor Market Definitioins Affects Empirical Findings.'' U.S. Census Bureau, Center for Economic Studies, Discussion Paper CES 17-49.

% Tolbert, Charles M., and Molly Sizer (1996). ``US Commuting Zones and Labor Market Areas: a 1990 update.'' ERS Staff Paper 9614. United States Department of Agriculture, Economic Research Service.

% Tolbert, Charles M., and Molly Sizer Killian, (1987). ``Labor Market Areas for the United States.'' Staff Reports 277959, United States Department of Agriculture, Economic Research Service.

% Fowler, Christopher S., Danielle C. Rhubert, and Leif Jensen, (2016). ``Reassesing and Revising Commuting Zones for 2010: History, Assessment, and Updates for U.S. `Labor-Sheds 1990-2010.'' \emph{Population Research and Policy Review} 35, 2, 263-286.

%  \item Commuting Zones
%  \begin{itemize}
%    \item delineation criteria: From 1980-2000, the USDA Economic Research Service delineated commuting zones to reflect the areas where people live and work across the U.S. The ERS used hierarchical cluster analysis to group counties into commuting zones, to maximize intra-commuting zone flows and minimize flows between commuting zones. The set of 709 commuting zones fully partition the United States land area; the majority of commuting zones do not cover densely settled land that would constitute a metropolitan area.
%    \item papers that use it: \citet{Autor_Dorn_Hanson}, \citet{Chetty_etal_2014}
%    \item Since commuting zones consist of counties, they cannot provide a granular estimate of metropolitan land coverage. Some CZs are larger than CBSAs, other CZs split up large CBSAs.
%  \end{itemize}


\end{comment}

Expanding beyond the United States, the European Union and the OECD delineate Functional Urban Areas in their member countries \citep{Dijkstra_Poelman_2012, Brezzi_etal, Dijkstra_etal_2019}. The first of four stages joins adjacent grid cells of 1 square kilometer into the same cluster if each has population density of at least 1,500 persons per square kilometer (3,885 persons per square mile). Resulting clusters that have population of at least 50,000 are considered \emph{urban centres}. These are analogous to U.S.\ Urbanized Areas except that they have a density threshold almost eight-fold higher and so exclude considerable suburban area.  The second stage combines the governmental administrative units that overlap each urban center, forming \emph{core cities}. The third stage joins core cities between which commuting ties exceed a threshold strength, in essence creating kernels. The final stage builds out from the kernels, attaching local government administrative units with which commuting ties exceed a threshold strength.\footnote{The European Union, OECD, the United Nations, and The World Bank jointly adopted an urban-rural classification system that complements the EU/OECD delineations \citep{eurostat_etal_2021, Dijkstra_etal_jue_2021}. Countries are gridded into cells of 1 square kilometer, each of which is assigned to one of three categories: an \emph{urban centre}, constructed as described; an \emph{urban cluster}, contiguous grid cells, not in urban centres, that have population density of at least 300 persons per square kilometer (777 persons per square mile) and total population of at least 5,000; and \emph{rural grid cells}, all cells not in an urban centre or urban cluster. These are respectively characterized as ``densely populated'', ``intermediate density'', and ``thinly populated''.}

%! the threshold strength between administrative units in core cities is measured by having a 15 percent outbound flow in either direction. The threshold for joining outlying administrative units is having an 15 percent outbound flow.

\cite{duranton_2015} goes a step beyond the examples above, granularly delineating metropolitan areas that are multilevel hierarchies. The directional strength between locations is measured by the commuting outflow rate, calculated as the flow from origin to destination relative to the number of workers residing in the origin. A first round of joins classifies each location as subsidiary to the one with which it has the highest outflow rate, contingent on the rate exceeding a judgmental threshold.  The resulting first-round clusters mix hub-and-spoke (B and C each attached to A) and chained configurations (C attached to B and B attached to A). These are then iteratively used to effect a second round of joins, attaching each first-round cluster to the one with which it has the strongest outflow rate. The iteration continues until no outflow rate exceeds the threshold. Applying the algorithm to U.S.\ counties constructs hierarchical clusters whose population is tightly correlated with that of matched CBSAs \citep{dingel_miscio_davis_2021}.
% In cases where the strongest outbound flows from each of two locations are with each other, the smaller location by population is attached as a satellite to the larger one.

\begin{comment} additional text on Duranton

% Applying the algorithm to municipalities in Columbia using a baseline threshold $\sigma$=0.10 delineates 84 metropolitan clusters with population in 2010 of at least 50,000. Using alternative thresholds, $\sigma$=0.05 and $\sigma$=0.20, delineate metropolitan clusters relatively similar to the baseline ones.


%Geographic building blocks are iteratively joined if origin-destination commuting flows normalized by workers residing in the origin, ${f}_{i,j}/{w}_i$, exceed a specified threshold, $\sigma$. When an origin has normalized outbound flows that exceed $\sigma$ to several destinations, it joins with the destination to which the outbound flow is largest. The joins are hierarchical: if ${f}_{i,j}/{w}_i$ and ${f}_{j,i}/{w}_j$ both exceed $\sigma$, the smaller location by population joins to the larger one. A first round of such joins creates clusters made up of multiple locations, combining hub-and-spoke (B and C each joined to A) and chained configurations (C joined to B and B joined to A). Flows among these resulting clusters are used to effect a second round of joins, with the procedure repeated iteratively until no further joins are possible. Doing so identifies a hierarchical set of cores within each metropolitan area rather than pre-specifying a core (CBSAs) or forgoing one (UAUCs and CZs). Applying the algorithm to U.S.\ counties using a threshold $\sigma \in [0.20, 0.25]$ approximately replicates U.S.\ metropolitan CBSAs \citep{dingel_miscio_davis_2021}.  %An arguable weakness of the algorithm, destinations where a large share of jobs are held by a metropolitan cluster's residents may not be included in that metropolitan cluster. For example, almost all workers living in the destination may also work there. (NOT SURE WHY THIS IS WEAKNESS)
% Applying the algorithm to municipalities in Columbia using a baseline threshold $\sigma$=0.10 delineates 84 metropolitan clusters with population in 2010 of at least 50,000. Using alternative thresholds, $\sigma$=0.05 and $\sigma$=0.20, delineate metropolitan clusters relatively similar to the baseline ones.

\end{comment}


Reliable commuting data is not available for most nations, and so delineating metropolitan areas must rely on other approaches. One uses satellite images of nighttime lights and land cover. For example, \citet{ch_martin_vargas_2021} identify clusters of adjacent pixels with nighttime light intensity above alternative thresholds to delineate between 4,200 and 6,700 metropolitan areas worldwide. \citet{dingel_miscio_davis_2021} similarly use nighttime light to cluster pixels, which they match with intersecting administrative units. Using U.S.\ counties as administrative building blocks, the implied unions have population that is tightly correlated with the population of matched CBSAs across a range of light-intensity thresholds, validating using the methodology for some purposes when commuting data is not available.

Another approach focuses on physical structures. \cite{de_bellefon_etal_2021} classify small grid cells in France as urban if the density of buildings in them exceeds a threshold percentile across all grid cells in a country and then cluster the grid cells to construct urban areas.  Similarly, \citet{arribas-bel_etal_jue_2021} apply a spatial clustering algorithm to the precise location of all buildings in Spain to construct urban areas.

%Applied to France, the de_bellefon algorithm delineates 5,771 urban areas in 2014, with population ranging from 0 (all buildings are non residential) to more than 11 million.
% arribas et al delineating 717 urban areas in 2017 with population ranging from less than 5,000 to more than 4.5 million.

A third approach relies more directly on judgment. \cite{galdo_li_rama_2021} ask several groups of assessors to use Google Earth and Google Maps to classify a sample of locations in India as either urban or rural. A machine learning algorithm then uses several measurable characteristics of the locations---such as population, population density, land cover, and nighttime luminosity---to predict the urban/rural status of more than 500,000 villages, towns, and cities.



%Both \citet{de_bellefon_etal_jue_2019} and \citet{arribas-bel_etal_jue_2019} trade off the ability to access detailed socioeconomic data using administrative units as building blocks in favor of the geographic precision from using small grid cells as building blocks.

%\emph{Discuss Galdo, Li, Rama, JUE, 2021}.
%Still another approa relies more directly on human judgment rather than indirectly on it via the choice of criteria.  This is the approach taken by \citet{galdo_li_rama_jue_2019}, who label a representative sample of places in India as either urban or rural based on crowd-sourced judgments of images. A machine learning algorithm then uses these labels along with administrative and satellite imagery to predict the urban status of more than 500,000 villages, towns, and cities throughout India.

\begin{comment} Rozenfeld et al., which I now include as footnote accomanying description of UAs (
measurement accuracy; adaptation to using census tracts
%\footnote{Measurement accuracy is an important concern when using finely gridded population data. Allocating population measured for actual geographic units to a regular grid requires arbitrary assumptions, such as population being spread homogeneously within each measured unit.}
%An algorithm adapted to using U.S. census tracts as building blocks, which can handle their irregular shapes and widely varying land areas, is applied to the U.S. using $n^*=0$ and $\ell \in  [1,10]$ km.  For $\ell \in  [2.5,6]$ km, the clusters with highest population approximately correspond to metropolitan CBSAs.
%%\footnote{The adapted algorithm supplants the adjacency criterion with the requirement that joined tracts must have centroids within $\ell$ distance of each other. Census tract land area and population density are inversely correlated, reflecting that tracts are endogenously delineated to have target population between [1,500 in 2000| 1,200 in later years] and 8,000. In consequence, the algorithm joins tightly spaced, high-density tracts in metropolitan centers with increasingly large geographic tracts moving outward through moderate-density inner suburbs and low-density outer suburbs. With $n^*=0$, the resulting metropolitan clusters terminate where population density has fallen sufficiently and so tract land area has increased sufficiently to separate adjacent centroids by at least $\ell$.}

Additional details on parallel versus UAUC construction
%In addition to using census tracts rather than census blocks, the adapted algorithm drops the minimum thresholds of 500 persons per square mile for population density and 2,500 for the resulting clusters' total population; rather than limiting separating distances between the borders of qualifying census blocks, it limits the separating distance between tract centroids.\footnote{Population density remains an implicit criteria because tracts are endogenously drawn to include at least 1,500 people (1,200 beginning in 2010) and so tract centroids will be relatively far from each other in sparsely settled areas.} A baseline parameterization using a maximum separating distance of 3 km yields just under 23,500 clusters, of which just under 2,000 have population of 12,000 or more.
% \footnote{The zero density threshold implies that clusters end where they border grid cells in which no one lives.}

\end{comment}


\begin{comment} lots of material

%\citet{Bosker_etal} compare 4 approaches to defining metropolitan areas in Indonesia - 3 'satellite' based approaches using either satellite images of nighttime lights or the built environment, and 1 commuting based approach. They find that the 3 satellite based approaches typically define fewer metropolitan areas either because more metro areas are grouped together in high agglomeration areas or the areas do not meet the population or built up thresholds in areas with low agglomeration but sufficient inter-area commuting.


use raw population data measured on a grid of 0.2-by-0.2 km cells to partition the United Kingdom into population clusters. Their algorithm starts by combining the fine grid cells into geographic building blocks with dimension $\ell$ by $\ell$, $\ell \in  [0.2,2.6]$ km. The resulting blocks are eligible for clustering if they have population density, $D > D*$. Eligible blocks are joined in the same cluster if they are adjacent. With $D^*=0$, the clusters with highest population implied by a range of values for $\ell$ approximately correspond to the major U.K. regions.\footnote{The authors are ambivalent on the ``correct'' value of $\ell$, reflecting their focus on the considerable robustness of population and growth distributions across alternative distances.}

The raw population data for the United States are measured at the level of census tracts, which can have irregular shapes and vary in land area by orders of magnitude. Adapting the algorithm to handle this irregularity and wide variation, the adjacency criterion is supplanted by the requirement that joined tracts have centroids within $\ell$ distance of each other, $\ell \in  [1,10]$ km . The Census Bureau delineates tracts to have a target population between 1,200 and 8,000 [1,500 and 8,000 in 2000], inducing a negative correlation between tract land area and tract population density. In consequence, the algorithm joins tightly spaced, high-density tracts in metropolitan centers with increasingly large geographic tracts moving outward through moderate-density near suburbs and low-density distant suburbs. With $D^*=0$, the resulting metropolitan clusters terminate where population density has fallen sufficiently and land area has increased sufficiently to separate adjacent centroids by at least $\ell$. For $\ell \in  [2.5,6]$ km, the clusters with highest population approximately correspond to the metropolitan CBSAs delineated following the 2000 decennial census.
%! I had a hard time writing immediately above


%start with fine grids of population measuring $\ell$ by $\ell$. Cells are candidates for inclusion in a cluster if they have population density $D >  D^{*}$. The integration criterion is adjacency. The authors apply this algorithm to partition the United Kingdom, using $D^{*}= 0$ and kilometer scales ranging from $\ell = 0.2$ up to $\ell = 2.6$. For a range of these scales, the implied clusters with the largest population approximately correspond to U.K. urban regions. The underlying data for the United states is based on census tracts, which can have irregular shapes and endogenous land area ranging from \textbf{0.xx} $\mbox{km}^2$ up to more than 10,000 $\mbox{km}^2$ (Tracts are delineated to have a target population between 1,200 and 8,000, inducing a negative correlation between land area and population density). Adapting the algorithm to handle this variation, the adjacency criterion is supplanted by the requirement that joined tracts have centroids within $\ell$ of each other. In metropolitan locations, the adapted algorithm combines tightly packed small tracts within the densely populated center with increasingly large tracts moving outward through the moderate-density near suburbs and low-density distant suburbs. The resulting metropolitan clusters terminate where population density has fallen sufficiently and so land area has increased sufficiently to separate adjacent centroids by $\ell$.  Using $D^*=0$ and a range of scales from $\ell = 1$ km to $\ell = 6$ km, the

%partitions the U.S. in 2000 into approximately 23,500 clusters, the largest of which geographically approximate CBSAs.\footnote{The authors are mostly agnostic on the appropriate values of $D^*$  and $\ell$, reflecting a research focus on the robustness of population size and growth distributions alternative choices}.

% and the United States using kilometer scales $\ell \in \{1,\, 2 ,\, 3\}$. For selected scales, the implied clusters with population in the upper tail approximately correspond to metropolitan delineations. For example, applying the algorithm with a 3 km grid partitions the U.S. in 2000 into approximately 23,500 clusters, the largest of which geographically approximate CBSAs.\footnote{The underlying data for the U.S. is census tracts, which can have irregular shapes and, outside UAs/UCs, land area of more than \textbf{x,000} square kilometers. To handle this, the authors drop the adjacency criterion. Instead, tracts with $D > D^*$ are joined if their centroids are located within $\ell$ of each other. Tracts with sufficiently large land area will have no neighbowithin $\ell$ and so will constitute single-tract clusters. As tracts

% Census tracts are endogenously delineated to have population between 1,200 and 8,000 and so have land area inversely proportional to population density.\footnote{\textbf{Summary of actual population of census tracts in 2000}} In metropolitan locations, the algorithm combines tightly packed tracts in the densely populated center with increasingly large tracts moving outward through the near and distant suburbs. The resulting metropolitan clusters terminate where tracts' land area has increased sufficiently to separate adjacent centroids by $\ell$.  Setting $\ell = 3$km partitions the U.S. in 2000 into approximately 23,500 clusters, the largest of which geographically approximate CBSAs.


% Introduce as a general algorithm that is also applied to U.K. and Africa. [At a granular level], \citet{Rozenfeld_etal_PNAS} and \citet{Rozenfeld_etal} combine census tracts with positive population that have geographic centroids within $\ell$ distance of each other. Census tracts are endogenously delineated to have population between 1,200 and 8,000 [1,500 and 8,000 in 2000] and so have land area inversely proportional to population density.\footnote{\textbf{Summary of actual population of census tracts in 2000}} In metropolitan locations, the algorithm combines tightly packed tracts in the densely populated center with increasingly large tracts moving outward through the near and distant suburbs. The resulting metropolitan clusters terminate where tracts' land area has increased sufficiently to separate adjacent centroids by $\ell$.  Setting $\ell = 3$km partitions the U.S. in 2000 into approximately 23,500 clusters, the largest of which geographically approximate CBSAs.


\begin{itemize}
  \item Data sources other than commuting: Economists and geographers propose alternative ways to delineate metropolitan areas in order to systematically compare urban settlement patterns and metropolitan growth across countries or to overcome issues of limited data. Trade off in identifying agglomerations of people vs. agglomerations of human activity vs. labor market activity - are they sufficiently similar?
	\begin{itemize}
		\item population data:
		\item building location data:
		\begin{itemize}
			\item \citet{de_Bellefon_etal} use detailed building location data in France to delineate urban areas based on building density. Their approach allows them to identify the urban 'core' within the urban areas.
			\item \citet{Arribas-Bel_etal} apply a machine learning algorithm to group locations with high levels of building density into urban areas using the geolocation data of all buildings in Spain. They also identify employment centers within the urban areas and calculate the 'vertical land' of the urban areas using building heights. They conclude that their delineations are similar to commuting-based definitions (rather than administrative boundaries), yet more precise.
		\end{itemize}
		\item nighttime lights data:
		\begin{itemize}
			\item \citet{Dingel_etal} define metropolitan areas in Brazil, China, and India by aggregating areas of contiguous light based on nighttime satellite data.
		\end{itemize}
		\item Google Maps and Google Earth images
		\begin{itemize}
			\item \citet{Galdo_etal} classify places in India using machine learning and collective human judgements based on street view and satellite images from Google Maps and Google Earth.
		\end{itemize}
		\item mixed data:
		\begin{itemize}
			\item \citet{Moreno-Monroy_etal} define commuting zones around urban cores using population and travel data of geographic grids across the world. They train a logistic regression model using the OECD Functional Urban Area definitions to estimate the probability that cells outside a core urban area, defined as a contiguous set of grid cells with high population density, belong to that metropolitan area.
			\item \citet{Bosker_etal} compare 4 approaches to defining metropolitan areas in Indonesia - 3 'satellite' based approaches using either satellite images of nighttime lights or the built environment, and 1 commuting based approach. They find that the 3 satellite based approaches typically define fewer metropolitan areas either because more metro areas are grouped together in high agglomeration areas or the areas do not meet the population or built up thresholds in areas with low agglomeration but sufficient inter-area commuting.
		\end{itemize}
	\end{itemize}
  \item Alternative methods
	\begin{itemize}
		\item No core is identified as first step:
		\begin{itemize}
			\item \citet{duranton_2015} proposes an algorithm to aggregate spatial units based on commuting flows that meet a constant threshold. \citet{duranton_2015} applies the algorithm to administrative data in Colombia, where the algorithm endogenously identifies metropolitan areas without first anchoring them to an urban core.
		\end{itemize}
		\item Consider intra-CBSA commuting flows to identify polycentric metropolitan areas
		\begin{itemize}
			\item \citet{Tong_Plane} consider the commuting flows between counties in the same CBSA in order to identify polycentric urban structures. They also test an alternative algorithm that joins micropolitan and metropolitan areas with nearby metropolitan areas based on the inter-county commuting flows to further investigate the how many U.S. metros contain several urban centers within one metropolitan area (in contrast with the Census Bureau which only considers inter-CBSA flows when joining two CBSAs together).
		\end{itemize}
	\end{itemize}
\end{itemize}

\end{comment}

\section{Constructing KBMAs}


%As described in the introduction, we define a metropolitan area as a union of nearby, built-up locations with combined population of at least moderate scale and within which a significant share of residents and workers travel on a day-to-day basis among places of residence, places of employment, and places of consumption. Implicitly, most people who live, work, or consume in a metropolitan area do not travel outside it on a day-to-day basis.

Our metropolitan definition is purposely imprecise, leaving scope for judgment on whether specific delineations are consistent with it. Alternative parameterizations of the KBMA algorithm can match a wide range of judgments on encompassing places of residence and employment that are near each other while excluding lightly settled locations. Like the methodologies described above, no explicit component is included to encompass places of consumption.


\subsection{Algorithm}


We use census tracts, UAs, and UCs as our building blocks, taking advantage of carefully measured commuting flows among them. As with CBSAs, UAs and UCs with population above 10,000 serve as the cores of KBMAs. The first stage of the algorithm iteratively joins these cores together into kernels, paralleling the iterative construction of CZs. Doing so combines many of the UAs that are split by backward alignments. The second stage builds out from the kernels, attaching outlying tracts tied to them by sufficiently strong commuting flows.

We symmetrically measure the strength of the commuting tie between two locations by the sum of the inflow and outflow rates in both directions. Let $f_{i,j}$ represent the gross commuting flow from location $i$ to location $j$; let $e_i$ and $w_i$ respectively represent employment in location $i$ and workers residing there. The commuting strength between $i$ and $j$  (and between $j$ and $i$) is given by,
\begin{equation} %strength for joins of cores
\label{strength4}
{s}_{i,j}, {s}_{j,i} = \frac{{f}_{i,j}}{{w}_{i}} + \frac{{f}_{i,j}}{{e}_{j}} + \frac{{f}_{j,i}}{{w}_{j}} + \frac{{f}_{j,i}}{{e}_{i}}
\end{equation}
Measuring strength symmetrically abstains from imposing a hierarchy. Normalizing directional flows both by resident workers and employment equally weights the importance of a location serving as a source of labor supply and as a source of labor demand.

Pragmatically, summing the four terms rather than taking the maximum (the CBSA formula) avoids a bias against combining locations of similar size. The normalization of the gross flows yields relatively low values for all four terms when two locations have similar size compared to the maximum value of the four terms when one location is much larger than the other. Correspondingly, we judge that appropriately delineating metropolitan areas when measuring strength by the maximum value would require setting a much lower threshold, $\sigma$, for joining two similar locations than for joining two unequal ones.%For example, the maximum of the normalized flows between the Raleigh and Durham UAs is just 0.10., far below the threshold of 0.25 required to join counties into the same CBSA. But setting a threshold of 0.10 implies numerous joins of UAs we judge unambiguously belong to separate metropolitan areas.

The initial iteration of the first stage begins by constructing an N-by-N symmetric matrix of the commuting strength between all pairs of cores. The pair with maximum strength,  subject to having separating distance no more than $\delta$, is joined together and then a new (N-1)-by-(N-1) matrix is constructed. Iterating continues until the maximum pairwise strength drops below $\sigma$.

The second stage attaches outlying tracts to a kernel if the strength of their commuting tie with the kernel weakly exceeds $\sigma$, their distance from the kernel does not exceed $\delta$, and either their population or employment density weakly exceeds $\eta$.

We rely almost exclusively on one source of data, the Census Transportation Planning Package 2000 \citep{ctpp2000}. It re-tabulates responses to the journey-to-work questions on the long form of the 2000 decennial census by tract of employment and by origin-destination pairs.\footnote{A question on the 2000 decennial census long form and subsequent American Community Surveys asks respondents to give the address where they primarily worked the previous week. An alternative source of origin-destination commuting flows, the Census Bureau's LEHD Origin-Destination Employment Statistics (LODES), determines workplaces by the address of the business establishment with which employees are associated for payroll reporting. These establishment locations frequently differ from actual work locations. In addition, the LODES dataset does not cover non-payroll employment, such as self-employment.} %In addition, tract land area and centroid coordinates come from the geographic header files of the 2000 Decennial Census' Summary File 1, which are available for download from the Census Bureau. We also use Geocorr2000 \citep{geocorr2000}, a geographic correspondence engine, to identify the overlap in population and land area between census tracts and UA/UCs.
% Separately, a more recent CTPP retabulates responses to the 2012-16 American Community Surveys. An advantage of 2000 CTPP is that it has a sample size three times larger (based on the long form of the 2000 census) and so more comprehensively measures commuting flows, especially among sparsely settled tracts.

As a prerequisite to implementing our algorithm, we construct tract-based approximations of UAs and UCs with population above 10,000. Most census tracts are either fully overlapped by a UA/UC or else fully disjoint with any UA/UC; even so, many tracts partly intersect with a UA/UC. As described in an appendix, we use a fit criterion to set threshold shares of a tract's population and land area that must intersect with a UA/UC to include the tract in that UA/UC's approximation. Only 1,274 of the 1,374 UAs and eligible UCs are actually approximated,  reflecting that 100 of the eligible UCs have no intersecting tract that meets both the population and land-area thresholds. As described in the appendix, this shortfall only negligibly affects delineations.

%Specifically, we set these shares to minimize the sum of squared deviations for population and land area  between the approximated UA/UCs and the actual UA/UCs. In consequence, 100 of the 922 UCs with population above 10,000 have no intersecting tracts that meet both the population and land area thresholds and so are not approximated. We argue in an appendix that this failure is likely to only modestly affect the KBMA delineations. The 822 approximated UCs and 452 approximated UAs together constitute 1,274 cores.


\begin{comment} possible difference in sensitivity to sigma vs to sigma''.
%BUT I'M NOT SURE THIS IS TRUE Moreover, the ability to separately vary each of the strength and distance thresholds can give insight. For example, the implicit elasticity of land area with respect to population becomes larger as the threshold commuting strength to join two or more cores in the same metropolitan area is relaxed. In other words, land area becomes more responsive to changes in population as we make it easier to combine arguably separate metropolitan areas. But land's responsiveness is relatively insensitive to the threshold strength for joining outlying tracts.
\end{comment}


\begin{comment} Some additional examples of desired implausible parameterizations


% Other comparisons across local labor markets may be best served by setting relatively tight thresholds for joins.

%we construct KBMAs that we judge fall short of having ``moderate scale'' to test whether size distributions misleadingly cut off a lower tail. (We argue that they do not.)

%As mentioned in the introduction, estimating the strength of agglomeration may suggest dropping the distinction between large and small UAUCs in order to avoid truncating the estimating sample. Similarly, measuring the longest commutes  may suggest dropping limitations on separating distances. And measuring the centralization of employment may suggest dropping the density requirement for outlying census tracts.

\end{comment}




\begin{comment}  previous description of algorithm
%The CTPP 2000 constitutes three sets of tabulations: by place of residence, by place of work (respondents are asked for their workplace mailing address), and by origin-destination pairs. For each of these, weighted counts are reported for census tracts and for a second geography chosen by each state's department of transportation, typically census block groups or locally delineated transportation analysis zones.\footnote{State transportation departments have funded two more recent CTPP retabulations, based on the 2006--2010 and 2012--2016 American Community Surveys. One reason we choose to use the 2000 is that its sample size is more than three times as large. Another is that it is built from the same journey-to-work responses used to delineate CBSAs following the 2000 decennial census, implying that any differences between the two metropolitan proxies reflect methodology rather than data. The Census Bureau also publishes annual data on commuting flows between census blocks based on administrative records, the LEHD Origin-Destination Employment Statistics (LODES), which we plan to use in future research.}

%This algorithm endows KBMAs with several advantages compared to UAs, metropolitan CBSAs, and CZs. existing metropolitan proxies. Using census tracts as building blocks allows for a much closer geographic bounding of conceptual metropolitan areas than is possible with counties while also allowing for access to a much wider range of socioeconomic data than is possible with census block groups and blocks. Constructing kernels as combinations of UA and UC cores grounds KBMAs in a familiar delineation. Using commuting flows as an integration criteria closely ties the delineation to the underlying conception. Alternative parameterizations can establish the robustness of empirical results to idiosyncratic judgments.
\end{comment}

\begin{comment}  weaknesses of our algorithm
%Our algorithm also suffers from several weaknesses. First, it builds using statistical units specific to the United States. To the extent that commuting flow data are available only for larger geographic units, the algorithm may end up considerably overbounding conceptual metropolitan areas. Second, considerably less data is publicly available for census tracts than for counties. Third, the Census Bureau significantly redraws tracts prior to each decennial census, making it a considerable challenge to measure changes in characteristics over time frames longer than 10 years. Fourth, there is no explicit criterion tying delineation to places of consumption. This may lead to the exclusion of census tracts at metropolitan peripheries that host market and non-market amenities such as sports venues, shopping centers, and parks.
\end{comment}

\begin{comment} Previous subsection on data

Our main data source is the Census Transportation Planning Package 2000 \citep{ctpp2000}, a retabulation of responses to the journey-to-work questions on the long form of the 2000 decennial census. The CTPP 2000 constitutes three sets of tabulations: by place of residence, by place of work (respondents are asked for their workplace mailing address), and by origin-destination pairs. For each of these, weighted counts are reported for census tracts and for a second geography chosen by each state's department of transportation, typically census block groups or locally delineated transportation analysis zones.\footnote{State transportation departments have funded two more recent CTPP retabulations, based on the 2006--2010 and 2012--2016 American Community Surveys. One reason we choose to use the 2000 is that its sample size is more than three times as large. Another is that it is built from the same journey-to-work responses used to delineate CBSAs following the 2000 decennial census, implying that any differences between the two metropolitan proxies reflect methodology rather than data. The Census Bureau also publishes annual data on commuting flows between census blocks based on administrative records, the LEHD Origin-Destination Employment Statistics (LODES), which we plan to use in future research.}

We also use Geocorr2000 \citep{geocorr2000}, a geographic correspondence engine, to match census tracts with UAs and UCs. More specifically, it gives population and land allocation factors for tracts that are only partly encompassed within a UA or UC.

%\footnote{The tabulation by census tract of residence differs from other decennial census count data only in that it breaks out responses into more detailed bins.}
\end{comment}





\subsection{Parameterization}


Parameterizing the threshold commuting strength and maximum allowed separating distance to join locations and the threshold density of outlying tracts to attach them to kernels rests heavily on judgment. Any set of values will inevitably contradict many people's priors, either because of specific implied delineations or because of judgements of what qualifies as sufficiently integrated, near, and built-up. Alternative parameterizations illuminate how judgments affect delineations.
% ---$\sigma$, $\delta$, and $\eta$---

We judgmentally parameterize KBMAs to balance encompassing commuting flows and excluding sparsely settled land, while resting only lightly on separating distance to constrain joins. As such, KBMAs serve as a baseline likely to match most reasons for using metropolitan delineations. We also delineate more expansive kernel-based metropolitan regions, setting lower values for threshold commuting strength and density and a higher allowed separating distance, and more compact kernel-based urban areas, setting higher vales for threshold strength and density and a lower allowed separating distance.  %The former may better match studying extended connections, such as supply chains and long-distance commuting; the latter may better match studying narrow geographic spillovers and urban land use.

The kernel portion of the construction depends only on the commuting strength and distance parameters, $\sigma$ and $\delta$. The composition of many kernels is especially sensitive to the parametrization of $\sigma$, which we set to 0.25 based on our judgment of the implied unions. To span a wide range of judgments, we set $\sigma$ much lower, to 0.10, for kernel-based metropolitan regions and much higher, to 0.40, for kernel-based urban areas. We set the maximum allowed separating distance, $\delta$, to 20 miles, measured between the nearest tract centroids, consistent with our judgment of ``near'' and sufficiently high that it blocks relatively few joins during the kernel iteration.\footnote{The tract centroids we use are the \emph{internal points} reported by the Census Bureau. Some irregular-shaped tracts have geographic centroids that lie outside their physical territory, e.g. crescent shapes and tracts split by a water body. In these cases, the Census Bureau locates the internal point in the tract portion nearest the geographic centroid.} Our alternative metropolitan region and urban area parameterizations respectively set $\delta$ to 40 miles and to 10 miles. Under the KBMA parameterization, iterating ends after 343 joins, leaving 931 kernels.



% 0.25  343 iterations/931 kernels

\begin{comment}  ****  INCIDENCE OF EXCLUDED PARTLY-ENCOMPASSED TRACTS GETTING INTO FMA ***

!! THIS IS BASED ON AN EARLIER PARAMETERIZATION

excluded from a UA approximation, more than 15 percent are nevertheless added to an FMA during the third and fourth stages of the algorithm. Among excluded tracts with at least half of their population in a single UA, almost 30 percent are so added. %to an FMA during the third and fourth stages.


 tab i_uax i_fma2000 if i_portionr_ua_afactp50==1

           |       i_fma2000
     i_uax |         0          1 |     Total
-----------+----------------------+----------
         0 |         2     41,742 |    41,744
         1 |     1,864        762 |     2,626   (762/2626=0.290)
-----------+----------------------+----------
     Total |     1,866     42,504 |    44,370



. tab i_uax i_fma2000 if i_portionr_ua_afactp50l50==1

           |       i_fma2000
     i_uax |         0          1 |     Total
-----------+----------------------+----------
         0 |         1     40,090 |    40,091
         1 |         1          4 |         5
-----------+----------------------+----------
     Total |         2     40,094 |    40,096

. tab i_uax i_fma2000

           |       i_fma2000
     i_uax |         0          1 |     Total
-----------+----------------------+----------
         0 |    18,019     42,158 |    60,177
         1 |     4,244        881 |     5,125
-----------+----------------------+----------
     Total |    22,263     43,039 |    65,302
** --> so almost all of the tracts subsequently added to the FMA
have at least 50 percent of thier pop in the UA approximation (762 of the 881)

. tab i_fma2000_single i_fma2000_enclosed if i_uax==1 & i_fma2000==1

i_sec_trac |     i_cont_tract
         t |         0          1 |     Total
-----------+----------------------+----------
         0 |         5         81 |        86
         1 |       795          0 |       795
-----------+----------------------+----------
     Total |       800         81 |       881

. tab i_fma2000_uas i_fma2000_uc if i_uax==1 & i_fma2000==1

           |       i_sec_UC
  i_sec_UA |         0          1 |     Total
-----------+----------------------+----------
         0 |       876          5 |       881
-----------+----------------------+----------
     Total |       876          5 |       881



. tab i_uax i_fma2000 if ( i_portionr_ua_threshpyln == 1 | i_portionr_ua_threshpyly == 1)

           |       i_fma2000
     i_uax |         0          1 |     Total
-----------+----------------------+----------
         0 |         2     41,742 |    41,744
         1 |     1,107        680 |     1,787
-----------+----------------------+----------
     Total |     1,109     42,422 |    43,531

.



%. sum id_tract2000 i_fma2000_single i_fma2000_enclosed if afact_portionr_pop_share >= 0.50 & afact_portionr_pop_share <0.65
%
%    Variable |        Obs        Mean    Std. Dev.       Min        Max
%-------------+---------------------------------------------------------
%id_tract2000 |      3,283    354682.8    419547.5        100     998800
%i_fma2000_~e |      3,283    .0210174    .1434639          0          1
%i_fma2000_~d |      3,283    .0097472    .0982604          0          1
%
%. sum id_tract2000 i_fma2000_single i_fma2000_enclosed if afact_portionr_pop_share >= 0.50 & afact_portionr_pop_share <0.65& i_fma2000==1
%
%    Variable |        Obs        Mean    Std. Dev.       Min        Max
%-------------+---------------------------------------------------------
%id_tract2000 |        101      241779    283304.2        800     881004
%i_fma2000_~e |        101    .6831683    .4675616          0          1
%i_fma2000_~d |        101    .3168317    .4675616          0          1
%
%. sum id_tract2000 i_fma2000_single i_fma2000_enclosed if afact_portionr_pop_share >= 0.50 & afact_portionr_pop_share <0.65& afact_portionr_land_share >= 0.50
%
%    Variable |        Obs        Mean    Std. Dev.       Min        Max
%-------------+---------------------------------------------------------
%id_tract2000 |      1,481    347471.7      418779        100     997300
%i_fma2000_~e |      1,481    .0101283    .1001623          0          1
%i_fma2000_~d |      1,481    .0054018    .0733226          0          1
%
%. sum id_tract2000 i_fma2000_single i_fma2000_enclosed if afact_portionr_pop_share >= 0.50 & afact_portionr_pop_share <0.65& afact_portionr_land_share >= 0.50 & i_fma2000==1
%
%    Variable |        Obs        Mean    Std. Dev.       Min        Max
%-------------+---------------------------------------------------------
%id_tract2000 |         23    260646.6    322800.6       1500     864101
%i_fma2000_~e |         23    .6521739    .4869848          0          1
%i_fma2000_~d |         23    .3478261    .4869848          0          1

\end{comment}

\begin{comment}  *** Fig illustrating 500 pm2; and pdf of afactors

\begin{comment} still more notes on UAUCAs (I spent a ton of time, perhaps wasted, worrying about this)

% excluded tract figures in fma_calculations_misc.2020_08_29a.xlsx and
%(Figure \ref{tract-allocation-factors})

% THIS FIGURE ILLUSTRATES THE CONSERVATIVE NATURE OF THE 500 PER SQUARE MILE THRESHOLD
%\begin{figure}[tbp]
%%\begin{center}
%\includegraphics[scale = 1.0, trim = 20mm 199mm 10mm 16mm, clip]{fma_figure_tract_max_density_distribution_2020_06_28a.pdf}
%%\end{center}
%%\vspace{-8mm}
%\caption{\label{tract-max-density-distribution} \textbf{Distribution of Maximum Density} {\normalfont \footnotesize \hspace{1.5mm} Red histogram shows the distribution of the maximum of population and worker density across tracts that are fully encompassed in an Urbanized Area. Blue histogram shows the same across tracts that are partly encompassed in an Urbanized Area and that meet the threshold population and land area shares to be included in the tract-based approximation.}}
%\end{figure}

% *** POSSIBLY INCLUDE IN APPENSIX
%\begin{figure}[tbh]
%%\begin{center}
%\includegraphics[scale = 1.0, trim = 20mm 199mm 10mm 16mm, clip]{fma_figure_afactor_distributions_2020_08_24a.pdf}
%%\end{center}
%%\vspace{-8mm}
%\caption{\label{tract-allocation-factors} \textbf{Distribution of Land and Population Allocation Factors.} {\normalfont \footnotesize \hspace{0.5mm} Figure shows the tracts' population and land area allocated to a partly encompassing Urbanized Area.}}
%\end{figure}

\end{comment}

Prior to iterating, the 1,274 cores form more than 810,000 pairs. Constrained to those separated by no more than 20 miles, the initial iteration joins the Miami and Key Biscayne cores, which have pairwise commuting strength of 1.28 and separating distance of 5.2 miles. Table \ref{kernel_comparison_strength} illustrates some of the later iterations as $\sigma$ is incrementally lowered from a strength of 0.30 to its KBMA parameterized value of 0.25 and then further down to a strength of 0.20. For example, the 0.30 strength threshold is sufficiently low to join Raleigh and Durham with one additional core,  San Francisco and San Jose with nine additional cores, and Los Angeles and Riverside--San Bernadino with 12 additional cores. Lowering $\sigma$ to 0.25 joins the already combined Manchester and Nashua cores with Boston, Worcester, and two others. Further lowering $\sigma$ to 0.20 joins this cluster with Barnstable Town (Cape Cod) and three others. Doing so also joins the Washington D.C. and Baltimore clusters, as well as the Los Angeles and Oxnard clusters. It would additionally join the Indio--Cathedral City--Palm Springs core to the Los Angeles cluster except that the separating distance between them is a few tenths above the allowed 20 miles.

\begin{table}[tb] %Table Kernel Composition under Alternative Commuting Strength Thresholds (kbma_for_display.2025_02_04a.xlsx)
\caption{\label{kernel_comparison_strength} \textbf{Kernels under Alternative Commuting Strength Thresholds}}
\begin{center}
\includegraphics[scale = 0.90, trim = 18mm 104mm 10mm 19mm, clip]{kbma_table_alt_kernels_2025_04_14a.pdf}
\end{center}
\vspace{-8mm}\begin{flushleft}
\footnotesize{Notes: The left column reports cores joined in selected kernels under a commuting strength threshold an increment above the KBMA parameterized value. The middle and right columns report additional cores joined from lowering the strength threshold to its KBMA value ($\sigma$ = 0.25) and then lowering it a further increment. Joined cores must be separated by no more than 20 miles ($\delta$ = 20). An \href{https://www.kansascityfed.org/documents/10788/rwp25-01_kernel_iterations.zip}
{\underline{online workbook}} enumerates the sequence of iterative joins under alternative maximum distances.}
\end{flushleft}
\end{table}
%, measured between the nearest tract centroids ($\delta$ = 20 miles).

\begin{comment}; additional text, including subjective defense of sigma=0.25
% additional text
%We favor the baseline specification over the relaxed one in part because the latter joins some cores that we think of more as weekend destinations for escape from large metropolitan areas rather than as extensions of them (e.g., Oxnard and Palm Springs as destinations from Los Angeles; Cape Cod as a destination from Boston; Port Huron as a destination from Detroit; and Olympia as a destination from Seattle). We favor the baseline specification over the tight one primarily because it matches the value used by the OMB.
%Under the tighter and baseline parameterizations, all joins are between cores separated by less than 20 miles. Under the more relaxed parameterization all joins are separated by less than 20 miles, except the one that joins Indio--Cathedral City and the Los Angeles kernel, which are separated by 20.3 miles.
%Instead setting $\delta$ to its baseline value of 20 miles prevents no joins under the baseline specification and only one join under the relaxed threshold, that between
%\footnote{Consistent with our judgment, the Indio--Cathedral City--Palm Springs core constitutes its own television market, separate from the television market encompassing the Los Angeles and Riverside-San Bernadino cores.}
\end{comment}

\begin{comment} Figuring out merge strengths for specific joins
% deciphering merge strengths under baseline for luaca size (50k) and distance (20 miles)
% look at luaucas_p65l30 in kbma_kernels_lambda_sigma_delta

% --> the s I'm reporting is of the "final" join in the sequence. But the pairwise strength between the specific UAUCAs will differ if either had a previous join to it.

% ** San Francisco and San Jose
% Antioch and Concord join with 0.530537
%--> then SF joins with Antioch/Concord at 0.48479
%--> then Liverpool CA joins SF/Antioch/Concord at 0.7745522
%--> then Vajello joins at 0.506135
%--> then San Jose joins at 0.517465

% Los Angeles and Riveside-San Bernadino
% Los Angeles is index is 196
% Riverside is 276
% note: lower number "survives" merges
% Los Angeles joins with Mission Viejo (217), s= 0.785
% Los Angeles/Mission Viejo (surviving index is 196) join with Santa Clarita (301), s=0.780
% Los Angeles/Mision Viejo/SAnta Clarita (surviving is 196) join with Simi Valley (313) s=0.515
% Los Angeles/Mision Viejo/SAnta Clarita join/Simi Valley (surviving is 196)join with Thousand Oaks (336) s=0.586
% Los Angeles/Mision Viejo/SAnta Clarita join/Simi Valley/Thousand Oaks (surviving is 196) join with Lancaster/Palmdale (174), s=0.495
% Los Angeles/Mision Viejo/SAnta Clarita join/Simi Valley/Thousand Oaks Lancaster/Palmdale (surviving is 174) join with Riveside--San Bernadino, s=0.378


% Washington D.C. (357) and Baltimore (26)
% Washington D.C. joins with St. Charles (286), s= 0.662
% Washington D.C--S. Charles join with Fredericksville (114), s=0.368

% Baltimore (26) joins with Westminster(36) @ s=0.538
% Baltimore--Westminster (26) joins with Aberdeen (1) @ s=0.535

% Baltimore-Westminster-Aberdeen (1) joins with Washington D.C--St Charles--Fredericksvill at @ s=0.203

%\footnote{Suggestive of belonging to the same metropolitan area, a Google search of the exact word combination ``Raleigh Durham" returns more than 19 million results.}
%Similarly, the San Francisco and San Jose UA approximations are separated by less than 1 mile, respectively have 1,405,000 and 726,000 residents who work, and 158,000 residents who commute in one direction or the other. In this case the maximum of the normalized flows is just 0.118. The sum of the four is 0.325

\footnote{Implicitly crowd-sourced judgments support joining the Raleigh and Durham cores and the San Francisco and San Jose cores. For example, a Google search of the exact word combination ``Raleigh Durham" returns more than 19,000,000 results. And Wikipedia defines the ``Bay Area'' to include San Franciso, San Jose, and the remaining cores that are joined under our baseline specification.}


\end{comment}

\begin{comment} possible text describing additional details of cores
%\footnote{In addition to the four pairs shown at the bottom right of Table \ref{kernel_comparison_strength}, moving from the baseline to the relaxed commuting threshold joins nine other pairs of UAUCAs: Johnson City, TN with Kingsport, TN-VA; Bonita Springs--Naples, FL with Cape Coral, FL; Atlantic City, NJ with Wildwood--North Wildwood--Cape May, NJ; Macon, GA with Warner Robbins, GA; Killeen, TX with Temple TX; Indianapolis, IN with Anderson, IN; Columbus, OH with Newark, OH; Odessa, TX with Midland, TX; and Port St. Lucie, FL with Vero Beach-Sebastian, FL.}
%Similarly, the kernel that includes Los Angeles combines eight UAUCAs under the tight commuting threshold. Loosening the threshold to its baseline value leaves the kernel unchanged; loosening the threshold to its relaxed value expands the kernel to include the Oxnard and Camarillo UAUCAs, which themselves were already combined in a single kernel under the tight specification.
\end{comment}

\begin{comment} television market stuff

%The Indio-Cathedral City--Palm Springs core also constitutes its own television market, reinforcing our judgment that it belongs to a different metropolitan area than the one encompassing the Los Angeles and Riverside--San Bernadino cores.

%relaxed also combines Baltimore with Washington D.C. television markets. As FRankie outlines in television_markets.2022_03_01a.pdf, a bunch of our baseline KBMAs include cores in different televisions markets. But these examples are usually small cores located between larger ones and so arguably located near the periphery where two television markets border each other.  Some television markets encompass multiple KBMAs, which is unsurprising given that KBMAs can have population far below that needed to support a television market.
\end{comment}

% 0.30: 280 iterations/994 kernels
% 0.25  343 iterations/931 kernels
% 0.20  427 iterations/847 kernels


We judge that setting $\sigma$ to a commuting strength of either 0.30 or 0.25 constructs a set of kernels that is consistent with our metropolitan definition. One reason we choose the lower value is to keep the threshold comfortably below the join of Raleigh and Durham, which occurs at a strength slightly above 0.30. We are more skeptical that the set of kernels implied by lowering $\sigma$ to a strength of 0.20 is consistent with our definition. A number of the incremental joins are with cores we think of as seasonal rather than day-to-day destinations. Specifically, Oxnard, Cape Cod, Olympia, and Port Huron all market themselves as coastal getaways.

In addition, setting $\sigma$ to a commuting strength of 0.25 rather than 0.20 implies a more robust buffer against joining the Indio--Cathedral City--Palm Springs core to the Los Angeles kernel, consistent with our prior that it constitutes its own metropolitan area. Otherwise, not joining the two rests fragilely on their separating distance being a tad above the parameterized maximum. More generally, we methodologically prefer commuting strength to serve as the primary determinant of joins, reflecting that strength typically decreases with separating distance and so in part implicitly accounts for distance.
%We experimented allowing for unlimited separating distance, but doing so attaches some sparsely settled tracts to kernels hundreds of miles away during the buildout stage, even with a moderately high strength threshold.
%In contrast, we are ambivalent about whether Washington and Baltimore belonged to the same metropolitan area in 2000.
% Indio-Cathedral City--Palm Springs is separated from Los Angeles kernel by 20.3 miles

% ****  WHY WE NEED EITHER A DISTANCE OR A DENSITY RESTRICTION A parameterization that has no density restriction on tracts and no max separating distance: joins a tract in Montana to the Los Angeles kernel, reflecting that four of its 12 employed residents reported working in census tracts in the Los Angeles kernel the previous week. The farthest join under this parameterization is between a tract in the Aleutian Islands and the Philadelphia kernel.}

We judge that the kernel iterations appropriately recombine UAs split from each other by the backward alignment with the 1990s MSAs. We are able to identify 32 such splits, 19 of which are recombined at the KBMA parameterization. Several other pairs are recombined if $\sigma$ is set to a strength of 0.20,  including Port Huron and Detroit, Portsmouth and Boston, and Johnson City and Kingsport. All but four pairs are recombined at the parameterization for kernel-based metropolitan regions, which sets $\sigma$ to a strength of 0.10.

Not rejoining the remaining four pairs of metropolitan regions illustrates the success of the kernel iteration at avoiding constructing long chains of cores. One of the remaining splits of urban area agglomerations divides New York and Bridgeport--Stamford into separate UAs, between which the pairwise commuting strength is 0.19. The iterative sequencing initially joins each to other cores, diluting the pairwise strength between the resulting kernels. Further iterative dilutions delay combining New York and Bridgeport--Stamford in the same kernel until $\sigma$ falls below a strength of 0.08. As another example, the iterative sequencing delays recombining the split San Diego and Mission Viejo UAs until $\sigma$ falls below 0.05, at which point the San Diego and Los Angeles kernels join.
%Similarly, the sequencing delays recombining Kenosha, WI and Rancine, WI cores until $\sigma$ drops below 0.03, at which point the Chicago and Milwaukee kernels join.% Kenosha and Racine.
% delays joining New York and Philly--which are not adjacent--until sigma drops below 0.05

In contrast to the high sensitivity for specific kernels, an appendix figure illustrates the gradual decline in
the number of kernels as $\sigma$ is lowered. It also illustrates that relatively few iterative joins are constrained by the maximum allowed separating distance, $\delta$, at the parameterizations of KMBAs, kernel-based metropolitan regions, and kernel-based urban areas.


\begin{comment} distance threshold matters not much Flows into and out of KBMAs as they vary with the distance threshold

%In contrast, commuting inflows and outflows are extremely insensitive to the distance threshold. More specifically, inflows as a share of employment and outflows as a share of employed residents remain very close to their baseline values as the distance threshold is relaxed from 10 miles to 50 miles. One reason is that the actual number of people flowing into and out of tracts that meet the strength and density thresholds but not the distance threshold is very low compared to the number of employed residents and employees in the baseline KBMAs.

% outflow to non-KBMA
% median:       10 miles = 0.078   baseline = 0.077   50 = 0.078
% 90th pctile   10 miles = 0.157   baseline = 0.157   50 = 0.156
%

% inflow from non-KBMA
% median:       10 miles = 0.216    baseline = 0.215  50 = 0.213
% 90th pctile   10 miles = 0.375    baseline = 0.372  50 = 0.372

\end{comment}
\begin{comment} Considerations on choosing distance threshold:
% I ALSO HAVE SOME NOTES ON THIS IN FMA_FOR_DISPLAY/T.KERNELS_ALT_DIST
% St. Charles with D.C.: sum = 0.6450; max = 0.5181 residents in S work in DC; 0.1177 workers in St.C live in DC
% Fredericksburg with D.C.: sum = 0.3150; 0.2176 residents in F work in D.C; 0.0911 workers in F live in D.C.)

%Separately, I’ve decided to reduce the baseline maximum distance to 10 miles from 15.  The implied change in the composition of kernels is slight: it breaks apart New Orleans, Slidell, and Mandeville-Covington into separate kernels; it breaks of Victorville—Hesperia—Apple Valley from Los Angeles—San Bernadino; it breaks Fairfield, Vacaville, and Napa off from San Francisco; and it breaks apart Brownsville and Harlingen TX.  As we’ve discussed in the past, I regret the first of these. But I find it hard to justify catering the rule to a single example, especially one reflects a large body of water in the middle (Lake Pontchartain). And to the extent that considering each of Slidell and Mandeville-Covington as their own FMAS, this is really a statement about needing to impose a higher minimum population threshold rather than the need to combine them with New Orleans.   On the other hand, the Victorville—Hesperia--Apple Valley UAUCA is on the other side of a fairly steep mountain range (with skiing resorts) from the San Bernadino UAUCA. If we did join it, we would be hard pressed to justify not joining in Lancaster—Palmdale, which is 15.01 miles from the Los Angeles UAUCA  on the other side of the same mountain range.

\end{comment}


\begin{figure}[tb] % Sensitivity of land area and commuting inflows to the buildout density threshold
\caption{\label{size_alt_density}\textbf{Sensitivity of Built-Out Kernels to the Density Threshold}}
\includegraphics[scale = 0.98, trim = 20mm 182mm 10mm 20mm, clip]{kbma_figure_calibrating_eta_2025_03_20a.pdf}
\vspace{-10mm}\begin{flushleft}
\footnotesize{Notes: Left panel shows the median and 90th-percentile ratios of the land area of the buildout portion of built-out kernels relative to the land area of the kernel portion as $\eta$ is increased from 0 to 500. The dashed vertical line corresponds to the parameterized KBMA value, $\eta$ = 200. The right panel shows the median and 90th percentile rates of commuting inflows. For comparability, kernels are restricted to the 302 with population of at least 50,000. A complementary appendix figure illustrates buildout ratios for population and employment, and the commuting outflow rate.}
\end{flushleft}
\end{figure}




The buildout stage requires additionally parameterizing $\eta$, the minimum settlement density for eligible tracts to be attached to a kernel. We judgmentally set it to 200 (residents per square mile or workers per square mile, whichever is higher), balancing encompassing commuting flows and excluding sparsely settled land.

Figure \ref{size_alt_density} illustrates the tradeoff. The left panel shows the ratio of land in the buildout portion of KBMAs to land in their kernel portion as $\eta$ is increased from 0 to 500. The median and 90th percentile buildout ratios at 200 are 0.8 and 1.9, respectively. Both begin to blow up as $\eta$ is lowered below this. The right panel shows the median and 90th percentile of KBMAs' commuting inflow rates. Both rise significantly as $\eta$ is increased from 0.

We judge that the marginal benefit of encompassing more flows by decreasing $\eta$ below 200 approximately equals the marginal cost of encompassing additional sparsely settled land area. To span a wide range of judgments, we drop any minimum for kernel-based metropolitan regions, setting $\eta$ to 0; we set it moderately higher, to 500, for kernel-based urban areas. (500 is the threshold population density for census blocks to be included in a UA or UC.) An analogous appendix figure shows population and employment buildout ratios and the encompassment of commuting outflows, all of which are less sensitive to the parameterization of $\eta$.

\begin{table}[tb] % 50 Largest KBMAs by Population in 2000
\caption{\label{kbmas_top50} \textbf{50 Largest KBMAs by Population in 2000}}
%\centering
\includegraphics[scale = 0.87, trim = 17mm 134mm 5mm 18mm, clip]{kbma_table_top50_2025_03_07a.pdf}
%\end{center}
\vspace{-8mm}\begin{flushleft}
\footnotesize{Note: An enumeration of all KBMAs, with additional variables, is included as an appendix and
\href{https://www.kansascityfed.org/documents/10743/rwp25-01_online_kbmas_enumeration.pdf}{\underline{online}}.}
\end{flushleft}
\end{table}



Lastly, we classify only the 361 built-out kernels that have population above 50,000 as KBMAs, matching the convention used by several of the delineations described in the previous section.  Of course, higher or lower population thresholds may better match various purposes. Table \ref{kbmas_top50} lists the 50 KBMAs with largest population.\footnote{We title KBMAs using the Census Bureau's algorithm for titling UAs and UCs \citep{Census_2002}. The first listed name is the incorporated place with highest population. The names of the next two incorporated places by population are included if they have population exceeding 250,000 or if their population is at least two thirds that of the largest incorporated place. State postal abbreviations are ordered to correspond to any places included in the title and then by descending order of each state's population in the KBMA.} An enumeration of all KBMAs along with detailed data on underlying census tracts, cores, built-out kernels, and the iterative joins constructing kernels are available from the paper's
\href{https://www.kansascityfed.org/research/research-working-papers/a-better-delineation-of-us-metropolitan-areas/}
{\underline{webpage}}.

%for the KBMA parameterization includes detailed data on underlying census tracts, cores, all built-out kernels, and the iterative joins constructing kernels.


\begin{comment} % More language justifying the various thresholds, and summing over max

% Using the maximum strength measure (and $\sigma$ sufficiently low to join Raleigh and Durham) results in some delineated FMA kernels that we judge excessively large compared to those that result from using the summed strength measure. For example, measuring strength by the maximum normalized flow with a threshold  of 0.10 and our baseline distance threshold, $\delta=$10 miles, implies the Boston FMA kernel extends from Providence east through Cape Code and north up through southern Maine; the New York kernel extends from Trenton north through the Hudson Valley up to the latitude of the Massachusetts border; and the San Francisco kernel extends from Santa Cruz northwest to Santa Rosa. Instead using summed strength with our baseline threshold, $\theta=0.25$, breaks this Boston set of joins into five FMA kernels, the New York ones into three FMA kernels, and the San Francisco ones into four FMA kernels.
%Using a distance threshold similarly helps achieve FMA delineations that more closely match our judgment. For example, removing the distance threshold from our baseline delineation joins four additional UA approximations to the San Francisco kernel (Napa, Fairfield, Vacaville, and Tracy), which we judge are better left out. The robustness section below describes this and other implied differences. Conceptually, a distance threshold arguably corresponds with defining metropolitan areas as ``discrete'' geographies. On the other hand, any desired level of discreteness can be achieved by setting $\sigma$ sufficiently high. Using a distance threshold risks overfitting delineations to idiosyncratic judgment. For comparison, neither the CZ delination algorithm nor \citet{duranton_2015} apply a distance threshold. \emph{To be rewritten to include results of using $\delta$=15 miles rather than baseline 10 miles. }

%, $i$ and $j$, by the sum of their four normalized flows: $\mathbf{s}_{i,j}, \mathbf{s}_{j,i} = \mathbf{f}_{i,j}/\mathbf{w}_{i}+\mathbf{f}_{j,i}/\mathbf{e}_{i}+\mathbf{f}_{j,i}+\mathbf{w}_{j}$, $\mathbf{f}_{i,j}/\mathbf{e}_{j}$.

%Our measure of strength is essentially the same as the measure used by OMB for joining outlying counties to CBSAs, except that we treat the two locations symmetrically rather than assuming one is subsidiary. At the first iteration, the pair of UAs with strongest commuting ties is joined, provided that $\mathbf{s}_{i,j}$ is at least $\sigma$ and their separating distance---measured between the nearest tract centroids---is no more than $\delta$. The resulting UA combination replaces its components as a location in the second iteration. This continues until no further joins are possible.

%Strength might alternatively be measured by summing the four normalized flows rather than taking their maximum. Doing so emphasizes the importance of commuting to resident workers and employers throughout a joined set of locations rather than to only workers or only employers in only one constituent location. (A subsection below describes the sensitivity of FMA delineations to using this alternative measure of strength.)\footnote{We consider the ERS strength measure, $(\mathbf{f}_{i,j}+\mathbf{f}_{j,i})/\min(\mathbf{w}_{i},\mathbf{w}_{j})$, more idiosyncratic and so do not explore the sensitivity of results to using it. One reason is the measure splits out counties within the same CBSA into portions of multiple other CZs. (Our concern with CBSAs is that they overbound conceptual metros; but the specific counties composing CBSAs seem reasonable.) Another reason is that the ERS strength measure is unbounded above. For example, ``reverse'' commuting flows from an FMA kernel into a nearby tract can manyfold exceed workers residing in the destination tract.}
% Among SF area UAs, joins are in same sequential order regardless of whether strength measured by

\end{comment}

\begin{comment} % additional justifications for our strength measure

%\footnote{The ERS algorithm for Commuting Zones calculates strength as $(f_{A,B}+f_{B,A})/\min(r_A\, , \, r_B)$. One reason we favor our calculation is that the inclusion of the destination ratios more transparently captures the contribution of commuting to meeting spatially concentrated labor demand. Another reason is that the ERS calculation rises without bound as labor force participation in a destination UA goes to zero, a stylization that might describe UAs whose residents are primarily retirees.}  Thus at each iterative step we determine all candidate joins based on the distance and commuting threshold criteria and then effect the join with highest strength; the iteration continues until no further joins are possible.

%\footnote{The ordering can effect the final composition of kernels. Under our baseline specification, the Concord CA UA meets the distance and flow criteria to join with both the San Francisco and Antioch UAs. It has stronger commuting ties with Antioch and so joins with it first. However, the resulting combination of the Concord and Antioch UAs falls just shy of meeting the flow criterion with the San Francisco UA. Had Concord instead joined first with San Francisco, all three UAs would be combined in the same kernel.}
%The iteration ordering affects the final delineation by breaking some potential chains of UAs. For example, $f_{AB}/E_B$ and $f_{B,C}/E_C$ may each exceed $\sigma$ but $f_{A,BC}/E_{BC}$ may fall short of $\sigma$. In this case, all three will be combined if A and B are joined first; but A may remain separate if B and C are joined first.

\end{comment}

\begin{comment}  % more discussion on thresholds
%For our baseline parameterization, we set $\sigma$ equal to 0.25 and $\delta$ equal to 10 miles. The former is the same value used by OMB to determine the outlying counties of CBSAs. The latter corresponds to our judgment of the time driving away from peripheral suburbs through sparsely settled land after which people might judge they had left our conception of a metropolitan area. Under this baseline, the iteration combines 452 UA tract-based approximations into 420 FMA kernels.


% Indio-Cathedral City--Palm Springs with Los Angeles kernel: strength==0.20, distance = 20.3 miles
% Santa Maria with Santa Barbara kernel: strength=0.1908, distance=41.2 miles
% Mandeville-New Orleans = 21.8 miles
% Mandeville-Slidell     = 14.7 miles
% New Orleans-Slidell    = 14.6 miles

% Lancaster-Palmdale with Los Angeles et al., s=0.4911; dist = 15.01 miles
% Tracy              with San Francisco,      s=0.4480; dist = 15.9 miles

%The third stage of our algorithm consists of two sets of joins to the FMA kernels to form interim FMAs. First, tract-based approximations of UCs---constructed with the same inclusion thresholds used to assign tracts to UA approximations---are joined to an FMA kernel if their commuting flows with the kernel and distance from it meet the criteria for joining UA approximations.\footnote{Many UC approximations consist of a single tract.} Under the baseline parameterization, doing so joins xxx tract-based UCs to yyy FMA kernels, with zz or more UCs attaching to the largest kernels.%\footnote{The Minneapolis-St.\ Paul and Denver FMAs include 20 and 18 UC approximations, respectively. The New York, Pittsburgh, and Houston FMAs include between 10 and 14 UC approximations.}

\end{comment}


\section{Realized Delineations}

To gauge our success in matching our metropolitan definition, we first show maps of realized KBMAs and describe KBMAs' overlap with metropolitan CBSAs. We then document the extent to which KBMAs encompass commuting inflows and outflows.

\subsection{Illustrations}

Figure \ref{kc_kbma_zoomed_in} illustrates the composition of the Kansas City MO--KS KBMA. The kernel's three cores correspond to the Kansas City, MO--KS UA; a portion of a major suburb, Lee's Summit; and a very small suburban UA, Excelsior Springs.  Most of the buildout tracts are contiguously connected to the kernel; the detached ones correspond to UCs with population below 10,000. Most excluded tracts are expansive, reflecting sparse settlement. The main exception, the cluster of small tracts near the northwest corner, corresponds to the Leavenworth UC, whose commuting strength with the Kansas City kernel falls just short of the threshold to combine with it and whose built-out population on its own is too low to qualify as a KBMA.


\begin{figure}[htb]% Composition of Kansas City KBMA
\centering
\caption{\label{kc_kbma_zoomed_in} \textbf{Composition of the Kansas City KBMA}}
%\begin{center}
\includegraphics[scale = 0.53, trim = 17mm 26mm 0mm 33mm, clip]{kbma_map_kc_2025_04_09.png}
%\end{center}

\vspace{-5mm}
\begin{flushleft}
\footnotesize{Notes: Census tracts, demarcated in gray, have population density that is inversely correlated with their land area. The displayed area lies entirely within the Kansas City MO-KS metropolitan CBSA.}
\end{flushleft}
\end{figure}


\begin{figure}[htb]%Kansas City and neighboring KBMAs
\centering
\caption{\label{map_kc_vicinity} \textbf{Kansas City and Neighboring KBMAs}. {\normalfont \footnotesize }}\includegraphics[scale = 0.58, trim = 14mm 39mm 0mm 38mm, clip]{kbma_map_kc_region_2025_04_09.png}

\vspace{-5mm}
\begin{flushleft}
\footnotesize{Notes: Blue lines demarcate the borders of metropolitan CBSAs. Census tracts, demarcated in gray, have population density that is inversely correlated with their land area. }
\end{flushleft}
\end{figure}



Figure \ref{map_kc_vicinity} zooms out. Kansas City and its three neighboring KBMAs each occupy the most densely settled portion of a corresponding metropolitan CBSA. The commuting strength between the Kansas City and Lawrence kernels is a tick below 0.150, sufficient to combine them under the parameterization for kernel-based metropolitan regions.


The correspondence between KBMAs and metropolitan CBSAs is less clean in Southern California. As illustrated in Figure \ref{los_angeles_at_al}, the Los Angeles KBMA encompasses essentially the entire population of the Los Angeles CBSA, most of the population of the Riverside-San Bernadino CBSA, and a significant portion of the population of the Oxnard-Thousand Oaks CBSA. The parameterization for kernel-based metropolitan regions combines the Los Angeles, Oxnard, and Indio KBMAs together, closely corresponding to an implicitly crowd-sourced judgement of ``Greater Los Angeles'', defined by Wikipedia as the combination of the Los Angeles, Riverside-San Bernadino, and Oxnard-Thousand Oaks CBSAs.

\begin{figure}[tb] %Los Angeles and neighboring FMAs
\caption{\label{los_angeles_at_al} \textbf{Los Angeles and Neighboring KBMAs}}
%\begin{center}
\includegraphics[scale = 0.74, trim = 32mm 39mm 0mm 32.5mm, clip]{kbma_map_socal_2024_10_04.png}
%\end{center}
%\vspace{-2mm}
\vspace{-8mm}\begin{flushleft}
\footnotesize{Notes: Blue lines demarcate the borders of metropolitan CBSAs. Census tracts, demarcated in gray, have population density that is inversely correlated with their land area.}
\end{flushleft}
\end{figure}

% %if not distance limit, 0.25 to 0.20 extends LA  further east to encompass the Indio KBMA (but latter also rerquires relaxing distance threshold above 20 miles)

\begin{comment} Geographic notes & San Bernadino et al calculations

% Bakersfield CBSA borders Los Angeles CBSA to the North
% Lancaster-Palmdale separated from LA by San Gabriel Mountains; Victorville--Hesperia-Apple Valley separated from Riverside San Bernadino by the San Bernadino Mountains.

see fma_calculations_misc.2020_10_03a.xlsx
% OLDER:
% Riverside-San-Bernadino CBSA pop=3,254,821; land=27259.85 (mi2); wrk= (source:fma.baseline.part3.fmas.2020_08_29a.dta and fma_calculations_misc.2020_08_29a.xlsx
% Riverside-San-Bernadino FMA  pop=1,463,156  land = 561.5 (mi2)        (source: fma.baseline.part3.fmas.2020_08_29a.dta and fma_calculations_misc.2020_08_29a.xlsx)
% pop_share = 0.4495
% land_share = 0.0206
% lowers its population ranking from 13th among CBSAs to 26th among FMAs
\end{comment}



\begin{figure}[htb] % KBMAs in Southern New England
%\begin{center}
%\hspace{-2mm}
\caption{\label{sne_kbmas} \textbf{KBMAs in Southern New England}}
\includegraphics[scale = 0.695, trim = 24mm 22mm 0mm 25mm, clip]{kbma_map_sne_kbmas.2025_03_20a.png} %boston_2022_04_21a.pdf
%\end{center}
%\vspace{-8mm}

\vspace{-5mm}\begin{flushleft}
\footnotesize{Notes: Blue lines demarcate the borders of metropolitan CBSAs. Gray lines demarcate tract borders.}
\end{flushleft}
\end{figure}
% The symmetric combination of allocation factors (Panel C) is more restrictive than the baseline combination, notwithstanding the more relaxed population minimum.


KBMAs along the Atlantic Coast of southern New England are tightly packed. For example, as illustrated in Figure \ref{sne_kbmas}, the Boston KBMA adjoins four other KBMAs and lies within 30 miles of four more. It also encompasses the most densely settled portions of three metropolitan CBSAs: Boston, Worcester (directly to its west), and Manchester-Nashua (to its northwest).


\begin{figure}[htb] % Kernel-Based Urban Areas in Southern New England
%\begin{center}
%\hspace{-2mm}
\caption{\label{sne_kbuas} \textbf{Kernel-Based Urban Areas in Southern New England}}
\includegraphics[scale = 0.695, trim = 24mm 22mm 0mm 25mm, clip]{kbma_map_sne_kbuas.2025_03_20a.png} %boston_2022_04_21a.pdf
%\end{center}
%\vspace{-8mm}

\vspace{-5mm}\begin{flushleft}
\footnotesize{Notes: Blue lines demarcate the borders of metropolitan CBSAs. Gray lines demarcate tract borders.}
\end{flushleft}
\end{figure}
% The symmetric combination of allocation factors (Panel C) is more restrictive than the baseline combination, notwithstanding the more relaxed population minimum.

Figure \ref{sne_kbuas} shows the kernel-based urban areas in the same footprint, which fragment many KBMAs.  For example, the Boston KBMA fragments into four separate kernel-based urban areas: Boston, Worcester, Leominster--Fitchburg, and Manchester--Nashua. Other KBMAs ``disappear.'' Specifically, the built-out kernels corresponding to the Kingston, Glens Falls, Pittsfield, Concord, and Lewiston KBMAs no longer have population that exceeds 50,000 and so fail to qualify as kernel-based urban areas.\footnote{For each of the three parameterizations, a workbook with detailed variables for all built-out kernels is available from the paper's
\href{https://www.kansascityfed.org/research/research-working-papers/a-better-delineation-of-us-metropolitan-areas/}{\underline{webpage}}.} The median land area of the surviving 346 kernel-based urban areas is 34 percent smaller than the median land area of KBMAs.


\begin{figure}[htb] % Kernel-Based Metropolitan Regions in Southern New England
%\begin{center}
%\hspace{-2mm}
\caption{\label{sne_kbmrs} \textbf{Kernel-Based Metropolitan Regions in Southern New England}}
\includegraphics[scale = 0.695, trim = 24mm 22mm 0mm 25mm, clip]{kbma_map_sne_kbmrs.2025_03_20a.png} %boston_2022_04_21a.pdf
%\end{center}
%\vspace{-8mm}

\vspace{-5mm}\begin{flushleft}
\footnotesize{Notes: Blue lines demarcate the borders of metropolitan CBSAs. Gray lines demarcate tract borders.}
\end{flushleft}
\end{figure}

Conversely, the kernel-based metropolitan regions in the same footprint, shown in Figure \ref{sne_kbmrs}, consolidate many KBMAs and vastly expand them outward to encompass swathes of rural area. For example, the Boston kernel-based metropolitan region spans five KBMAs: Boston, Providence, Barnstable Town, Rochester-Dover-Portsmouth, and Concord; its land area is more than twice the aggregate of the five KBMAs. The median land area of kernel-based metropolitan regions is more than 16 times that of KBMAs.



\subsection{Comparison to Metropolitan CBSAs}

Table \ref{table_kbma_cbsa_compare} compares KBMAs with the metropolitan CBSA in which they have their most populous core. This matching is straightforward for the majority of KBMAs but less obvious for others. For example, the criterion implies comparing both the Boston and Rochester--Dover--Portsmouth KBMAs to the Boston--Cambridge--Quincy CBSA. In consequence, some metropolitan CBSAs, such as Worcester and Manchester--Nashua, have no KBMA compared against them. A complementary appendix table reciprocally compares metropolitan CBSAs with the KBMA that has the most populous core in them.
\begin{comment} six cases of two KBMAs compared to same CBSA
Boston--Worcester, MA--NH—CT and Dover, NH are compared to the Boston-Cambridge-Quincy, MA-NH Metropolitan Statistical Area
Columbus, OH and Newark, OH are compared to the Columbus, OH Metropolitan Statistical Area
Detroit--Ann Arbor, MI and Port Huron, MI are compared to the Detroit-Warren-Livonia, MI Metropolitan Statistical Area
Killeen, TX and Temple, TX are compared to the Killeen-Temple-Fort Hood, TX Metropolitan Statistical Area
Santa Barbara, CA and Santa Maria, CA are compared to the Santa Barbara-Santa Maria-Goleta, CA Metropolitan Statistical Area
Visalia, CA and Porterville, CA are compared to the Visalia-Porterville, CA Metropolitan Statistical Area
\end{comment}


\begin{table}[htb] %KBMAs vs metropolitan CBSAs
\caption{\label{table_kbma_cbsa_compare} \textbf{KBMAs Compared to Matched Metropolitan CBSAs}}
\includegraphics[scale = 0.90, trim = 18mm 186mm 0mm 20mm, clip]{kbma_table_kbma_v_metrocbsa_2025_04_09a.pdf}

\vspace{-5mm}
\begin{flushleft}
\footnotesize{Notes: Each KBMA is compared to the metropolitan CBSA in which it has its most populous core. The summary statistics on relative size and overlap exclude the 19 KBMAs whose largest core is in a micropolitan CBSA. The aggregate statistics compare all KBMAs with all metropolitan CBSAs. A complementary appendix table compares metropolitan CBSAs with the KBMA that has the most populous core in them.}
\end{flushleft}
\end{table}

Most KBMAs have population moderately below that of their comparison CBSA, employment modestly below it, and land area far below it. The median ratio of KBMA size to metropolitan CBSA size is 0.72 for population, 0.87 for employment, and 0.12 for land area. But some KBMAs that have cores in multiple metropolitan CBSAs, such as the Boston KBMA, considerably exceed their comparison CBSA in population.  %[described in discussion of corresponding figure-->] In cases where two [or more] KBMAs are compared to the same CBSA, one typically has population and employment far below that of the comparison.

\begin{comment}

%Some KBMAs that join together multiple cores, such as Boston--Worcester, have population, employment, or land area larger than those of their comparison CBSA.

%sometimes have population, employment, and land area above that of their comparison

%join numerous cores, population

% and employment moderately below that of their comparison CBSA and land area far below it. The median KBMA-to-CBSA ratios of population and employment are 0.75 and 0.87 and the median land ratio is 0.14. As illustrated in the map of Kansas City and its neighboring KBMAs, most KBMAs are fully encompassed by their comparison KBMA.

% For example, the Boston KBMA has population 22 percent above that of the Boston CBSA, reflecting that the KBMA encompasses significant land area outside the CBSA. Even so, only 80 percent of the population in the Boston CBSA is located in the Boston KBMA. The Dover KBMA is also paired with the Boston CBSA, which completely encompasses it. But Dover's population is only a small fraction as large.


%most of the population and employment of their comparison CBSAs but only a small portion of their land area.

%On average, as measured by median, KBMAs captur

 % compares KBMAs to the CBSA in which their most populous core is primarily located. On average, as measured by medians, KBMAs have population and employment that are respectively 75 and 87 percent that of their comparison CBSA and land area that is 14 percent that of their comparison CBSA. In some cases, such as Boston and Los Angeles, the KBMA exceeds the comparison CBSA in population and employment. In other cases, such as Dover NH and Indio, KBMA population and employment are a small fraction those of the comparison CBSA.


% the most populous of the five cores in the Boston--Worcester KBMA is the Boston UAUCA, which lies in the Boston--Cambridge--Quincy CBSA. This pairing criterion is asymmetric in two ways. Two or more KBMAs are sometimes compared to the same CBSA. For example, both the Boston--Worcester and Dover KBMAs are compared to the Boston CBSA. And some CBSAs, such as Worcester and Manchester--Nashua have no KBMA compared against them.} The top portion of Table \ref{table_kbma_cbsa_compare} lists some additional pairings, and reports the corresponding comparisons of population, employment, and land area.

\end{comment}

Figure \ref{pop_kbma_v_cbsa} correspondingly plots the population of KBMAs against their comparison CBSAs. All of the labeled KBMAs with solid markers that have population above their comparison CBSA include at least one core that anchors a different CBSA. All of the labeled KBMAs with solid markers that have population below their comparison are paired to a CBSA in which another KBMA also has a core.

\begin{figure}[htb] %Population: Scatter of KBMAs versus CBSAs
\caption{\label{pop_kbma_v_cbsa} \textbf{Population of KBMAs versus Comparison Metropolitan CBSA}}
%\begin{center}
\includegraphics[scale = 1.05, trim = 10mm 178mm 0mm 19mm, clip]{kbma_figure_pop_kbma_v_cbsa_2025_04_09a.pdf}
%\end{center}

\vspace{-3mm}
\begin{flushleft}
\footnotesize{Notes: Each KBMA is compared to the metropolitan CBSA in which it has its most populous core. The figure excludes the 19 KBMAs whose largest core is in a micropolitan CBSA.}
\end{flushleft}
\end{figure}

\begin{comment} some alternate text

%CBSA  CBSAs whose KBMAs have considerably lower population have portions encompassed by a second KBMA. For example, most of the population of the Riveside--San Bernadino CBSA, too which the Indio KBMA is compared, is encompassed by the Los Angeles--Riverside-Mission Viejo KBMA.

For descriptive purposes, we compare each KBMA to the CBSA in which its most populous core is located. For example, the Boston--Worcester KBMA has five cores spread across three metropolitan CBSAs. Its most populous core, the Boston UAUCA, lies in the Boston--Cambridge--Quincy CBSA and so we compare the Boston--Worcester KBMA to it. As illustrated in Figure \ref{pop_kbma_v_cbsa}, the Boston--Worcester unsurprisingly has population and employment well above its comparison CBS; it also has more land area. Los Angeles--Riverside--Mission Viejo  similarly has larger population and employment than its comparison CBSA (Los Angeles--Long Beach--Santa Ana) but smaller land area. Conversely


% the Los Angeles--Riverside--Mission Viejo KBMA's largest core is the Los Angeles--Long Beach--Santa Ana UAUCA and so we compare it to the Los Angeles--Long Beach--Santa Ana CBSA.
%The maps illustrate that KBMAs and metropolitan CBSAs do not correspond one-to-one. For the purpose of comparison, however, we can assign a single CBSA to each KBMA based on the location of the latter's largest core by population. For example, the Boston-Worcester KBMA's largest core, the Boston UAUCA, is located in the Boston--Cambridge--Quincy CBSA and so we compare the Boston--Worcester KBMA to it. The Dover KBMA's only core is also located in the Boston-Cambridge--Quincy CBSA and so we also compare the Dover KBMA to it. Similarly, the Los Angeles--Riverside--Mission Viejo KBMA is compared to the Los Angeles--Long Beach--Santa Ana CBSA; and the Indio KBMA is compared to the Riverside--San Bernadino CBSA.

%As a starting point for comparison, Figure \ref{pop_fma_v_cbsa} plots the population of the baseline KBMAs against the population of their corresponding metropolitan CBSA. As the match between the two proxies is not one to one, some of the comparisons are between a KBMA and the largest CBSA it overlaps; others are between the largest KBMA that overlaps a CBSA. Almost all KBMAs have population below that of the corresponding CBSA. Almost all KBMAs with population at least moderately above their corresponding CBSA include a core that anchors a second CBSA. For example, a significant portion of the population of the Boston--Worcester--? KBMA is encompassed by the Worcester CBSA....  Conversely, almost all corresponding CBSAs whose KBMAs have considerably lower population have portions encompassed by a second KBMA. For example, most of the population of the Riveside--San Bernadino CBSA, too which the Indio KBMA is compared, is encompassed by the Los Angeles--Riverside-Mission Viejo KBMA.


% compares KBMAs to the CBSA in which their most populous core is primarily located. On average, as measured by medians, KBMAs have population and employment that are respectively 75 and 87 percent that of their comparison CBSA and land area that is 14 percent that of their comparison CBSA. In some cases, such as Boston and Los Angeles, the KBMA exceeds the comparison CBSA in population and employment. In other cases, such as Dover NH and Indio, KBMA population and employment are a small fraction those of the comparison CBSA.
%To address this asymmetry, we include an appendix table that compares each metropolitan CBSA to the KBMA in which its largest core is primarily located.\footnote{This ``reverse'' assignment algorithm leaves XXX relatively small metropolitan CBSAs---all with population below XXX---without a comparison KBMA. The reason is that the tract-based approximation of these metropolitan CBSAs' largest core Urbanized Area has population below the 50,000 threshold required to serve as a core of a KBMA.}



\end{comment}


\begin{figure}[htb]%Commuting Flows of KBMAs and Metropolitan CBSAs
\caption{\label{commuting_flows} \textbf{Commuting Flows: KBMAs and Metropolitan CBSAs}}
\includegraphics[scale = 0.99, trim = 24mm 116mm 24mm 24mm, clip]{kbma_figure_flows_kbmas_cbsas_2025_03_07a.pdf}

\vspace{-7mm}
\begin{flushleft}
\footnotesize{Notes: Inflow rates are measured as a share of employment; outflows rates, as a share of residents who are employed.}
\end{flushleft}
\end{figure}

%\vspace{-10mm} {failed attempt to decrease vertial space between figure immediately above and text}

\subsection{Commuting}

Figure \ref{commuting_flows} shows commuting inflow and outflow rates for KBMAs and metropolitan CBSAs. As illustrated in an appendix figure, the commuting inflows overwhelmingly originate from nearby census tracts that are not attached to a KBMA; commuting outflows split about equally to unattached tracts and to other KBMAs. Measured by the median and 90th percentile rates, KBMAs underperform metropolitan CBSAs in encompassing commuting, unsurprising in the context of the KBMA parameterization's tradeoff to exclude sparsely settled land.

\begin{figure}[tb]%Commuting Flows of kernel-based metropolitan regions and urban areas
\caption{\label{commuting_flows_altparams} \textbf{Commuting Flows: Kernel-Based Metropolitan Regions and Kernel-Based Urban Areas}}
\includegraphics[scale = 0.99, trim = 24mm 118mm 24mm 22mm, clip]{kbma_figure_flows_scatters_kb_alts_2025_03_07a.pdf}
{\normalfont \footnotesize .}

\vspace{-11mm}
\begin{flushleft}
\footnotesize{Notes: Inflow rates are measured as a share of employment; outflow rates, as a share of residents who are employed. The Tracy, CA kernel-based urban area, which has log population of 10.9, has a commuting outflow rate of 0.76, above the displayed range.}
\end{flushleft}
\end{figure}

KBMAs are more competitive by other measures of commuting. The KBMA flow rates have upper envelopes that decline more steeply with population, reflecting that KBMAs more successfully hold down flows for large metropolitan areas. For example, the 90th-percentile inflow rate for KBMAs with population above 500,000 is 0.20, only modestly above the 90th-percentile inflow rate of 0.17 for metropolitan CBSAs with population above 500,000. Even more competitive, the 90th percentile outflow rate for the same set of KBMAs is 0.13, below the outflow rate of 0.17 for the same set of metropolitan CBSAs.

As illustrated in Figure \ref{commuting_flows_altparams}, the kernel-based metropolitan regions considerably hold down commuting flows compared to KBMAs. For example, they achieve median and 90th-percentile inflow rates and a 90th percentile outflow rate below those of metropolitan CBSAs. Commuting flows are considerably higher for kernel-based urban areas.




\subsection{Size}

\begin{figure}[tb]% PDFs of Metropolitan Population
\caption{\label{population_distributions} \textbf{The Distribution of Metropolitan Population}}
%\begin{center}
\includegraphics[scale = 0.98, trim = 17mm 190mm 10mm 20.5mm, clip]{kbma_figure_pdf_pop_2025_03_20a.pdf}
%\end{center}

\vspace{-2mm}
\begin{flushleft}
\footnotesize{Notes: The horizontal axes measure the logarithm of population at the lower bound of bins with width of 0.4 log points. The lower bound of the leftmost bin corresponds to a population of 50,000 for all histograms. The green-dashed histogram in the left panel describes the set of all CBSAs with population of at least 50,000, both metropolitan and micropolitan.}
\end{flushleft}

\end{figure}


Figure \ref{population_distributions} shows the histogram of KBMAs' population: on the left compared against those of metropolitan CBSAs and UAs; on the right, against those of kernel-based metropolitan regions and kernel-based urban areas. The KBMA distribution slopes downward across all of the bins, approximately matching the UA distribution. In contrast, the metropolitan CBSA distribution has frequency that increases across the three lowest bins before turning down. The upward-sloping portion reflects that the scale criterion for metropolitan CBSAs applies to their core population rather than total population. To the extent that population buildouts---the ratio of non-core to core population---have a unimodal distribution, relatively few metropolitan CBSAs will have population only moderately above 50,000. In contrast, the more cleanly truncated set of all CBSAs with population above 50,000, both metropolitan and micropolitan, has population frequency that slopes monotonically  down, approximately matching the KBMA and UA distributions.



As illustrated in the right panel, the population distributions of the three kernel-based parameterizations approximately match each other. This may seem counterintuitive: During the kernel construction stage, the additional iterations associated with a lower threshold commuting strength and a higher allowed separating distance merge together what otherwise would be separate kernels, shifting the population distribution of built-out kernels rightward. But the shift also pushes more built-out kernels above the 50,000 threshold, contributing to keeping the population distribution approximately unchanged. Equally important, as described previously, the iterative calculation of pairwise commuting strength avoids constructing long chains of large cores, limiting the rightward shift at the top of the population distribution. In aggregate, the 435 kernel-based metropolitan regions have 29 percent more population than the 361 KBMAs; the 346 kernel-based urban areas have 9 percent less population.


Figure \ref{fig_pop_rank_v_size} plots the logarithm of the population rank of observations for each of the kernel-based delineations against the logarithm of their population, a standard benchmark for local delineations (e.g., \citeauthor{rosen_resnick_1980}, \citeyear{rosen_resnick_1980}; \citeauthor{gabaix_qje_1999}, \citeyear{gabaix_qje_1999}; \citeauthor{gabaix_ioannides_2004}, \citeyear{gabaix_ioannides_2004}; \citeauthor{soo_2005}, \citeyear{soo_2005}; \citeauthor{rozenfeld_etal_2011}, \citeyear{rozenfeld_etal_2011}). The rank--size relationship is expected to be approximately linear for Pareto distributions, with a slope close to -1 if the distribution is Zipf, a specialization of Pareto \citep{gabaix_qje_1999}.\footnote{A Pareto distribution has CDF, $\mbox{F}(S)\, =\, 1 - (\widetilde {S}/S)^{\zeta} \mbox{ for size } S \ge \widetilde{S}, \, \zeta > 0$. It specializes to a Zipf's distribution when the shape parameter, $\zeta$, equals 1.} Notwithstanding R-squared values very close to 1, the plots are visibly concave.


\begin{figure}[tb]% Population Rank versus Population Size
\caption{\label{fig_pop_rank_v_size} \textbf{Population Rank vs.\ Size}}
%\begin{center}
\includegraphics[scale = 0.93, trim = 24mm 189mm 10mm 17mm, clip]{kbma_figure_pop_rank_v_size_2025_03_07a.pdf}
%\end{center}
% "Tract land area is measured in square miles."
%The absolute value of the slope equals the Pareto coefficient.
\vspace{-7mm}
\begin{flushleft}
\footnotesize{Note: Subtracting 0.5 from the rank improves the fit in the presence of small-sample bias \citep{gabaix_ibragimov}.}
\end{flushleft}
\end{figure}

As emphasized by \cite{eeckhout_2009}, the logarithmic rank--size relationship is peculiar, including distorting the visual relationship across the largest observations. Consistent with this, the high R-squared values of linear fits are misleading, among other reasons because the relationship is monotonic by construction. As reported in the appendix, the Monte Carlo simulations prescribed by \cite{gabaix_ioannides_2004} cannot reject that the largest 50 observations of each of the kernel-based delineations are drawn from a Pareto distribution. But they do reject Pareto at the 0.05 level for the largest 75 observations of each. Similarly, they reject Pareto for the largest 75 metropolitan CBSAs and UAs. And they reject it across a much wider range of kernel-based parameterizations, including permutations that set separate values of $\sigma$ and separate values of $\delta$ for the kernel and buildout stages \href{https://www.kansascityfed.org/research/research-working-papers/size-of-us-metropolitan-areas/}{\citep{rappaport_humann_2023}}.


\begin{comment}  Comparing population of large built-out kernels

see worksheet "top built-out kernels" in kbma_for_fisplay workbook.

% I had considered also comparing the populations for several other built-out kernels. But the spans of some were considerable. For example, the Los Angeles kernel does not join with Riverside-san Bernadino until sigma below 0.40; ditto for San Francisco and San Jose.

\end{comment}


\begin{comment} False starts on explaining decreasing probability density

** FOR NEXT ITERATION, LOWEST MINIMUM DENSITY THRESHOLD SHOULD BE SET ABOVE 0, perhaps 50 or at least 25

%With the exception of parameterizations that impose neither a maximum distance nor a threshold density for joining singletons to kernels ($\sigma^{\prime \prime} = 0$ \emph{and} $\eta = \infty$), all yield histograms that peak at a bin with lower-bound population below 10,000. Moreover

%Similarly, a ``small-core'' parameterization of the KBMA algorithm that sets $\lambda$ to 10,000 and otherwise retains the baseline values has a histogram that peaks at a bin with lower-bound population of 22,400. delineate a set a of KBMAs plausibly consistent with our metropolitan conception have an analog for which population frequency declines from a lower bound below 10,000. The analog for each such parameterization is identical to it except for setting $\lambda$ to 2,500---the minimum for a cluster of census blocks to qualify as an Urban Cluster.  For example, the minimal-core KBMA parameterization, described in the previous section, sets $\lambda$ to 2,500 and otherwise uses the baseline parameter values. It thus serves as an analog for the baseline and small-core parameterizations and also as an implicit analog for itself. Its histogram, illustrated in the right panel of Figure \ref{population_distributions}, peaks at the bin with a lower-bound population of 5,600.

%The only parameterizations we have constructed that delineate KBMAs not plausibly consistent with our conception leave both the maximum distance to singleton tracts and their minimum density unconstrained ($\delta^{\prime \prime} = \inf$ \emph{and} $\eta = 0$). This parameter combination joins some kernels with minimally populated singleton tracts more than 1,000 miles away, which we judge as violating the criterion that metropolitan areas are made up of locations that are nearby. \textbf{For example, a specific combination}


%\footnote{We refrain from setting $\lambda$ below 2,500, reflecting that the UAUCs are defined to have a minimum population of 2,500. s our cores approximate have a minuimum population of 2,500 UAUCs are defined to have a minimum population of 2,500 and so we limit our limit our  by definition  and so we limit our analysis to UAUCA cores with population of at least 2,500, to While some our UAUCAs have population below 2,500}

%For example, the minimal-core KBMAs, described in the previous section, are constructed setting $\lambda$ to 2,500 and using the baseline values for all other parameters. All of the exceptions---i.e., the parameterizations that set $\lambda$ to 2,500 and for which the bin with peak frequency has a lower-bound population above 10,000---leave both the distance and the density for joining singleton tracts unconstrained But

%Analogously, we compare the baseline KBMA parameterization against a  ``small-core'' parameterization that sets $\lambda$ to 10,000: The median population of the baseline KBMAs (215,000) comfortably exceeds the  population of the largest small-core KBMA that has no core with population above 50,000 (173,000). Similarly, we compare the small-core KBMAs against a ``minimal-core'' parameterization that sets $\lambda$ to 2,500. In this case, the median population of the small-core KBMAs comfortably exceeds the population of the largest minimal-core KBMA that has no core with population above 10,000.

The largest minimal-core KBMA with no core above 50,000 is Arroyo Grande--Grover Beach, CA, pop 172,561; it has 5 cores with mean core pop of 27,203.

how to: sort luauca final or sheets in sumstats workbooks decreasing in kbma pop
add a column that has avg pop per luauca: so a necessary condition to have no cores >= 50k is that avg pop per core <50k.  Then you can check for the highest pop with avg core < 50k: do all of the cores have <50; for 2.5k cores, the highest pop with avg < 50k is Utica NY, which has 4 cores (at least of one of which has pop>50k)...next highest pop with avg core <50k is Arroyou Grande, which has 5 cores, all of which check out as having pop < 50k.

* The largest minimal-core KBMA with no core above 10,000 is Helena MT, pop 31,353, 1 core with pop 8,570.
* The median population of small-core KBMAs (kernel pop = 10,000) is 51,115


The largest minimal core KBMA with no core that has pop >= 10k must have pop no higher than 22,971 (Greenville Texas, which has 2 cores and is the largest minimal-core pop with total kernel pop <= 20,000 (so largest pop with possibility that both cores are below 10k)

A large literature debates the stochastic distribution that best describes  realized population across geographic units. We argue that the KBMA family of delineations is consistent with underlying probability density functions that decline as population increases above the range subject to truncation bias. In addition, a sufficiently low population threshold for cores, $\lambda$, can keep the biased range below the minimum population that we judge meets our metropolitan conception of moderate scale.\footnote{The minimal-core parameterization has median population of 18,139, which we judge falls short of moderate scale. Separately, the KBMA algorithm relies on the threshold size for cores, $\lambda$, to achieve the moderate-scale criterion of our metropolitan conception. Instead, we could append a sixth stage, requiring built-out KBMAs to have population above a specified minimum. Parameterizations with the core threshold set below 2,500 would be possible with access to the clusters of blocks delineated by the Census Bureau that fall short of the population threshold to qualify as a UC.}




\end{comment}

A benchmark alternative hypothesis is that population distributions are lognormal over their entire range of observations, spanning from very low levels of population to the largest. For example, \citet{eeckhout_2004} shows that the union of municipalities and census designated places in 2000, which includes locations with population as low as 1, is well approximated by a lognormal population distribution with peak frequency at a population close to 1,400.\footnote{Census designated places are statistical entities consisting of a closely settled, locally recognized concentration of population that is identified by name. Designation relies heavily on the opinions of local residents, organizations, and officials \citep{census_1997_cdps}. This reliance induces a selection bias, specifically that local support for designation is likely to be stronger for locations with larger population.} One reason lognormal is important is that it is asymptotically implied by stochastic processes that satisfy Gibrat's law, having growth rates that are uncorrelated with initial levels \citep{eeckhout_2004,Lee_Li_2013}.

We are skeptical. As illustrated in the appendix, the frequency distribution of the union of UAs and UCs is convex downward sloping from its minimum population of 2,500. The 3,610 UAs and UCs in 2000 together accounted for only 79 percent of the U.S.\ population, at minimum requiring tens of thousands of smaller clusters to encompass the remaining 59 million residents. Similarly, the union of municipalities and census designated places in \citet{eeckhout_2004} accounted for only 71 percent of the U.S.\ population \citep{Levy_2009}. Bolstering our skepticism, more recent research finds that population growth rates of various geographical delineations were historically correlated with population levels \citep{holmes_lee_2010, dittmar_2011,michaels_et_al_2012,desmet_rappaport_jue_2017,rappaport_2018}. For example,  \href{https://www.kansascityfed.org/research/economic-review/4q18-rappaport-faster-growth-larger-less-crowded-locations/}{\cite{rappaport_2018}} documents that intermediate-sized locations grew fastest during recent decades. %For example, \cite{rappaport_2018} argues that shifts in the fundamental benefits and costs of size have contributed to intermediate-sized locations having the fastest population growth during recent decades.


\begin{comment} superseded paragraph on problems of log normal distributions
We reconcile our results with arguments that the geographic distribution of population is log normal by pointing to other truncation biases. For example, the union of incorporated municipalities and Census Designated Places in \cite{eeckhout_2004, eeckhout_2009} include observations with only 1 resident. But as noted by \cite{Levy_2009}, the observations together account for only 73 percent of the U.S.\ population, with most remaining U.S.\ residents likely to be living in places with very low population.\footnote{The Census Bureau, with consultation from local officials, historically delineated Census Designated Places (CDPs) with the intent to create analogs of incorporated municipalities. In preparation for Census 2000, it dropped population as a criterion. The remaining criteria for designation as a CDP in 2000 leave considerable scope for subjectivity \citep{census_1997_cdps}. The propensity of any non-municipal location to be judged a CDP is probably increasing in its population.} And the bottom portions of the lognormal distributions in \cite{desmet_rappaport_jue_2017} are made up of U.S.\ counties, which historically have been delineated endogenously: Newly admitted  U.S.\ states typically had only a handful of counties, which were subsequently split as they became more settled \citep{thorndale_dollarhide_1987}. One consequence is that vast regions of sparsely settled land, especially in western states, were divided into only a handful of counties, depressing the number of counties with low population compared to a less skewed partition of land.\footnote{On the other hand, \cite{holmes_lee_2010} partition the United States into more than 85,000 grid cells, each 6 miles on a side. After truncating the 17 percent of cells that have no population, they find that population frequency increases over more than half of its distribution, peaking in the bin with population between 100 and 1,000. Even so, they reject that the population distribution is consistent with log normal.}

\end{comment}


% *** vvvv HISTORIAGRAPHY OF MY THINKINING ON POP DISTRIBUTIONS vvv ***

\begin{comment} Previous text and false starts for this (January 2025) revision

%In contrast, we can always reject that the upper tails of KBMA population distributions are Pareto, contrary to a competing benchmark stylization (e.g., \citeauthor{rosen_resnick_1980}, \citeyear{rosen_resnick_1980}; \citeauthor{gabaix_qje_1999}, \citeyear{gabaix_qje_1999}; \citeauthor{gabaix_ioannides_2004}, \citeyear{gabaix_ioannides_2004}; \citeauthor{soo_2005}, \citeyear{soo_2005}; \citeauthor{rozenfeld_etal_2011}, \citeyear{rozenfeld_etal_2011}).\footnote{A Pareto distribution has CDF, $\mbox{F}(S)\, =\, 1 - (\widetilde {S}/S)^{\zeta} \mbox{ for size } S \ge \widetilde{S}, \, \zeta > 0$. It specializes to a Zipf's distribution when the shape parameter, $\zeta$, equals 1.}

%NOT REALLY TRUE: which can be interpreted as observations having log population that is equally bunched throughout its distribution.

%The rank-size scatters in Figure \ref{fig_pop_rank_v_size} plot the logarithm of observations' population rank against the logarithm of their population for the baseline parameterization and two of the alternative benchmarks. Pareto distributions imply an expected relationship that is linear. After excluding observations in the bottom half of the distributions to purge the truncation bias (gray markers), the fitted linear relationships are nominally tight, with R-squared values close to 0.97. But the rank-size relationship is monotonic by construction, thereby requiring a recalibration of the qualitative tightness implied by R-squared values. More importantly, the red rank-size scatters are visibly concave, a critique also made by \cite{eeckhout_2004} for metropolitan CBSAs and by \cite{black_henderson_2003} for the metropolitan statistical areas delineated following the 1990 decennial census. Equivalently, log population is more bunched at the top of its distribution than at the bottom compared to a Pareto benchmark. As is visibly evident, this concavity is even stronger if the observations in the bottom half of the distribution are included.

%upon 20 trillionth reading: Gabaix acknowledges that the slope of the rank-size becomes less than 1 in absolute value as you include more cities in the lower part of the distribution. (says for United States 100,000 or less); he attributes to growth that has a higher variance for smaller cities. Says look at Dobkins and Ioniddes. I don't understand the specification in Veneri (normalizes the quadratic term). But it wouldn't effect the coefficient on the quadratic term, which can be seen by explicitly squaring the normalized term.  The different estimates essentially capture the ratio of the nth smallest to the largest: the higher it is, the steeper the slope.


%In contrast, we can always reject that the upper tails of KBMA population distributions are Pareto, contrary to a competing benchmark stylization (e.g., \citeauthor{rosen_resnick_1980}, \citeyear{rosen_resnick_1980}; \citeauthor{gabaix_qje_1999}, \citeyear{gabaix_qje_1999}; \citeauthor{gabaix_ioannides_2004}, \citeyear{gabaix_ioannides_2004}; \citeauthor{soo_2005}, \citeyear{soo_2005}; \citeauthor{rozenfeld_etal_2011}, \citeyear{rozenfeld_etal_2011}).\footnote{A Pareto distribution has CDF, $\mbox{F}(S)\, =\, 1 - (\widetilde {S}/S)^{\zeta} \mbox{ for size } S \ge \widetilde{S}, \, \zeta > 0$. It specializes to a Zipf's distribution when the shape parameter, $\zeta$, equals 1.}

%NOT REALLY TRUE: which can be interpreted as observations having log population that is equally bunched throughout its distribution.

%The rank-size scatters in Figure \ref{fig_pop_rank_v_size} plot the logarithm of observations' population rank against the logarithm of their population for the baseline parameterization and two of the alternative benchmarks. Pareto distributions imply an expected relationship that is linear. After excluding observations in the bottom half of the distributions to purge the truncation bias (gray markers), the fitted linear relationships are nominally tight, with R-squared values close to 0.97. But the rank-size relationship is monotonic by construction, thereby requiring a recalibration of the qualitative tightness implied by R-squared values. More importantly, the red rank-size scatters are visibly concave, a critique also made by \cite{eeckhout_2004} for metropolitan CBSAs and by \cite{black_henderson_2003} for the metropolitan statistical areas delineated following the 1990 decennial census. Equivalently, log population is more bunched at the top of its distribution than at the bottom compared to a Pareto benchmark. As is visibly evident, this concavity is even stronger if the observations in the bottom half of the distribution are included.

%upon 20 trillionth reading: Gabaix acknowledges that the slope of the rank-size becomes less than 1 in absolute %value as you include more cities in the lower part of the distribution. (says for United States 100,000 or less); he attributes to growth that has a higher variance for smaller cities. Says look at Dobkins and Ioniddes. I don't understand the specification in Veneri (normalizes the quadratic term). But it wouldn't effect the coefficient on the quadratic term, which can be seen by explicitly squaring the normalized term.  The different estimates essentially capture the ratio of the nth smallest to the largest: the higher it is, the steeper the slope.



%As suggested by the nearly identical histograms, the rank size relationship of population for the kernel-based urban areas and kernel-based is nearly indistuinghisable to the relationship for the KBMAs.

%Quadratic rank-size regressions, also using only the top half of observations, boost R-squared values almost to 1 and confirm the statistical significance of the concavity....
\end{comment}

\begin{comment}Table reporting quadratic regressions \ref{table_pop_rank_v_size} reports results for the baseline and alternative benchmark

parameterizations.\footnote{The idiosyncratic nature of the rank-size relationship causes OLS standard errors to considerably understate uncertainty.
As suggested by  \citet{gabaix_ioannides_2004}, we run 100,000 Monte Carlo simulations, each of which draws from a Pareto distribution the same number of observations used in the estimation and then regresses log rank on linear and quadratic log population. The resulting distribution of simulated t-statistics for the quadratic coefficient does not depend on the Pareto shape parameter. Table \ref{table_pop_rank_v_size} reports the critical value at which the CDF equals 0.025.} \setcounter{mcnote}{\thefootnote} The lower rows document that the fitted rank-size relationship also considerably differs from Pareto in the \emph{upper tail} of population distributions. As shown in the right-most columns, we can similarly reject a linear rank-size relationship for the top half of metropolitan CBSAs and for all UAs (all, because they are cleanly truncated). We can also reject linearity for all other KBMA parameterizations we have constructed, including an additional 57 reported in Appendix E. We thus conclude that the data process generating the U.S.\ system of metropolitan areas importantly differs from DGPs that generate a Pareto distribution.


\begin{table}[tb] %Quadratic Rank-Size Regressions
\hspace{30mm}
\includegraphics[scale = 0.88, trim = 17mm 138mm 0mm 18mm, clip]{kbma_table_pop_rank_v_size_2023_06_05a.pdf}
\vspace{-8mm}
\caption{\label{table_pop_rank_v_size} \textbf{Quadratic Rank-Size Regressions}.
{\normalfont \footnotesize \hspace{-0.5mm} Table reports results from regressing log(rank - 0.5) on linear and quadratic log(population). Subtracting 0.5 from the rank improves the fit in the presence of small-sample bias \citep{gabaix_ibragimov}. The enumerated smallest population is the minimum among observations included in the regression. Statistical significance of the quadratic coefficient is based on Monte Carlo simulations described in footnote \themcnote . **: differs from 0 at 0.05 level; ***: at 0.01 level. \dag: The bottom rows report the fitted slope at benchmark ranks based on quadratic regressions that use only the 64 largest observations; the underlying coefficients differ from those reported in the table. The baseline parameterization sets $\lambda=$ 50,000; $\sigma ,\sigma^{\prime},\sigma^{\prime \prime}=0.25$; $\delta ,\delta^{\prime},\delta^{\prime \prime}=20$; and $\eta=200$.}}\vspace{-4mm}
\end{table}

\end{comment}

\begin{comment} elaboration on idiosyncracy of rank-size relationship

%For example, an exact uniform distribution---arguably, the antithesis of a Pareto distribution---with the same number of observations over the same population range as the top half of the baseline parameterization yields an R-squared value of 0.77.

% However, as noted by \cite{eeckhout_2009}, the rank-size comparison is highly idiosyncratic. For example, it has exceptionally wide uncertainty bounds for ranks with low numbers. Depending on the normalization, this implies either considerable uncertainty about the distribution of the largest units or the smallest ones.
%Similarly, the rank-size relationship for UAs, which are cleanly truncated, is visibly concave.
% uniform distribution R2 in jordan_uniform_sim.2023_04_18a.xlsx

\end{comment}

\begin{comment} Explicit CDF for Pareto and Zipfs and additional reference to rossi-hansberg_wright article

%Realized population distributions are closely approximated by a linear rank-sizerelationship for a variety of geographic units within numerous countries at numerous points in time (e.g., \citeauthor{rosen_resnick_1980}, \citeyear{rosen_resnick_1980}; \citeauthor{dobkins_ioannides_2000}, \citeyear{dobkins_ioannides_2000};  \citeauthor{rossi-hansberg_wright_2007}, \citeyear{rossi-hansberg_wright_2007}).


%The estimated slope of the linear relation is typically close to -1, a special case of a Pareto distribution that satisfies Zipf's law.\footnote{A Pareto distribution has CDF, $\mbox{F}(S)\, =\, 1 - (\widetilde {S}/S)^{\zeta} \mbox{ for size } S \ge \widetilde{S}, \, \zeta > 0$. A Zipf's distribution has shape parameter, $\zeta$, equal to 1.}  But we can always reject a linear relationship.

\end{comment}

\begin{comment} %previous version description of debate on size distribution

A large literature debates the stochastic distribution that best describes  population across geographic units. One group of papers argues that population levels are most consistent with a Pareto distribution (e.g., \citeauthor{gabaix_qje_1999}, \citeyear{gabaix_qje_1999}; \citeauthor{gabaix_ioannides_2004}, \citeyear{gabaix_ioannides_2004}; \citeauthor{rozenfeld_etal_2011}, \citeyear{rozenfeld_etal_2011}). A Pareto distribution has a probability density function that declines over its entire domain and an expected linear relationship between the log of observations' population rank and the log of their population level.  Realized population distributions are closely approximated by a linear rank-size relationship for a variety of geographic units within numerous countries at numerous points in time (e.g., \citeauthor{rosen_resnick_1980}, \citeyear{rosen_resnick_1980}; \citeauthor{dobkins_ioannides_2000}, \citeyear{dobkins_ioannides_2000}; \citeauthor{soo_2005}, \citeyear{soo_2005}; \citeauthor{rossi-hansberg_wright_2007}, \citeyear{rossi-hansberg_wright_2007}). The estimated slope is typically close to -1, a special case of a Pareto distribution that satisfies Zipf's law.\footnote{A Pareto distribution has CDF, $\mbox{F}(S)\, =\, 1 - (\widetilde {S}/S)^{\zeta} \mbox{ for size } S \ge \widetilde{S}, \, \zeta > 0$. A Zipf's distribution has shape parameter, $\zeta$, equal to 1.} Other papers agree that the distribution over geographic units with the largest levels of population may be consistent with Pareto but that the distribution over units that span a wider population range is lognormal \citep{eeckhout_2004, eeckhout_2009, Lee_Li_2013, desmet_rappaport_jue_2017}. %\cite{ioannides_skouras_jue_2013} agree that the distribution over units with low population is consistent with lognormal but that the distribution over those with high population is not.

We argue that U.S.\ metropolitan areas, as proxied by a very wide range of KBMA parameterizations, have an underlying population distribution that fundamentally differs from both Pareto and lognormal. The KBMA family of delineations have population frequencies that peak near their smallest observations, which is inconsistent with lognormal, especially after accounting for a truncation bias. And statistical tests based on Monte Carlo simulations reject that the relationship between log rank and log population is linear, instead finding that population is more bunched at the top of distributions than at the bottom.

More substantively, the size distributions of \emph{all} KBMA parameterizations that we have delineated imply that population growth rates across the KBMAs of a parameterization were correlated with their contemporary population. If their stochastic growth had instead been independent of population in mean and variance (i.e., that it been ``orthogonal'' or, equivalently, characterized by Gibrat's law), their asymptotic population distribution would be lognormal \citep{eeckhout_2004}. Moreover, the entry of new KBMAs over time (i.e., geographic clusters with population increasing above the parameterized threshold to serve as a core) together with orthogonal growth implies the asymptotic population distribution above an appropriately chosen lower bound would be Zipf's \citep{gabaix_qje_1999, blank_solomon_2000}.

Instead, the rank-size relationships at the very top of KBMA distributions are much too steep to be consistent with Zipf's law, implying that the population growth of the largest KBMAs in a parameterization has been characterized by convergence. Conversely, the rank-size relationships at the bottom of KBMA distributions are much too flat to be consistent with Zipf's law, suggesting that the population growth of smaller KBMAs has been characterized by divergence. This pattern of diverging population across smaller KBMAs and converging population across the largest KBMAs matches the pattern of metropolitan CBSA growth from 1940 to 2017 \citep{desmet_rappaport_jue_2017, rappaport_2018}.

We reconcile our results with arguments that the geographic distribution of population is log normal by pointing to other truncation biases. For example, the union of incorporated municipalities and Census Designated Places in \cite{eeckhout_2004, eeckhout_2009} include observations with only 1 resident. But as noted by \cite{Levy_2009}, the observations together account for only 73 percent of the U.S.\ population, with most remaining U.S.\ residents likely to be living in places with very low population.\footnote{The Census Bureau, with consultation from local officials, historically delineated Census Designated Places (CDPs) with the intent to create analogs of incorporated municipalities. In preparation for Census 2000, it dropped population as a criterion. The remaining criteria for designation as a CDP in 2000 leave considerable scope for subjectivity \citep{census_1997_cdps}. The propensity of any non-municipal location to be judged a CDP is probably increasing in its population.} And the bottom portions of the lognormal distributions in \cite{desmet_rappaport_jue_2017} are made up of U.S.\ counties, which historically have been delineated endogenously: Newly admitted  U.S.\ states typically had only a handful of counties, which were subsequently split as they became more settled \citep{thorndale_dollarhide_1987}. One consequence is that vast regions of sparsely settled land, especially in western states, were divided into only a handful of counties, depressing the number of counties with low population compared to a less skewed partition of land.\footnote{On the other hand, \cite{holmes_lee_2010} partition the United States into more than 85,000 grid cells, each 6 miles on a side. After truncating the 17 percent of cells that have no population, they find that population frequency increases over more than half of its distribution, peaking in the bin with population between 100 and 1,000. Even so, they reject that the population distribution is consistent with log normal.}


\end{comment}

\begin{comment} Discussion of literature on stochastic distribution
%A large literature debates the stochastic distribution that best describes  population across geographic units. One group of papers argues that population levels are most consistent with a Pareto distribution (e.g., \citeauthor{gabaix_qje_1999}, \citeyear{gabaix_qje_1999}; \citeauthor{gabaix_ioannides_2004}, \citeyear{gabaix_ioannides_2004}; \citeauthor{rozenfeld_etal_2011}, \citeyear{rozenfeld_etal_2011}). A Pareto distribution has a probability density function that declines over its entire domain and an expected linear relationship between the log of observations' population rank and the log of their population level.  Realized population distributions are closely approximated by a linear rank-size relationship for a variety of geographic units within numerous countries at numerous points in time (e.g., \citeauthor{rosen_resnick_1980}, \citeyear{rosen_resnick_1980}; \citeauthor{dobkins_ioannides_2000}, \citeyear{dobkins_ioannides_2000}; \citeauthor{soo_2005}, \citeyear{soo_2005}; \citeauthor{rossi-hansberg_wright_2007}, \citeyear{rossi-hansberg_wright_2007}). The estimated slope is typically close to -1, a special case of a Pareto distribution that satisfies Zipf's law.\footnote{A Pareto distribution has CDF, $\mbox{F}(S)\, =\, 1 - (\widetilde {S}/S)^{\zeta} \mbox{ for size } S \ge \widetilde{S}, \, \zeta > 0$. A Zipf's distribution has shape parameter, $\zeta$, equal to 1.} Other papers agree that the distribution over geographic units with the largest levels of population may be consistent with Pareto but that the distribution over units that span a wider population range is lognormal \citep{eeckhout_2004, eeckhout_2009, Lee_Li_2013, desmet_rappaport_jue_2017}. %\cite{ioannides_skouras_jue_2013} agree that the distribution over units with low population is consistent with lognormal but that the distribution over those with high population is not.

We argue that U.S.\ metropolitan areas, as proxied by a very wide range of KBMA parameterizations, have an underlying population distribution that fundamentally differs from both Pareto and lognormal. The KBMA family of delineations have population frequencies that peak near their smallest observations, which is inconsistent with lognormal, especially after accounting for a truncation bias. And statistical tests based on Monte Carlo simulations reject that the relationship between log rank and log population is linear, instead finding that population is more bunched at the top of distributions than at the bottom.

More substantively, the size distributions of \emph{all} KBMA parameterizations that we have delineated imply that population growth rates across the KBMAs of a parameterization were correlated with their contemporary population. If their stochastic growth had instead been independent of population in mean and variance (i.e., that it been ``orthogonal'' or, equivalently, characterized by Gibrat's law), their asymptotic population distribution would be lognormal \citep{eeckhout_2004}. Moreover, the entry of new KBMAs over time (i.e., geographic clusters with population increasing above the parameterized threshold to serve as a core) together with orthogonal growth implies the asymptotic population distribution above an appropriately chosen lower bound would be Zipf's \citep{gabaix_qje_1999, blank_solomon_2000}.


\end{comment}

\begin{comment}  * literature on rank-size, pareto

%Eeckhout:

% Gabaix: gibrat--> Zipf's (pareto with exponent -1); generalizes to allow for entry. But needs a "grain of sand" for there to be a steady state distribution: a lower bound on size close to 0. A mechanism to keep small cities from becoming too small. "random walk with (lower) barrier". the closer barrier is to 0, the closer the pareto coefficient will be to -1.

% Zipf's law: technically is that for a large size S, P(Size > S) = a/S (i.e.: (prob that size >= S) * size = constant); Rank-size rule is an approximate way of stating Zipfs law.  A power law (aka Pareto law) has P(size >= S) ~ a/(S^zeta) for large S. (zeta is "power law exponent" aka "pareto exponent"); Gibrats:  the growth rate of entities of size S has distribution function with mean and variance independent of S.

% Gabaix and Ioannides: "Predicting a value in a range say [0.8 1.2] may be included in the list of criteria used to judge the success of urban theories. Agree that there are genuine doubts about whether Zipf's law as a description of the entire city size distribution. Black and Henderson as one example.

% significance of Zipf vs Pareto:

%\ref{rozenfeld_etal_2011} argue that cluster population well described by a power law [pareto distribution] for clusters with population above 12,000 (based on statistical tests) and above 3,000 (visually)

%Blank and Solomon, Physica A, 2000:
Gabaix wrong! But you can get Pareto if there is entry of new locations (i.e., not the case that the # of metros is fixed, which could take the form of time-invariant lower bound on population plus a growing number of locations). But there are additional conditions in terms of choosing the size threshold and the rate of location entry to get Zipfs specialization (and perhaps also Pareto).

\end{comment}

\begin{comment}  * Alternative wording summarizing pop distribution results

%A wide range The family of KBMA delineations suggests that the probability density function of U.S. metropolitan areas' population slopes downward across its entire range, even starting from levels arguably far too low to meet the moderate-scale criterion. Similarly, the PDF of UAUCs' population slopes downward across its entire range. Together we interpret these as suggesting that distribution of \emph{all} population clusters in the United States is unlikely to be distributed lognormal. We also argue that the distribution does not falls off too slowly to be consistent with a Pareto distribution, even along its upper tail.

A standard benchmark for describing the upper tail of a population distribution is to estimate how well it is described by a Pareto distribution \citep[e.g.,][]{gabaix_qje_1999, gabaix_ioannides_2004, eeckhout_2004}. The Pareto CDF is given by, $\mbox{F}(S)\, =\, 1 - (\widetilde {S}/S)^{\zeta} \mbox{ for size } S \ge \widetilde{S}, \, \zeta > 0$. Let $N$ denote the number of observations with $S \ge \widetilde{S}$ and $r$ denote the rank of each observation in descending order of $S$ (i.e., the observation with highest $S$ has rank 1). The rank of an observation with population $S$ is approximated by $N \cdot (1-\mbox{F}(S))$, which in turn implies that $\log(r) \approx k - \zeta \log(S)$, where $k$ is a constant. Thus, the absolute value of the slope coefficient from regressing log rank on log size estimates the Pareto coefficient, $\zeta$. Importantly, \cite{gabaix_ioannides_2004} document that OLS standard errors from such rank-size regressions vastly understate the degree of uncertainty, probably reflecting the positive correlation among residuals introduced by ranking observations.
%\footnote{\cite{gabaix_ioannides_2004} note that these approximations are a good fit for observations of high rank but not for those of the highest rank. For example, doubling the total number of observations doubles the expected number of observations with size above any $S \ge \widetilde{S}$ but has little predictive power for the size relationship between the two largest observations.}

A standard benchmark for describing the upper tail of a population distribution is to estimate how well it is described by a Pareto distribution \citep[e.g.,][]{gabaix_qje_1999, gabaix_ioannides_2004, eeckhout_2004}. The Pareto CDF is given by, $\mbox{F}(S)\, =\, 1 - (\widetilde {S}/S)^{\zeta} \mbox{ for size } S \ge \widetilde{S}, \, \zeta > 0$. Let $N$ denote the number of observations with $S \ge \widetilde{S}$ and $r$ denote the rank of each observation in descending order of $S$ (i.e., the observation with highest $S$ has rank 1). The rank of an observation with population $S$ is approximated by $N \cdot (1-\mbox{F}(S))$, which in turn implies that $\log(r) \approx k - \zeta \log(S)$, where $k$ is a constant. Thus, the absolute value of the slope coefficient from regressing log rank on log size estimates the Pareto coefficient, $\zeta$. Importantly, \cite{gabaix_ioannides_2004} document that OLS standard errors from such rank-size regressions vastly understate the degree of uncertainty, probably reflecting the positive correlation among residuals introduced by ranking observations.
%\footnote{\cite{gabaix_ioannides_2004} note that these approximations are a good fit for observations of high rank but not for those of the highest rank. For example, doubling the total number of observations doubles the expected number of observations with size above any $S \ge \widetilde{S}$ but has little predictive power for the size relationship between the two largest observations.}


\end{comment}

\begin{comment}  possible footnote: would need to construct what a normal distribution would be with the 3,610 UAUCs as its upper tail.


\footnote{Of course, we cannot rule out that the populations of UAUCs constitute the upper tail of a distribution that has peak frequency at some intermediate level. But if so, the total number of locations would need to be extremely high: The 3,610 UAUCs accounted for 79 percent of U.S. residents in 2000.  Partitioning the remaining 59 million U.S. residents into clusters of 2,499 each would require more than 23,000 additional locations.

Partitioning them in a way  Splitting them into settlements inthe remaining residents in a way
remaining residents into clusters of 2,499 persons each would require more than 23,000 locations. Splitting remaining residents in a way that approximately half live  in a way that
More than 23,000 additional locations would be required to split the remaining 59 million U.S.\ residents into settlments of 2,499 persons each.

splitting them in a way consistent with a log normal distribution would require many times more locations. So there would need to be tens of thousands of locations with population between the level with peak frequency and 2,500.}

\end{comment}

\begin{comment}  * Alternative attempts at describing population frequency of baseline KBMAs, CBSAs, UAs, UAUCs, ...

fma.cbsa2003_density.2020_05_07a.xlsx:
221 micropolitan CBSAs out of 560 have pop >=50k

% !!!!!  The micropolitan CBSA with the largest pop (Torrington CT) has population equal to the 43rd percentile of the Metropolitan CBSAs. So using the top half is likely to be clean (unless there are CBSA equivalents with core below 10,000 and pop above the median).

%The population distributions of tzhe baseline KBMAs, metropolitan CBSAs, and UAs each have frequency that decreases with size over most of their population domain. As illustrated in the left panel of Figure \ref{population_distributions}, the frequency of UAs peaks at the lowest population bin, which includes observations with population from 60,000 to 89,000. The frequencies of the baseline KBMAs and metropolitan CBSAs peak at the second lowest population bin, which includes observations with population up to midpoint of 133,000.

%The initial steps upward in the frequency of KBMAs and CBSAs are misleading, as both delineations arguably have ``missing observations'' in the bottom portion of their population distribution. Specifically, both delineations are truncated from below based on core population rather than total population. Numerous implicit observations, constructed identically except for having core population below the specified threshold, have total population considerably above that threshold. More importantly, many of these implicit observations have total population above that of some explicit observations. For example, almost half of micropolitan CBSAs, which have cores with population between 10,000 and 50,000, have total population above that of the smallest metropolitan CBSA. We thus argue that delineations that use relatively low thresholds for core population are

%For example,  as illustrated in the right panel, the union of micropolitan and metropolitan CBSAs has population frequency that slopes downward from a bin with midpoint of 49,000. The population frequency of KBMAs parameterized with a threshold core of 2,500 slopes downward from a bin with midpoint below 10,000. UAUCs, which are cleanly truncated at 2,500, have population frequency that declines along its entire domain.

We argue that a truncation bias contaminates using CBSAs and KBMAs to estimate the stochastic process generating metropolitan size in the sense that the realized size range has ``missing'' observations. The ambiguity of what constitutes ``moderate scale'' suggests to us that an unbiased estimate of the size distribution above \emph{any} specific value should be invariant to \emph{decreasing} the postulated lower bound determining size. Metropolitan delineations built around a core with specified lower bound violate this condition. For example, there is considerable overlap between the population ranges of metropolitan and micropolitan CBSAs. As a result, an estimate of the underlying size distribution of metropolitan areas with population above 50,000 based only on  metropolitan CBSAs---which by construction have cores exceeding 50,000---will differ from an estimate of the underlying size distribution of metropolitan areas with population above 50,000 based on all CBSAs---which by construction have cores exceeding 2,500.

%By construction, the population distributions for all three are truncated from below at a threshold of 50,000. For the KBMAs and metropolitan CBSAs, this truncation also pushes down the frequency in the lower bins because of latent cores with population below the threshold that would have population above it when built out. More generally, the baseline KBMAs should be thought of as the observations with the highest population among a superset of areas that approximate our metropolitan conception shorn of its criterion that population exceeds a moderate threshold. For land area, the KBMA distribution is shifted to the right from the UA distribution, reflecting both the joining of cores to form kernels and the buildout from kernels into surrounding tracts. This still leaves the distribution of KBMA land area 2 log points to the left of the distribution of metropolitan CBSA land area.
%\footnote{For example, Micropolitan CBSAs, which have cores with population from 10,000 to 50,000, belong to a superset of locations that includes metropolitan CBSAs. The population range of the largest third of micropolitan CBSAs overlaps the population range of the smallest third of metropolitan CBSAs.}


By construction, KBMAs have total population weakly above the threshold, $\lambda$, for a UAUCA to be a core. But using $\lambda$ as the bound above which to estimate the underlying population distribution introduces a truncation bias in the sense that the realized distribution has ``missing'' observations in its lower portion.


introduces a truncation bias in the sense that lowering the core threshold from $\lambda_0$ to $\lambda_1$ affects the distribution of population above $\lambda_0$: $\Theta (n_i \ge \lambda_0 | \lambda = \lambda_0)$ differs from $\Theta (n_i \ge \lambda_0 | \lambda = \lambda_1 < \lambda_0)$. In essence, the population distribution of KBMAs parameterized with $\lambda_0$  has ``missing observations''. The same critique applies to the distribution of metropolitan CBSAs, reflected by the large overlap between their population range and that of micropolitan CBSAs (which are anchored by UAUCs with population between 10,000 and 50,000 rather than by UAUCs with population above this).

Fortunately, this truncation bias can be avoided by measuring the distribution of population above a bound sufficiently higherthan the core threshold. For example, the median population of metropolitan CBSAs (223,000) comfortably exceeds the maximum population of micropolitan CBSAs (182,000), suggesting that the population distribution of \emph{metropolitan} CBSAs with population above the median is free of truncation bias.\footnote{This would be true if no CBSA constructed around a core UAUC with population below 10,000 had population above the metropolitan median.} In essence, any missing observations of the metropolitan distribution would have population below the median. Similarly, we set $\lambda$ to 2,500 to construct an alternative set of KBMAs that allows for small cores. Among KBMAs in this set that have no cores above 50,000, the maximum population is XXXX, comfortably below the median population of the baseline KBMAS. This suggests that the population distribution of the baseline KBMAs with population above the median is free from truncation bias. By analogy, we hypothesize that the population distribution of the alternative KBMAs is free from truncation bias for population above its median value.



\end{comment}

\begin{comment} alt wording: reject lognormal corrob by clean truncation delineations
%The rejection of lognormal is corroborated by the population distributions of several geographic delineations that are truncated cleanly based on total population. For example, UAUCs (and hence UAs) have a population frequency that slopes downward over their entire population domain. The frequency of the U.S.\ clusters constructed by \citet{rozenfeld_etal_2008, rozenfeld_etal_2011} remains near its peak for population up to about 3,000 and then turns downward; the frequency of their U.K.\ clusters has the same shape, turning down at a population of about 300.
%Rozenfeld et al.\ (\citeyear{rozenfeld_etal_2008, rozenfeld_etal_2011})
\end{comment}

\begin{comment} log normal distribution stuff

%More than one third of micropolitan CBSAs had population above 50,000 in 2000; six had population above 150,000.

%Importantly, it is unclear whether the right-hand portions of the various population and land distributions are consistent with the being the right tail of a log normal distribution. The latter is a natural benchmark because it is asymptotically implied if growth rates are uncorrelated with size. As documented by \cite{eeckhout_2004}, the populations  of the universe of U.S.\ municipalities and unincorporated Census Designated Places---25,359 locations in 2000---are well described by a log normal distribution; consistent with this, the growth rate of U.S.\ municipalities from 1990 to 2000 was uncorrelated with their initial population.
\end{comment}

\begin{comment}: comparison of slope to Zipfs; implications for convergence

** I decided to cut this. The Eeckout papers

The Blank, Aharon and Solomon, Sorin, Physica, 2000 article shows the Gabaix 1999 is incorrect in the conditions needed for orthogonal growth to generate a Zipf's distribution. Rather than a reflective lower bound (Gabaix), it needs to be that observations can "exit" by declining below some specified lower-bound level of population.  Moreover, Eeckout shows that a lognormal distribution can have a rank-size representation with the slope steeper than -1, even though log-normal is implied by orthogonal growth.

More descriptively, the fitted slope of the quadratic rank-size relationship is well below 1 in absolute value at the smallest observation in each estimation sample (i.e., the smallest observation with population that equals or exceeds the median). The significance is that a slope with absolute value of 1 corresponds to the special case when the Pareto Distribution satisfies Zipf's law. Under certain conditions, a Zipf's distribution asymptotically follows from growth that is orthogonal to size, in the sense that the mean and variance of growth rates are independent of size.

the rank-size relationships steepen as population increases. At the smallest observation of each sample (i.e., at the smallest observation with population greater than or equal to the median), the fitted slope is well below 1 in absolute value, the Zipf's benchmark; at the largest observation of each sample, the fitted slope is well above 1 in absolute value. Compared to Zipf's, the population distribution is always considerably more bunched at the top.

\end{comment}

\begin{comment} deleted material enumerating similarity at top of distribution

The upper portions of KBMA population distributions are relatively similar across a wide range of parameterizations. As a first measure, the bottom rows of Table \ref{table_pop_rank_v_size} compare the fitted slopes at various ranks from quadratic regressions that use only the largest 64 observations. Across the six benchmark parameterizations, the fitted slopes are moderately close at ranks 64 and 32 and very close at ranks 16 and higher.\footnote{The Zipf's specialization of a Pareto distribution has a slope of -1. Compared to this, all of the parameterizations, as well as the metropolitan CBSAs and UAs, have a less bunched distribution at rank 64 and a considerably more bunched distribution across the largest observations.}


Table \ref{table_pop_top200} compares population at a selection of population ranks, from 1 to 200. The population of the largest observation in each of the six sets, which always corresponds to the KBMA that includes the New York core, varies from 18.6 million to 19.4 million, a difference of just 4.2 percent (column 7). The second and third largest observations---which always correspond to the KBMAs that include the Los Angeles and Chicago cores, respectively---similarly span a relatively narrow range.

\begin{table}[tb] %population at benchmark ranks, 1 through 200
\hspace{30mm}
\includegraphics[scale = 0.85, trim = 17mm 140mm 0mm 18mm, clip]{kbma_table_pop_top200_2023_04_30a.pdf}
\vspace{-8mm}
\caption{\label{table_pop_top200} \textbf{Population at Benchmark Ranks}.
{\normalfont \footnotesize \hspace{0.5mm} Column 7 reports the maximum relative difference among the possible pairings from columns 1 to 6, measured as the absolute difference in population divided by mean population. Columns 1 and 2 have identical kernels as do columns 4 to 6. Columns 1 and 4 have identical buildout parameters as do columns 2 and 5. For all delineations, the first, second, and third ranked observations include the New York, Los Angeles and Chicago cores, respectively. Values are rounded to the nearest thousand.}}
\end{table}

In contrast, the next few ranks, from 4th through 7th, span relatively wide ranges of population, with a maximum difference of almost 31 percent. Accounting for the widening is the relaxation of the strength and distance parameters for joining cores, $\sigma$ and $\delta$, which considerably boosts the population of several KBMAs and reshuffles their ranks. In particular, the relaxation joins the Providence, Dover, and Barnstable cores to the baseline Boston--Worcester kernel, boosting the KBMA that includes Boston from rank 6 in columns 1 to 3 to 4th in column 4 and 5th in columns 5 and 6. Similarly, the relaxation joins the baseline Baltimore kernel to the baseline Washington D.C. kernel, boosting the KBMA that includes Washington from rank 10 in columns 1 to 3 to 5th in column 4 and 4th in columns 5 and 6.

Below these few ranks, however, the relative range of population across the benchmark parameterizations stays moderate, averaging 13 percent over ranks 8 through 25, and 18 percent over ranks 26 through 75. But then, continuing further down, the relative population range widens, increasingly exceeding 30 percent as rank drops below 100 and increasingly exceeding 50 percent as rank drops below 150. At these lower ranks, the wide dispersion of relative population holds both across the parameterization of the kernel joins (column 1 versus 4 and 2 versus 5) and across the parameterization of the buildout (column 1 versus 2 and 3 versus 4 versus 5).\footnote{An additional pattern in Table \ref{table_pop_top200} is that the population at most of the highest 75 ranks is boosted by relaxing the kernel join (e.g., switching from column 1 to column 4) but, less intuitively, diminished by relaxing the kernel join at ranks below this. The diminishment partly reflects that a large share of intermediate-population baseline kernels are joined to the largest ones and partly that there are less KBMAs under the relaxed joins, implying that a given rank is closer to the bottom of the distribution.}

Table \ref{table_pop_top020} zooms in on the 20 largest KBMAs under the baseline parameterization. For most of these, the range of population across the six benchmark parameterizations is small or moderate. At one extreme, the KBMA that includes Miami as its largest core and the one that includes San Diego as its largest core each range by less than 2 percent across the parameterizations. At the other extreme, the KBMA delineation that includes Boston as its largest core ranges by 34 percent across parameterizations and the one that includes Washington D.C. ranges 53 percent. Moreover, the Baltimore core no longer anchors its own KBMA under the specifications that relax kernel joins.


\begin{table}[tb] %Population of largest baseline KBMAs
\hspace{30mm}
\includegraphics[scale = 0.87, trim = 17mm 134mm 0mm 18mm, clip]{kbma_table_pop_top020_2023_05_24a.pdf}
\vspace{-8mm}
\caption{\label{table_pop_top020} \textbf{Population of Largest Baseline KBMAs}.
{\normalfont \footnotesize \hspace{0.5mm} Table compares the population rank and population across the benchmark specifications of the KBMA whose largest core includes the enumerated municipality.  Column 7 reports the maximum relative difference among the possible pairings from columns 1 to 6, measured as the absolute difference in population divided by mean population. Columns 1 and 2 have identical kernels as do columns 4 to 6. Columns 1 and 4 have identical buildout parameters as do columns 2 and 5. Baltimore is included in same KBMA as Washington, D.C. under the parameterizations with relaxed joins between cores (columns 4 and 5). Values are rounded to the nearest thousand.}}
\end{table}

\end{comment}

\begin{comment} Alternative table showing fitted rank-size slopes
\begin{table}[tb] %\label{table_fitted_slopes_obs64} % Fitted Rank-Size slopes for top 64 obs
\hspace{30mm}
\includegraphics[scale = 0.88, trim = 17mm 163mm 0mm 18mm, clip]{kbma_table_pop_rank_v_size_obs064_2023_03_28a.pdf}
\vspace{-8mm}
\caption{\label{table_fitted_slopes_obs64} \textbf{Fitted Rank-Size Slopes for the Largest 64 Observations}.
{\normalfont \footnotesize \hspace{0.5mm} Table reports fitted slopes from regressing log(rank - 0.5) on linear and quadratic log(population) for the largest 64 observations in each delineation.\textbf{no need to show population levels, including for smallest pop.}}\vspace{-4mm}}
\end{table}

Across the KBMA delineations reported in Table \ref{table_fitted_slopes_obs64}, only the largest observations are bunched meaningfully tighter than the Zipf's benchmark.


The underlying rank-size coefficients estimated using the top-ranked 64 observations analytically imply that fitted slopes exactly equal -1 at similar ranks (26.2 to 33.2) and at similar levels of population in 2000 (1.5 million to 1.8 million). But convergence that is economically significant should more tightly bunch observations. For example, the fitted slopes steepen below -1.2 at about the 16th-largest observation. Visual inspection of the three scatters in Figure \ref{fig_pop_rank_v_size} shows a discrete increase in bunching moving from the 13th largest KBMA, Phoenix, to the 12th largest one, Atlanta. And at the very top of the population distribution, the New York and Los Angeles KBMAs are closely bunched.\footnote{The specific ranking of the largest 13 KBMAs is nearly identical across the three delineations in Figure \ref{fig_pop_rank_v_size}, the only exception being that Miami is ranked ahead of Dallas under the baseline and small-core delineations but below it under the low-density delineation. Across the three delineations, the population of Phoenix ranges from 3.0 to 3.1 million; the population of Atlanta, from 4.0 to 4.4 million; the population of Los Angeles, from 15.2 to 15.4 million; and the population of New York from 18.6 to 19.2 million. Under all three delineations, the Riverside-San Bernadino core is included in the Los Angeles KBMA.}


\end{comment}

\begin{comment} ***** RESULTS OF QUAD RANK-SIZE POP REGRESSIONS ****

** critical values for quad coefficient in kbma.alt_parameterizations.2022_03_24d.xlsx **

Baseline (292 obs total; top 146 obs):
** reject that log-rank for obs >= median is linear at 0.05 level based on two tail test
-->kbma_regressions_baseline_2022_03_17b.log: t-stat on quad, 146 obs = -21.85   reject at 0.05 based on two-tail test
--> mc_log_rank_size_shape1000sim100k_obs146_2022_01_26a.log:  CDF = 0.025 @ critical t-val = -18.859 (0.010 @ -22.552)

Core thresh = 25k (445 obs total; top 223 obs)
Core thresh = 10k (721 obs total, top 361 obs)
Core thresh = 2.5k (1,250 obs total; top 625 obs)


452 UAs:
--> regression in kbma_for_display.yyyy_mm_dd.xlsx, pop_distributions tab: t-stat on quad -37.2922
--> mc_log_rank_size_shape1000sim100k_obs452_2021_05_16a.log: CDF=0.025@ crit t = -31.19 (0.005 @ -41.998)

3610 UAUCs:
--> regression in kbma_for_display.yyyy_mm_dd.xlsx, pop_distributions tab: t-stat on quad -80.30
* 0.025 crit t = -79.619

181 metro CBSAs (>= median)
--> regression in kbma_for_display.yyyy_mm_dd.xlsx, pop_distributions tab: t-stat on quad -24.57
--> 0.025 crit t = -20.743; 0.005 crit t=-28.422

461 all CBSAs (>= median)
--> regression in kbma_for_display.yyyy_mm_dd.xlsx, pop_distributions tab: t-stat on quad -31.6428
--> 0.025 crit t = -31.647  (just a tick away from rejecting at 0.05);
--> 0.05 crit t = -26.353  (ie. reject at 0.10 level)



\end{comment}

\begin{comment} some details on Monte Carlo results

%\footnote{For the 362 metropolitan CBSAs, the estimated t-statistic corresponding to the quadratic term in a rank-size regression equals -44.3; based on 100,000 Monte Carlo simulations that draw 362 observations, the critical value at which the CDF for the t-statistic equals 0.005 is -37.8. For the 452 UAs, the estimated t-statistic equals -37.3; based on Monte Carlo simulations that draw 452 observations, the critical value at which the CDF equals 0.025 is -31.2.}
%UAs: t-stat = -37.29222604; CDF = 0.025 @ -31.2  (mc_log_rank_size_shape1000sim100k_obs452_2021_05_16a.log)
%metro CBSAs = -44.25755264; CDF = 0.005 @ -37.8  (mc_log_rank_size_shape1000sim100k_obs362_2021_05_16a.log)
\end{comment}

\begin{comment} truncated linear regressions
%Consistent with this, running the rank-size regression using only the smallest 50 KBMAs gives an estimated slope of  -0.414 (0.009); running it using only the largest 50 gives an estimated slope of -1.275 (0.035)
%Consistent with this, running the rank-size regression using only the smallest 40 KBMAs gives an estimated slope of  -0.378 (0.009); running it using only the largest 40 gives an estimated slope of -1.348 (0.041)
%Using only the smallest 50 of these [core 10,000] in a rank-size regression gives an estimated slope of -0.269 (0.003).
\end{comment}

\begin{comment} some discussion of Gibrat's law
%A more definitive judgment on distribution shape would by helped by estimating the extent to which the population growth of KBMAs has been independent of their population. Growth that is independent of size implies an asymptotic distribution that is log normal \citep{eeckhout_2004}.  While \cite{eeckhout_2004} documents that the population growth of municipalities from 1990 to 2000  was indeed independent of size, the randomness of municipal borders implies that this independence may not hold for metropolitan areas.
\end{comment}

\begin{comment} longer description of Desmet and Rappaport (2017) and Rappaport (2018).
%Estimating the growth process generating the population distribution should give additional insight. Growth that is independent of size---Gibrat's law---implies that size converges to a log normal distribution \citep{eeckhout_2004}.  Consistent with Gibrat's law, \citet{ioannides_overman_rsue_2003} present non-parametric estimates using panel data that suggest that the population growth of the largest 110 metropolitan areas was approximately independent of size throughout most of the twentieth century. In contrast, \cite{desmet_rappaport_jue_2017} and \cite{rappaport_frbkc_2018} present cross-sectional evidence that growth was negatively correlated with size across the very largest metropolitan areas from 1960 to 1980 and again from 2000 to 2017. If it persisted, this negative correlation would lead to more bunching at the top of the population distribution than is associated with the right tail of a log normal distribution. The same papers document that growth rates across locations with population up to 500,000 have been positively correlated with size since about 1940, also calling into question the fit of a log normal distribution.
\end{comment}

\begin{comment}  STRONGER LANGUAGE ON IMPLAUSIBILITY OF CONSTANT PARETO COEFFICIENT: implies the largest metropolitan area
%From a broader perspective, it seems implausible to us that a rank-size relationship with a Pareto coefficient above -0.90 could apply across all U.S.\ metropolitan areas. For example, with a Pareto coefficient of -0.90 and 292 metro areas with population above 50,000 (the number of baseline KBMAs), the largest metro area would have a population of 27.4 million, 50 percent above that of the New York City metropolitan CBSA in 2000. With 362 metro areas with population above 50,000 (the number of metropolitan CBSAs in 2000), the largest metro area would have a population of 34.8 million.

%More broadly, it would strain plausibility for a rank-size relationship with Pareto coefficient much below 0.95 to apply across the entire population distribution of U.S.\ metropolitan areas. For example, if 300 metro areas had population exactly distributed according to a power law with Pareto coefficient of -0.95 and the smallest had a population of 50,000, the largest would have a population just over 20 million, about the size of the New York City metropolitan area as measured by various delineations. But with a Pareto coefficient of -0.90, the largest metro area would have a population of 28 million; with a Pareto coefficient of -0.85, it would have population of 41 million. Conversely, if the largest metro area had a population of 20 million, a Pareto coefficient of -0.90 would imply the 300th largest metro area had a population of 35,000; a Pareto coefficient of -0.85 would imply that the 300th largest metro area had a population of 24,000. In contrast, the 300th largest UA and metropolitan CBSA have respective population of 84,000 and 113,000. And as described earlier, 452 UAs have population above 50,000 as do 362 metropolitan CBSAs
\end{comment}

\begin{comment}  older text explaining zipf/pareto <---> orthognal growth

Our robust rejections that metropolitan proxies have a population distribution that is either lognormal or Pareto implies that the proxies have historically grown at rates that depended on their size. Stochastic population growth that has mean and variance independent of the level of population (i.e., that is characterized by Gibrat's law) implies an asymptotic population distribution that is lognormal \citep{eeckhout_2004}.  Additionally allowing for the entry of new observations implies an asymptotic distribution that is Zipf's, a special case of Pareto, above an appropriately chosen lower bound \citep{gabaix_qje_1999, blank_solomon_2000}.

Consistent with the rejection of lognormal and Pareto, \cite{desmet_rappaport_jue_2017} and \cite{rappaport_2018} document that population growth was positively correlated with population across small and intermediate-sized metropolitan proxies throughout the period 1880-2017 and negatively correlated with population across the largest metropolitan proxies throughout the period 1940 to 2017. This divergence across small and intermediate proxies contributes to a flat slope in rank-size space together with an accumulation of locations with population below the metropolitan threshold. This convergence across large proxies implies a slope in rank-size space that steepens with size.\footnote{Similar to us, \cite{holmes_lee_2010} report an inverted-U relationship between population growth and size for their 6-by-6 mile squares. In addition, they report a large mass of squares with population equal to 0. In contrast, \cite{michaels_et_al_2012} find that the population growth of county subdivisions from 1880 to 2000 was negatively correlated with initial population at low levels of population and positively correlated with population at high levels.} % \mbox{F}(S)\, =\, 1 - (\widetilde {S}/S)^{\zeta} \mbox{ for size } S \ge \widetilde{S}, \, \zeta > 0

\end{comment}

\begin{comment} Additional considerations on converge/diverge in Desmet Rappaport

%On the other hand, \cite{desmet_rappaport_jue_2017} and \cite{rappaport_2018} document that population growth was positively correlated with size across small- and intermediate-sized metropolitan proxies from 1880 to 2017---but note that it is driven by expected exceptionally high growth of a handful of small observations, so wouldn't necessarily imply convergence. Another factor is that overall population growth may contribute to the "entry" of small observations above a relatively low implicit threshold---since we're not keeping the threshold constant relative to national population---even locations growing well below national level may eventually surpass,e.g., 2,500.

%In contrast, \cite{desmet_rappaport_2017} and \cite{rappaport_frbkc_2018} document that the population growth of metropolitan areas has considerably deviated from such independence throughout U.S.\ history. For example, growth across smaller metropolitan areas was positively correlated with population throughout the period from 1940 to 2017 and growth across the largest metropolitan areas was negatively correlated with population from 1960 to 1980 and again from 2000 to 2017.}

\end{comment}

\begin{comment} text for describing bunching as consistent with several structural models
%The rejections of a linear rank-size relationship are also consistent with several structural models of metropolitan size.  For example, \cite{rossi-hansberg_wright_2007}  and \cite{duranton_2007} derive DSGE models of urban structure in which the steady-state population distribution is characterized by a concave rank-size relationship. \cite{rappaport_2016} finds that the population distribution of a system of metropolitan areas becomes more bunched at the top than the underlying productivity distribution, owing to worsening traffic congestion.
%the proportionate increase in exogenous productivity required to support a given proportionate increase in metropolitan population becomes larger as metropolitan population increases, owing to steeply increasing commuting congestion. In this sense, the ``price'' of growth becomes increasingly expensive with size, contributing to a bunching at the top of the population distribution
%: whereas a doubling of population from 1 million to 2 million might require a 5 percent increase in TFP, a subsequent doubling of population from 2 million to 4 million might require a 10 percent increase in TFP.

\end{comment}

\begin{comment}  Rossi-Hansberg and Wright (ReStud 2007); Duranaton (AER 2007)

Esteban Rossi-Hansberg and Mark L.J. Wright (2007) ``Urban Structure and Growth'', Review of Economic Studies, 74, 597-624
year = {2007}}

Gilles Duranton (2007). ``Urban Evolutions: The Fast, the Slow, and the Still''
American Economic Review, 97(1),197-221


* Both DSGE models of urban structure that deliver a negative correlation between metropolitan size and growth.

Duranton (2007) mechanism: cities have a single ``first nature'' non-traded goods industry (anchor against zero population) and 1+ ``second nature'' traded goods industry. Quality ladder with research sometimes yielding a ``cross-industry' innovation. Production must be where innovation is done. So you get industry churning across cities, with zero adjustment cost on industry and labor mobility (the ``fast''). Because cities have multiple industries, population rankings will change only gradually. The distribution of population across cities remains constant (the ``still''). --->
--> Cities with more industries are larger (by construction). They have more industries to lose relative to those to gain (in limit, a city with all industries can only lose them with nothing to win). Induces the negative correlation of size and the number of cities of any size (so frequency is decreasing with linear size). Intuition on inverse correlation of population growth rate and population less clear to me.

Rossi-Hansberg and Wright (2007): CITIES SPECIALIZE IN A SINGLE INDUSTRY because all agglomeration is city specific and commuting costs depend on total city population. Physical capital evolves slowly: essentially based on investment less depreciation; *** a reinforcing mechanism of inverse correlation is that growth depends on multiplicative increase in K; but opportunity cost to households (who rent out inudstry-specific physical capital) is linear (i.e., they can invest a fix amount of consumption good to get 1 unit of physical capital, which they can permanently assign to any specific industry); *****the key mechanism is that industries (and hence cities, which all specialize in a single industry) where physical capital is lower due to sequence of negative TFP shocks have a higher marginal product of capital and so ex ante a greater return to physical capital. The additive opportunity cost of capital investment compared to the multiplicative growth induced by capital investment reinforces the faster growth RATE of smaller cities.

A series of negative productivity shocks in an industry leads to a contraction of durable physical capital and out-migration of perfectly mobile labor from cities specialized in it, inducing a negative correlation of the marginal product of capital and city size.


\end{comment}

% *** ^^^ HISTORIAGRAPHY OF MY THINKINING ON POP DISTRIBUTIONS ^^^ ***

% !! DISTRIBUTION OF LAND AREA !!


\begin{figure}[tb]% Distribution of Metropolitan Land Area
\caption{\label{land_distributions} \textbf{Distribution of Metropolitan Land Area}}
\includegraphics[scale = 0.97, trim = 17mm 185mm 8mm 20.5mm, clip]{kbma_figure_pdf_land_2025_03_20a.pdf}

\vspace{-8mm}
\begin{flushleft}
\footnotesize{Notes: The horizontal axes measure the logarithm of land area of the lower bound of bins with width of 0.4 log points. For each histogram, the lower bound of the leftmost bin corresponds to the observation with minimum land area.}
\end{flushleft}
\end{figure}

Figure \ref{land_distributions} compares histograms of land area. Unsurprisingly, the KBMA land distribution is located to the right of the UA distribution and considerably to the left of the metropolitan CBSA distribution. More surprising, the KBMA distribution represents a near parallel rightward shift from the UA distribution, including preserving the latter's pronounced rightward skew. The land distribution of kernel-based urban areas, shown in the right panel, is also a parallel shift rightward from UAs but less far than KBMAs. In aggregate, the 346 kernel-based urban areas have 39 percent less land area than the 361 KBMAs.


The land distribution of kernel-based metropolitan regions compresses that of metropolitan CBSAs, including truncating the latter's right tail. For example, the Dallas--Forth Worth--Arlington kernel-based metropolitan region has the largest land area, 20,500 square miles, comfortably below the land areas of the Riverside-San Bernadino and Anchorage metropolitan CBSAs. The least expansive metropolitan region occupies 334 square miles, more than twice that of the least expansive metropolitan CBSA. In aggregate, the 435 kernel-based metropolitan regions have land area more than 10 times that of the 361 KBMAs.

\begin{comment} similarity of land area in the upper bins when holding eta constant

As with population, the size of the largest KBMAs by land area is relatively insensitive to the parameterization of the minimum core size, $\lambda$. For example, the distributions of land area across observations in the upper portions of the baseline and minimal-core delineations, as measured by the count rather than the share of observations in each of the upper bins, are nearly identical.

%the \emph{count} of observations in the upper bins of the minimal-core parameterization---rather than the displayed share of observations---is nearly identical to the count for the baseline parameterization: as with population, the size of the largest KBMAs by land area is relatively insensitive to the minimum core size, $\lambda$.
%the insensitivity of the delineation of large KBMAs to the threshold population for cores, $\lambda$.
%the upper portion of the minimal-core land distribution measured by the counts is nearly the same as that for the baseline parameterization, confirming the insensitivity of the delineation of large KBMAs to the threshold population, $\lambda$.
% [ADDITIONAL DETAIL ON MINIMAL-CORE VS BASELINE; BUT NEED TO CONFIRM WITH TABLES ANALAGOUS TO THOSE FOR POPULATION \emph{count} of observations in the upper bins of the minimal-core parameterization---as opposed to the frequency as a share of all observations in these bins---is nearly identical to the count for the baseline parameterization. As is the case for population, the land area at benchmark ranks of the minimal-core parameterization as well as the land area of the KBMAs that include a specific largest core are very close to the land area of the corresponding baseline KBMA.

\end{comment}


Figure \ref{figure_land_v_pop} plots the logarithm of land area against the logarithm of population for the three kernel-based parameterizations. The fitted linear relationship for each has a slope less than 1, indicating that higher levels of population are associated with less-than-proportionally higher levels of land area. The slope corresponds to the implicit elasticity of land area with respect to population, estimated to be 0.80 for KBMAs.\footnote{For comparison, \cite{ahlfeldt_pietrostefani_2019} estimate that the implicit elasticity of land area to population is 0.57 across 70 U.S.\ Functional Urban Areas and to range from 0.29 to 0.85 across Functional Urban Areas in each of the additional 13 countries for which they have sufficient data. \cite{combes_duranton_gobillon_2019} estimate that the implicit elasticity of land area to population is approximately 0.7 across urban areas in France.}

\begin{figure}[tb]%Land Area vs Population (KBMAs, k-b metro regions, k-b urban areas)
\caption{\label{figure_land_v_pop} \textbf{Land Area versus Population}}
\includegraphics[scale = 0.93, trim = 18mm 188mm 10mm 20mm, clip]{kbma_figure_land_v_pop_2025_03_08a.pdf}

\vspace{-3.5mm}
\begin{flushleft}
\footnotesize{Note: Standard errors for the slope coefficient are reported in parentheses.}
\end{flushleft}
\end{figure}
% "Tract land area is measured in square miles."
% {\normalfont \footnotesize \hspace{0.5mm} Solid lines show the fitted linear relationship. The slope coefficients, which estimate the implicit elasticity of land area with respect to population, all statistically differ from 1 at the 0.01 level.}


The estimated elasticity is lower and the fit less tight for kernel-based metropolitan regions, reflecting that they encompass a higher share of lightly settled land and so can more easily densify to accommodate population growth. Conversely, the estimated elasticity is higher and the fit tighter for kernel-based urban areas, reflecting their lower capacity to densify. As illustrated in the appendix, the estimated implicit elasticities for metropolitan CBSAs and UAs are similar to those of kernel-based metropolitan regions and kernel-based urban areas, respectively.


The tight, less-than-proportional relationship between land area and population intuitively suggests that centripetal forces are pulling residents and establishments to locate near a central location, for example to take advantage of clustered employment \citep{alonso_1964,mills_1967,muth_1969} or amenities. The pull of these centripetal forces may strengthen with distance as the marginal opportunity cost of locating a bit farther away bends upward \href{https://www.kansascityfed.org/research/research-working-papers/productivity-congested-commuting-metro-size-2016/}{\citep{rappaport_2016}}. Consistent with this, as illustrated in the appendix, the relationship of land to population for the union of UAs and UCs is concave, statistically significant at the 0.01 level. The fitted elasticity declines from 1.01 at the smallest observation to 0.71 at the largest. This concavity probably also reflects topography, which constrains expansion in many large metropolitan areas \citep{saiz_qje_2010}.



\begin{comment} qualitative contribution of truncation bias
%\footnote{I think this means that their is a clockwise pivot as the truncation bias diminishes: at the lower end of the population range, land expansion mostly reflects geographical expansion of the kernel. But at higher pops, the land expansion is more equally distributed between the buildout and the kernel...The truncation bias causes observations with low population to have a disproportionately high ratio of population in their kernel relative to in their buildout, where population density is considerably lower. The truncation bias diminishes as the population of explicit KBMAs increases from low levels, thereby inducing the buildout-to-kernel ratio of  population to increase more steeply than would be the case if latent observations were also included.}
\end{comment}

\begin{comment}
\begin{table}[tb] % Land Area vs Population
\includegraphics[scale = 0.89, trim = 17mm 132mm 0mm 18mm, clip]{kbma_table_land_v_pop_2023_06_05a.pdf}
%\end{center}
\vspace{-8mm}
\caption{\label{table_land_v_pop} \textbf{Land Area versus Population}. {\normalfont \footnotesize \hspace{0.3mm} Table reports results from regressing log(land area) on log(population). The enumerated smallest populations are the minimum among observations included in each regression. \dag \dag: Quadratic regressions are normalized so that the linear coefficient estimates the implicit elasticity of land area with respect to population at the population of the smallest included observation. Null hypotheses are that the linear coefficient equals 1 and that the quadratic coefficient equals 0.\hspace{0.5mm} **: reject null at 0.05 level; ***: at 0.01 level. \dag: The bottom rows report the fitted implicit elasticity of land area with respect to population at benchmark populations from quadratic regressions that use only the 64 largest observations and so have underlying coefficients that differ from those reported in the table. The baseline parameterization sets $\lambda=$50,000; $\sigma ,\sigma^{\prime},\sigma^{\prime \prime}=0.25$; $\delta ,\delta^{\prime},\delta^{\prime \prime}=20$ miles; and $\eta=200$.}\vspace{-4mm }}
\end{table}
\end{comment}


\begin{comment} Discussion of possible quadratic relationship and how to interpret
The quadratic regressions reported in Table \ref{table_land_v_pop} and Appendix F suggest that the relationship between land area and population is concave, with the elasticity of land to population declining further below 1 as metropolitan areas become larger. To be sure, the estimated quadratic coefficient statistically differs from 0 for only a minority of the parameterizations we have constructed. But fitted quadratic relationships for most parameterizations are characterized by a considerable decline in the implicit land elasticity. For example, the fitted elasticity for the baseline parameterization estimated using the largest 64 observations declines from 0.86 at a population of 1 million to 0.49 at the population of its largest KBMA.\footnote{Using the same number of the largest observations keeps the population range similar across parameterizations. Using only the 64 largest prioritizes fitting the upper portion of the joint distribution. All parameterizations that we have constructed for which the fitted land elasticity across the 64 largest observations does not meaningfully decline have $\eta$ set to 0.} The decline is similar for the minimal-core benchmark, a bit less for the relaxed kernel and tight-joins/high-density benchmarks, and shifted down to a lower range of elasticity for the benchmarks that set $\eta$ to 0.

A declining land elasticity suggests that centripetal forces become more acute as size increases. One possible mechanism is traffic congestion. For example, \cite{rappaport_2016} models metropolitan land use in the context of monocentric employment, exogenous variations in total factor productivity, and endogenous traffic congestion. Land area expands close to proportionately with population across small metropolitan areas but considerably less than proportionately across larger ones. The land elasticity even turns negative as population continues to increase, reflecting traffic sufficiently punishing to rule out commuting from distant suburbs.
\end{comment}


\begin{comment} the elasticity may be near one for smallest obs
%The quadratic regressions also suggest that the elasticity of land to population may be close to 1 when metropolitan areas are small. For example, the fitted elasticity for the minimal-core parameterization at the smallest population included in the regression, a bit above 18,000, is 0.946.

%The quadratic regressions reported in Table \ref{table_land_v_pop} and a corresponding appendix table are normalized so that the linear coefficient estimates the elasticity of land to population at the observation with largest population. The estimated linear coefficient lies considerably below 1 for all KBMA parameterizations we have run, for UAUCs and their UA subset, and for CBSAs and

\end{comment}

\begin{comment} dependence of the tightness of land/pop relationship

The tightness of the linear relationship depends primarily on the density threshold and maximum distance for joining singletons to kernels. Unsurprisingly, including sparsely settled tracts in KBMAs weakens the relationship. With the remaining parameters set at their baseline, the R-squared value declines from 0.85 at $\eta$ = 200 (its baseline) to 0.78 at $\eta$ = 100 and then further to 0.66 at $\eta$ = 50 and 0.64 at $\eta$ = 0 (reported in Appendix E).  More broadly, setting $\eta$ to its baseline or higher suffices to keep the R-squared value above 0.80 for all parameterizations that we have constructed. Separately, setting $\delta^{\prime \prime}$ to 10 miles or less also suffices to keep the R-squared value above 0.80 for all parameterizations we have constructed.

\end{comment}

\begin{comment} land v pop for metropolitan CBSAs and UAx

%The land area and population of UAs are even more tightly correlated, with an R-squared value of 0.89 and an estimated elasticity of land to population that declines from near 1 at the smallest population to 0.67 at the largest population. In contrast, the land area and population of metropolitan CBSAs are more weakly correlated (R-squared = 0.30). We interpret this contrast as affirming our judgement that metropolitan CBSAs poorly match our metropolitan conception.
%(We use only the top half of observations by population due to the truncation bias.)


%It is also holds for metropolitan CBSAs, but the relatively loose fit for these suggests that land may be a less important constraint on CBSA population growth.

%Although the estimated range of elasticity is even lower for metropolitan CBSAs, the relatively loose fit (R-squared = 0.30) suggests that the abundant land area of most CBSAs is less of a constraint on population growth.

\end{comment}

\begin{comment} Dependence of nonlinearity of land v pop to parameterization

%In contrast, the nonlinearity of the land-population relationship depends on the parameterization. As documented in Table \ref{table_land_v_pop}, the estimated quadratic coefficient under the baseline parameterization does not statistically differ from 0. We attribute this failure to reject the null under the baseline to insufficient power: lowering the core population threshold, $\lambda$, to 25,000 boosts the number of observations by approximately half, allowing rejection at the 0.05 level. But for other parameterizations, such as lowering the threshold strength for joining cores, $\sigma$, or eliminating the threshold density for singletons ($\eta$=0), the estimated quadratic coefficient is too small to imply meaningful variation in the elasticity of land to population.

\end{comment}

\begin{comment} Sensitivity of curvature to truncation by population

For most parameterizations using obs with pop >= median, H0 of no curvature can't be rejected. This is true even over identical population ranges:
 --> min core pop =10,000 has very little curvature across obs with pop >= median (~50,000) BUT: min core pop = 50,000 has considerable curvature:
Hypothesized explanation: for lambda = 50,000, obs with pop modestly above 50k will have almost all of their population in their kernel, and little in the buildout; for these, it's easy to expand land area.  BUT: for lambda = 10,000, some obs with pop modestly above 50k may have a much higher share of their population in the buildout; for these, increasing further expanding land area may be difficult

Results suggest that it is easy for land to increase proportional to population when most population in a single core (probably by increasing the land area of the buildout, although it's possible that it is easier for the core itself to expand in land area). But, in essence, controlling for the share of population in the core, the curvature of the relationship between land area and population is less pronounced--it's below one everywhere.


\end{comment}

\begin{comment} %is land curvature consistent with log normal?
%This degree of curvature may be consistent with the KBMAs having land area that is distributed log normal but only if they do not represent significantly truncated subsample. For example, suppose a sample of 900 observations have land area drawn from a log normal distribution and the baseline KBMAS represent the 292 observations with the largest land area. In this case, the probability of observing a flat portion of the rank-size relationship would be low.
\end{comment}

\begin{comment}  Lots of alternative text discussing concavity of land v population

As documented in an appendix, regressing log land area on log population estimates similar curvature and R-squared values across a wide range of alternative parameterizations. The most salient exception we have found is the expansive parameterization that drops the density requirement for singletons: it is characterized by a linear relationship between land area and population with an implicit elasticity of 0.460 (0.024) and an R-squared value of 0.554. We interpret this exception as reflecting measurement error, in the sense that we judge that most of the expansive KBMAs' land area fails to meet the built-up component of our metropolitan conception.\footnote{We similarly interpret the relatively weak relationship between metropolitan CBSA land area and population as reflecting measurement error. Consistent with this, a linear regression of land on population for metropolitan CBSAs gives a coefficient of 0.451 (0.036), almost the same as the linear coefficient estimated using the expansive KBMAs.}


%!! invert this paragraph: topic sentence
More substantively, the elasticity of land area with respect to population for large KBMAs is inversely correlated with the threshold strength for joining cores. For example, the fitted elasticity at the 10th largest KBMA (by population) is 0.557 (0.061) for $\sigma = 0.50$; 0.752 (0.055) for $\sigma = 0.05$; and 0.796 (0.051) for $\sigma = 0.02$.  This suggests that variations in land area are more responsive to variations in metropolitan population arising from differences in the number of cores than they are to variations in population arising from differences in core size and buildouts.


As with the rank-size relationships, lowering the population threshold to serve as a metropolitan core reveals the joint distribution of land area and population down to a lower truncation point. Doing so suggests that settled land area increases more than proportionately with population across metropolitan-like locations that arguably fall short of having moderate scale. With a core threshold of 10,000, the implicit elasticity of land area with respect to population is 1.138 (0.031) at the 50th smallest KBMA. Based on a one tail test, the implicit elasticity continues to exceed 1 at the 0.01 significance level for KBMAs with population up to 32,000.
% kbma_analyze_land_v_pop_small_cores_2021_05_05a.log: estimated elasticity at 260th smallest KBMA (core>=10k) is 1.052453	with standard error 0.0213516. The associated t-stat, 2.4566,  with 290 degrees of freedom has CDF 0.993. So would see less than 1 percent of time if elasticity <= 1.

The curved implicit response of KBMA land area to population is consistent with results reported by \citet{rozenfeld_etal_2011}. Using a linear regression, they estimate the elasticity of land area with respect to population to be slightly above 1 for clusters with population of at least 12,000.\footnote{\citet{rozenfeld_etal_2011} regress log population on log area, yielding a coefficient of 0.958 (0.020) with R-squared value of 0.686. We prefer to think of the implicit elasticity as capturing the response of settlement to metropolitan population growth, for which their estimate corresponds to a value of 1.044 (the reciprocal of their coefficient).} However, a scatter plot of their data visibly exhibits either negative curvature or else a structural break at a population of about 60,000. In particular, the elasticity of land with respect to population appears to be well below 1 for clusters with population above 160,000.

More generally, the increasingly less-than-proportionate relationship of settled land area with respect to population as KBMAs' population becomes  larger suggests the  existence of centripetal forces that temper the geographic expansion of metropolitan areas. Commuting to centrally located employment may be one such source. \cite{rappaport_2016}, modeling a system of monocentric cities, reports a baseline elasticity of land area with respect to population that decrease from 0.89 to 0.36 as population increases from 100,000 to 3 million. Another possible source are centrally located amenities, such as pedestrian retail districts, nightlife, and cultural institutions. In addition, geographic constraints---such as oceans, lakes, wetlands, and mountains---may disproportionately constrain the expansion of large KBMAs compared to small ones \citep{saiz_qje_2010}.

\end{comment}

\begin{comment} alt sources of centripetal forces

%Alternatively, the land surrounding many metropolitan areas varies in quality, for example due to  natural features such as mountains and wetlands that require more costly construction techniques \citep{saiz_qje_2010}. The associated increase in marginal construction costs as metropolitan areas expand incentivizes densification.
%Alternatively, an agricultural hinterland that supplies land-intensive goods to nearby metropolitan areas can generate a force containing expansion \citep{Christaller_1933}. %However, dramatic decreases in transportation costs over the past 150 years would seem to make such an explanation unlikely.

\end{comment}

\begin{comment} rank-size slope and R2 for alternative top xx observations
% 500 density: slope=0.858  R2=0.920
% 250 density: slope=0.851  R2=0.898
% 200 density: slope=0.840  R2=0.882
% 150 density: slope=0.824, R2=0.856
% 100 density: slope=0.788, R2=0.809
% 000 density: slope=0.460, R2=0.552
\end{comment}

\begin{comment} Misc

%\footnote{The correlation of population and employment across FMAs and across UAs is partly by construction in the sense that both impose a minimum density on parcels of land eligible for inclusion. Even so }

%The tight correlations partly reflect construction criteria: UAs combine census blocks with minimum population density of 500 per square mile and most tracts in FMAs have population density of at least 200 per square mile.\footnote{The requirement that at least 35 percent of the land of kernel tracts be located in a UA implies they have a population density of at least 175 per square mile; more than 94 percent of singleton tracts have a population density of at least 200 (with the remainder having employment density of at least 200).} But they also imply some limits on the degree to which the internal distributions of population density across parcels of land within UAs and FMAs can vary. (Saiz, .


%We next describe several correlations among FMA size measures. Land area and population are tightly correlated, although with the former increasing less than proportionally with the latter (Figure \ref{fma_pop_land_wrk}, left panel). Los Angeles and San Francisco are outliers among the largest FMAs for having considerably less land than implied by the fitted relationship. Barnstable Town, MA is an outlier for having considerably more land than implied by the fitted relationship.
%\begin{figure}[tb] %FMA Population, Land Area, and Density
%%\begin{center}
%\includegraphics[scale = 1.0, trim = 20mm 180mm 10mm 16mm, clip]{fma_figure_metro_land_v_pop_2021_03_05a.pdf}
%%\end{center}
%%\vspace{-8mm}
%\caption{\label{fma_pop_land_wrk} \textbf{FMA Population, Land Area, and Density} {\normalfont \footnotesize \hspace{0.5mm} Annotation....}}% "Tract land area is measured in square miles."
%\end{figure}


%So lets show pop v pop, land v land for UAs 1 figure and for CBSAs 2nd figure and for 3rd fig: land v pop and employment v pop.  The latter will help identify either low/high LFP or else "leakage". Riverside San-Bernadino as leakage. (Thank Taeyoung).  Would also like to look at leakage from Hinesville FMA. Low labor force participation in Lady Lake,


% to get age demographics, census 2000 tract data is available in SF3 state files, summary level = 140.

% Table:  percentile distribution of pop and land: min, 1, 5, 10, 20, 40, 50, 60, 80, 90, 95, 99, max

% *** Figure: pairwise comparison, fma pop vs fma land AND fma employment vs fma pop

% Table: Top 20 FMAs by pop, with employment, land, and farthest distance (pop, pop rank; wrk, work rank, land, land rank, distance, distance rank)

The estimated elasticity for KBMAs implies that modestly dense settlement expands less than proportionally to increases in population. We interpret this as suggesting the existence of forces that temper geographic size and land expansion more strictly than they temper population. In a monocentric city framework, commuting to a central business district generates a centripetal force that does so.\footnote{Under a baseline calibration of a monocentric city model, Rappaport finds the elasticity of metropolitan land with respect to metropolitan population is close to 0.80 for population up to about 1 million.  As population increases above this, the elasticity falls considerably. } With polycentric employment, the cost of transporting traded goods to interior business and residential locations may serve to do so. Alternatively, an agricultural hinterland that supplies land-intensive goods to nearby metropolitan areas can generate a force containing expansion \citep{Christaller_1933}. %However, dramatic decreases in transportation costs over the past 150 years would seem to make such an explanation unlikely.



\end{comment}

\section{Conclusion}

Parameterizations of our kernel-based algorithm better match a broad definition of metropolitan areas compared to metropolitan CBSAs and other existing delineations. The KBMA parameterization, which balances encompassing commuting flows and excluding sparsely settled land, is likely to be appropriate for most purposes, including for planning, regulation, and research. The more expansive parameterization for kernel-based metropolitan regions and the more compact parameterization for kernel-based urban areas are likely to be appropriate for other purposes. Datasets for each are available for download. The algorithm's computer code will also be made available for download, allowing for extensive customization, including independently setting the commuting strength parameter and the separating distance parameter for the kernel and buildout stages.

A priority for future research is constructing kernel-based delineations for more recent years. Doing so will realize another advantage of KBMAs compared to CBSAs, the ability to decompose metropolitan growth into intensive and extensive margins (i.e., within and outside starting footprints). CBSAs poorly do so because settlement varies hugely within its county building blocks. Similarly, kernel-based delineations can closely track the geographic expansion of nearby metropolitan areas up against each other, in some cases merging into a single metropolitan area but not in others.

On the other hand, using census tracts as building blocks limits data availability. To be sure, the Census Bureau publishes tabulated tract data for some of its products, including decennial censuses and the American Community Survey. Conversely, anonymity requirements limit the availability of microdata for most counties and metropolitan CBSAs, lessening the disadvantage of using tracts as building blocks for some research purposes.

As argued by \cite{duranton_2021}, appropriate metropolitan delineations are required to address fundamental urban questions. KBMAs and other kernel-based parameterizations can contribute to doing so.

%Improved measurement of metropolitan characteristics and outcomes should strengthen and sharpen estimates of correlations among pthese. For example, improved measurement is likely to the strengthen the estimated positive effect of metropolitan size on productivity and wages \citep{ciccone_hall_1996, glaeser_mare_2001}, the estimated negative effect of China's WTO entry on local employment \cite{autor_dorn_hanson_aer_2013}, and the significant correlations between intergenerational income mobility and various measures of segregation, income inequality, schools, social capital, and family characteristics \citep{chetty_hendren_kline_saez_qje_2014}.


\begin{comment} false starts on conclusion section
%they allow metropolitan characteristics and outcomes---including size, growth, the composition of residents and firms, and local government policy---to be measured more accurately. In turn, the more accurate measurement should sharpen estimates of correlations among metropolitan outcomes and characteristics. The relationship between growth and size especially complements our paper. Growth that is independent of size---Gibrat's law---implies that size converges to a log normal distribution, which is characterized by a concave rank-size relationship\citep{eeckhout_2004}. But a concave rank-size relationship is also consistent with growth that is negatively correlated with size.


% Closely related to the Estimating the relationship between growth and size especially complements this paper. Growth that is independent of size---Gibrat's law---implies that size converges to a log normal distribution, which is characterized by a concave rank-size relationship\citep{eeckhout_2004}. KBMAs' concave population and land rank-size relationships are thus consistent with size distributions that are log normal. But the concave relationship of the KBMAs' land area with respect to population suggests that centripetal forces are tempering land expansion. The


% \cite{ioannides_overman_rsue_2003}, based on non-parametric estimates using panel data, argue that the population growth of the largest 110 metropolitan areas was approximately independent of size throughout most of the twentieth century and \cite{eeckhout_2004} documents the same independence for the growth of U.S.\ municipalities from 1990 to 2000.



% On the other hand, a negative correlation between size and growth across large metropolitan areas, as documented by \cite{desmet_rappaport_jue_2017} and \cite{rappaport_frbkc_2018} across large CBSAs for the periods across the very largest metropolitan areas from 1960 to 1980 and again from 2000 to 2017

% \cite{ioannides_overman_rsue_2003}, based on non-parametric estimates using panel data, argue that the population growth of the largest 110 metropolitan areas was approximately independent of size throughout most of the twentieth century and \cite{eeckhout_2004} documents the same independence for the growth of U.S.\ municipalities from 1990 to 2000.


% the correlation between population growth and the level of population.  Consistent with Gibrat's law, \citet{ioannides_overman_rsue_2003} present non-parametric estimates using panel data that suggest that the population growth of the largest 110 metropolitan areas was approximately independent of size throughout most of the twentieth century.



% In contrast, \cite{desmet_rappaport_jue_2017} and \cite{rappaport_frbkc_2018} present cross-sectional evidence that growth was negatively correlated with size across the very largest metropolitan areas from 1960 to 1980 and again from 2000 to 2017. If it persisted, this negative correlation would lead to more bunching at the top of the population distribution than is associated with the right tail of a log normal distribution. The same papers document that growth rates across locations with population up to 500,000 have been positively correlated with size since about 1940, also calling into question the fit of a log normal distribution.


%Estimating the growth process generating the population distribution should give additional insight. Growth that is independent of size---Gibrat's law---implies that size converges to a log normal distribution \citep{eeckhout_2004}.  Consistent with Gibrat's law, \citet{ioannides_overman_rsue_2003} present non-parametric estimates using panel data that suggest that the population growth of the largest 110 metropolitan areas was approximately independent of size throughout most of the twentieth century. In contrast, \cite{desmet_rappaport_jue_2017} and \cite{rappaport_frbkc_2018} present cross-sectional evidence that growth was negatively correlated with size across the very largest metropolitan areas from 1960 to 1980 and again from 2000 to 2017. If it persisted, this negative correlation would lead to more bunching at the top of the population distribution than is associated with the right tail of a log normal distribution. The same papers document that growth rates across locations with population up to 500,000 have been positively correlated with size since about 1940, also calling into question the fit of a log normal distribution.


%Most directly, they should considerably reduce measurement error in estimating the relationship between metropolitan growth and size, including allowing for population growth to be decomposed into intensive and extensive margins (i.e., changes within the original delineation and changes attributable to changes in the delineation). Similarly, improved delineations should sharpen estimates of the correlation of other characteristics and outcomes with metropolitan size and with metropolitan growth. Particularly salient is better understanding the benefits and costs of metropolitan size (e.g., higher productivity and c


%For example, \citet{autor_dorn_hanson_aer_2013} use Commuting Zones as a proxy for labor markets to document a number of adverse outcomes correlated with exposure to trade with China; \citet{chetty_hendren_kline_saez_qje_2014} to identify correlates of income mobility.


% Metropolitan areas play a fundamental role in numerous economic processes, arguably more important than that of any other geographic unit smaller than a nation-state. Notwithstanding this importance, economists have only recently begun to show interest in accurately delineating them \citep{duranton_2021}.

% This shortfall is especially surprising for the United States, which extensively collects survey and administrative data on commuting flows. One possible reason is the availability of the three metropolitan proxies delineated by the U.S.\ government: metropolitan CBSAs and their predecessors by the Office of Management and Budget, UAs by the Census Bureau, and CZs by the Economic Research Service. The first and last of these are constructed based on commuting flows and so

% Two of the three most-used proxies, CBSAs and CZs, are indeed delineated based primarily on commuting flows but nevertheless stray far from our conception of metro areas. In particular, CBSAs and CZs are overwhelmingly made up of land that is not ``built-up'' under any common sense interpretation.


%More definitively, we judge metro areas to be more economically relevant than other types of locations. As argued in the introduction, they correspond to distinct markets for labor and many goods and services, to the geographic domain determining many matching and sharing externalities, and the geographic range for sharing many production and consumption amenities. This correspondence  \emph{exactly} matches the implicit definition of the ``cities''  \citet{eeckhout_2004} models to illustrate growth that is independent of size. Of course municipalities also matter, for example to understand how policy differences across the hundreds of municipalities within metropolitan areas shape sorting, land use, and other outcomes \citep{nrc_1999}. Both finer and more aggregated locations matter as well. For example, \cite{holmes_lee_2010} document that population growth from 1990 to 2000 across a partition of the U.S.\ into six-by-six-mile squares has an inverted U-shape with respect to population size. And \cite{holmes_jpe_1998} uses variation across states to establish that right-to-work laws affect decision on where to locate manufacturing establishments. But in contrast to these alternatives, only metropolitan areas arguably suffice to analyze a wide range of fundamental urban questions.


%Better correspondence to agglomerative geography: If so, they are also likely to yield better estimates of agglomerative strength, which can prove sensitive to misspecified delineations \citep{Bosker_etal_2020}. Eeckout suggests this is the appropriate unit to use.

%Using commuting flows among census tracts in 2000, we develop an algorithm that can be judgementally parameterized to approximate our conception. Under a baseline parameterization that balances encompassing commuting flows and limiting the inclusion of rural land, our Kernel-Based Metropolitan Areas capture almost all of the population and employment of metropolitan CBSAs but only a small portion of their land area. Under an alternative parameterization that puts minimal weight on limiting rural land, they capture almost all worker flows that are not between metropolitan areas.

%Metropolitan areas are a fundamental unit of economic analysis and so accurately delineating them is critical to understanding a wide range of phenomena. We conceive of metropolitan areas  conception of metropolitan areas as


%we conceive of a metropolitan area as a group of nearby built-up locations with combined population of at least moderate scale and within which a significant share of workers travel on a day-to-day basis among places of residence, places of employment, and places of consumption; correspondingly, most people who live, work, or regularly participate in other activities in a metropolitan area do not travel outside it on a day-to-day basis. This conception is purposely imprecise, leaving considerable scope for judgment on whether specific delineations approximate it.

%Accurately delineating metropolitan areas


%    \item papers that use it: \citet{Autor_Dorn_Hanson},
% looking at differences in outcomes across geographies.

%Eeckhout suggests that the "right" unit is the one that captures externalities.

\end{comment}


\spacing{1.2}
\bibliographystyle{aea} % *** requires file aea_jordan.bst
\bibliography{kbma_references_2025_04_dd}

\newpage
\appendix
\section*{Appendices}
\renewcommand{\thesubsection}{\Alph{subsection}}
\spacing{0.9}
\renewcommand{\baselinestretch}{0.95}
\small



\begin{enumerate}[A.]

  \item Supplemental Figures

    \vspace{-0.5mm}
    \begin{enumerate}[1.]
      \item Tract-Based Approximations of UAs and UCs
      \item Sensitivity of Kernel Construction to the Strength and Distance Parameters
      \item Sensitivity of Population, Employment, and Commuting Outflows to the Density Parameter
      \item The Population and Land Area of UAs and UCs
      \item Population Rank versus Size for Official Delineations
      \item Land Area versus Population for Official Delineations
      \item Commuting Flows for UAs and Commuting Zones
      \item Decomposition of KBMA Commuting Flows
    \end{enumerate}

  \vspace{0.5mm}
  \item Supplemental Tables

    \vspace{-0.5mm}
    \begin{enumerate}[1.]
      \item Monte Carlo Tests for Pareto Population Distributions
      \item Comparison of Metropolitan CBSAs with their Corresponding KBMA
      \item Metropolitan CBSA Pairs with Strong Cross Commuting
    \end{enumerate}


  \vspace{0.5mm}
  \item Supplemental Maps

    \vspace{-0.5mm}
    \begin{enumerate}[1.]
      \item Commuting Zones Near New York City
      \item KBMAs from New York City to Virginia Beach
      \item KBMAs in Northern California
      \item KBMAs in Florida
      \item KBMAs in Texas
    \end{enumerate}

  %\vspace{0.5mm}
  %\item Enumeration of KBMAs

  \vspace{0.5mm}
  \item Online: The files directly linked below are also available from the paper's \href{https://www.kansascityfed.org/research/research-working-papers/a-better-delineation-of-us-metropolitan-areas/}
      {\underline{webpage}}.

    \vspace{-0.5mm}
    \begin{enumerate}[1.]


\item Enumerations of

      \href{https://www.kansascityfed.org/documents/10743/rwp25-01_online_kbmas_enumeration.pdf}
      {\underline{KBMAs}} (kernel-based metropolitan areas), \href{https://www.kansascityfed.org/documents/10745/rwp25-01_online_kbuas_enumeration.pdf}
      {\underline{KBUAs}}  (kernel-based urban areas),  and \href{https://www.kansascityfed.org/documents/10744/rwp25-01_online_kbmrs_enumeration.pdf}
      {\underline{KBMRs}}  (kernel-based metropolitan regions)

      \item Maps of
      \href{https://www.kansascityfed.org/documents/10778/rwp25-01_online_kbmas_maps.pdf}
      {\underline{KBMAs}},
      \href{https://www.kansascityfed.org/documents/10780/rwp25-01_online_kbuas_maps.pdf}
      {\underline{KBUAs}}, and
      \href{https://www.kansascityfed.org/documents/10779/rwp25-01_online_kbmrs_maps.pdf}
      {\underline{KBMRs}}

      \item
      \href{https://www.kansascityfed.org/documents/10781/rwp25-01_online_maps_comparison.pdf}
      {\underline{Maps}} comparing KBMAs, KBUAs, and KBMRs


      \item Detailed tables (additional variables, constituent census tracts, cores, built-out kernels, and kernel iterations) for
      \href{https://www.kansascityfed.org/documents/10794/rwp25-01_online_kbmas_data.zip}{\underline{KBMAs}},
      \href{https://www.kansascityfed.org/documents/10789/rwp25-01_online_kbuas_data.zip}{\underline{KBUAs}}, and \href{https://www.kansascityfed.org/documents/10792/rwp25-01_online_kbmrs_data.zip}{\underline{KBMRs}}


      \item Shape Files for
      \href{https://www.kansascityfed.org/documents/10795/rwp25-01_online_kbmas_shape-files.zip}{\underline{KBMAs}},
      \href{https://www.kansascityfed.org/documents/10791/rwp25-01_online_kbuas_shape-files.zip}{\underline{KBUAs}}, and \href{https://www.kansascityfed.org/documents/10793/rwp25-01_online_kbmrs_shape-files.zip}{\underline{KBMRs}}



      \item \href{https://admin.kansascityfed.org/documents/10803/rwp25-01_cores_pairwise_flows.zip}
      {\underline{Pairwise flows between cores}} (prior to kernel construction)


      \item \href{https://www.kansascityfed.org/documents/10788/rwp25-01_kernel_iterations.zip}
      {\underline{Iterative kernel joins}} as commuting strength drops to 0.01. (underlying data for Supplemental Figure \ref{kernels_v_sigma})



    \end{enumerate}


\end{enumerate}

\clearpage
%\subsection{Supplemental Figures}
\renewcommand{\thefigure}{\mbox{A.\arabic{figure}}}
\setcounter{figure}{0}
\renewcommand{\thetable}{\mbox{A.\arabic{table}}}
\setcounter{table}{0}
\normalsize


\begin{figure}[htb]% Tract-Based Approximations of UAs and UCs
\caption{\label{append_fig_uaucas_v_uaucs} \textbf{Tract-Based Approximations of UAs and UCs}}%
%\begin{center}
\includegraphics[scale = 0.95, trim = 19mm 184mm 10mm 20mm, clip]{kbma_append_figure_uaucas_v_uaucs_2024_10_07a.pdf}
%\end{center}

\vspace{-3mm}\begin{flushleft}
\footnotesize{Notes: Census tracts are assigned to the approximation of a UA or a UC with population of at least 10,000 if the shares of their population and land in the UA/UC are at least 0.55 and 0.30, respectively. These thresholds minimize the sum of squared percentage differences of approximation population and land area from their actual values. Dashed lines are drawn where approximation size equals actual. Only 1,274 of the 1,374 qualifying UAs/UCs have at least one tract that meets both thresholds; the gray dots represent the 100 UCs with no tract that meets both. These null approximations only negligibly affect the delineated set of KBMAs. As illustrated by the Kansas City KBMA (Figure \ref{kc_kbma_zoomed_in}), many tracts overlapping unapproximated UCs join to a nearby kernel during the buildout. The built-out kernels of most remaining unapproximated UCs would likely fail to meet the 50,000 population threshold to qualify as a KBMA, similar to the 570 actual built-out kernels that  fail to qualify.}
\end{flushleft}
\end{figure}  % UAUCs_v_UAUCAs (scatter)

%the qualifying UCs, which range in population from 10,001 to 26,164,



\begin{figure}[t]% Sensitivity of Kernels to the Strength and Distance Parameters
%\begin{center}
\caption{\label{kernels_v_sigma} \textbf{Sensitivity of Kernel Construction to the Strength and Distance Parameters}}
\includegraphics[scale = 0.91, trim = 19mm 150mm 10mm 21mm, clip]{kbma_append_figure_kernel_sensitivity_2025_03_20a.pdf}
%\end{center}

\vspace{-6mm}
\begin{flushleft}
\footnotesize{Notes: The kernel-iteration stage of the KBMA algorithm starts with 1,274 separate cores (green dashed line), corresponding to tract-based approximations of UAs and UCs with population above 10,000. Moving from right to left, the figure shows the number of surviving kernels as $\sigma$ is decreased from 0.50 to 0.01 for each of three distance maximums (green, red, and blue lines) and for unconstrained distance (yellow line). At the KBMA strength threshold, $\sigma$=0.25, the 20-mile maximum distance binds for 22 latent joins (vertical distance between the red and yellow lines). At the kernel-based urban area strength threshold, $\sigma$=0.40, the 10-mile maximum distance binds for 37 latent joins (vertical distance between the green and yellow lines). At the kernel-based metropolitan region strength threshold, $\sigma$=0.10, the 40-mile maximum distance binds for 8 latent joins (distance between the blue and yellow lines). An \href{https://www.kansascityfed.org/documents/10788/rwp25-01_kernel_iterations.zip}
{\underline{online workbook}} documents each of the iterative joins for each of the distance maximums.}
\end{flushleft}
\end{figure}


\clearpage
\begin{figure}[htb] % Sensitivity of Population, Employment, and Outflows to the Density Parameter
\caption{\label{size_alt_density_complement}
\textbf{Sensitivity of Population, Employment, and Commuting Outflows to the Density Parameter}}
\includegraphics[scale = 0.90, trim = 20mm 182mm 10mm 20mm, clip]{kbma_append_figure_calibrating_eta_2025_03_20a.pdf}

\vspace{-7mm}\begin{flushleft}
\footnotesize{Notes: These complement the  land buildout ratios and commuting inflow rates in Figure \ref{size_alt_density}. Left panel shows the median and 90th-percentile ratios of the population and employment in the buildout portion of built-out kernels relative to the kernel portion as $\eta$ is increased from 0 to 500. The dashed vertical line corresponds to the parameterized KBMA value, $\eta$ = 200. The vertical scale is one-tenth that used for the land buildout ratio in Figure \ref{size_alt_density}, reflecting that population and employment are considerably less sensitive to $\eta$. The right panel shows the median and 90th percentile rates of commuting outflows, which are less sensitive to $\eta$ than the corresponding inflow rates shown in Figure \ref{size_alt_density}. For comparability, kernels are restricted to the 302 with population of at least 50,000.}
\end{flushleft}
\end{figure}
% Sensitivity to the Buildout Density Threshold

\vspace{-2mm}

\begin{figure}[H]% The Population and Land Area of UAs and UCs
%\begin{center}
\caption{\label{pop_land_UAsUCs} \textbf{The Population and Land Area of UAs and UCs}}%
\includegraphics[scale = 0.85, trim = 10mm 185mm 0mm 20.5mm, clip]{kbma_append_figure_uaucs_2025_03_20a}
%\end{center}

\vspace{-7mm}\begin{flushleft}
\footnotesize{Notes: The left panel shows the population distributions of UAs and the union of UAs and UCs. Its horizontal axis measures the log of population at the lower bound of bins with width of 0.4 log points. \hspace{1mm} The right panel plots the land area of UAs and UCs against their population. The solid and dashed lines respectively represent linear and quadratic fits. The negative estimated quadratic coefficient for the quadratic fit statistically differs from 0 at the 0.01 level. The slope of the quadratic fit declines from 1.01 at the population of the smallest UA/UC to 0.71 at the population of the largest.}
\end{flushleft}
\end{figure}


\clearpage
\begin{figure}[H]% Population Rank versus Population Size for Official
\caption{\label{fig_pop_rank_v_size_append}
\textbf{Population Rank versus Size for Official Delineations}}
%\begin{center}
\includegraphics[scale = 0.93, trim = 24mm 186mm 10mm 22mm, clip]{kbma_append_figure_pop_rank_v_size_metrocbsa_cbsa50k_ua_2025_01_04a}
%\end{center}

\vspace{-5mm}
\begin{flushleft}
\footnotesize{Notes: Subtracting 0.5 from the rank improves the fit in the presence of small-sample bias \citep{gabaix_ibragimov}. In the left panel, the sharp concavity of the rank--size scatter at the bottom of the distribution reflects that metropolitan CBSAs are truncated based on the population of their cores rather than their total population. As illustrated in the middle panel, the rank-size scatter is approximately linear at the bottom of the distribution for the set of all CBSAs, metropolitan and micropolitan, cleanly truncated at a total population of 50,000.}
\end{flushleft}

\end{figure}


\begin{figure}[H]%Land Area vs Population (KBMAs, UAs, metropolitan CBSAs)
\caption{\label{fig_append_land_v_pop}
\textbf{Land Area versus Population for Official Delineations}}
\includegraphics[scale = 0.93, trim = 18mm 185mm 10mm 22mm, clip]{kbma_append_figure_land_v_pop_mcbsa_cbsa50k_ua_2025_01_13a.pdf}

\vspace{-6mm}
\begin{flushleft}
\footnotesize{Note: Standard errors for the slope coefficient are reported in parentheses.}
\end{flushleft}

\end{figure}



\clearpage
\begin{figure}[tb]%Commuting Flows for UAs and Commuting Zones
\caption{\label{commuting_flows_uas_czs} \textbf{Commuting Flows for UAs and Commuting Zones}}

\includegraphics[scale = 0.99, trim = 24mm 116mm 24mm 20mm, clip]{kbma_append_figure_flows_scatters_uas_czs_2025_03_18a.pdf}

\vspace{-8mm}
\begin{flushleft}
\footnotesize{Notes: Commuting inflow rates are measured as a share of employment. Commuting outflow rates are measured as a share of residents who are employed. Delineations and commuting flows are based on the 2000 decennial census. The Hightstown, NJ UA,  has the highest commuting inflow and outflow rates, 0.74 and 0.81, respectively (above the displayed range). One additional UA has an inflow rate above the displayed range and nine additional ones have outflow rates above the displayed range. We use tract-based approximations of UAs to access commuting data, constructed with more relaxed threshold allocation factors than those for the kernel-based delineations. Specifically, we assign a tract to the tract-based approximation of a UA if at least 50 percent of its population is located in the UA, regardless of how much of its land area is in the UA. Doing so effects UA approximations that have population closer to the actual UA value. The associated inclusion of more unsettled land does not directly affect the measured commuting rates. The Commuting Zone delineations are disseminated by the Economic Research Service \citep{ers_2012}.}
\end{flushleft}

\end{figure}



\clearpage
\begin{figure}[tb]%Decomposition of KBMACommuting Flows, by unattached tracts and other KBMAs

\caption{\label{commuting_flows_decomp} \textbf{Decomposition of KBMA Commuting Flows}}
\includegraphics[scale = 0.99, trim = 24mm 116mm 24mm 20mm, clip]{kbma_append_figure_flows_kbmas_decomp_2025_03_07a.pdf}

\vspace{-7mm}
\begin{flushleft}
\footnotesize{Notes: The top panels show inflow rates to KBMAs and outflow rates from them with respective origins and destinations in tracts that are not part of a KBMA. The bottom panels show inflow rates to KBMAs and outflow rates from them with respective origins and destinations in another KBMA. Commuting inflows overwhelmingly originate from unattached tracts. Commuting outflows split about equally to unattached tracts and to other KBMAs.}
\end{flushleft}

\end{figure}




\newpage
%\subsection{Supplemental Tables}
\renewcommand{\thefigure}{\mbox{B.\arabic{figure}}}
\setcounter{figure}{0}
\renewcommand{\thetable}{\mbox{B.\arabic{table}}}
\setcounter{table}{0}


\begin{table}[H] %metropolitan CBSAs vs comparison KBMAs
\caption{\label{table_cbsa_kbma_compare} \textbf{Metropolitan CBSAs Compared to KBMAs}}
%\begin{center}
\includegraphics[scale = 0.89, trim = 18mm 173mm 0mm 18mm, clip]{kbma_append_table_metrocbsa_v_kbma_2025_03_11a.pdf}

\vspace{-7mm}
\begin{flushleft}
\footnotesize{Notes: This table complements Table \ref{table_kbma_cbsa_compare} in the main text. Each metropolitan CBSA is compared to the KBMA that has the most populous core in it. The summary statistics on relative size and overlap exclude 8 metropolitan CBSAs that have no comparison KBMA, reflecting that the UA/UC cores located in them belong to built-out kernels with population below 50,000. The aggregate statistics compare all metropolitan CBSAs with all KBMAs.}
\end{flushleft}

\end{table}


\vspace{5mm}
\begin{table}[H] % Monte Carlo Tests for Pareto Population Distributions
\caption{\label{table_append_rank_size} \textbf{Monte Carlo Tests for Pareto Population Distributions}}
\includegraphics[scale = 0.90, trim = 18mm 214mm 0mm 17mm, clip]{kbma_append_table_rank_size_monte_carlo_2025_03_08a.pdf}

\vspace{-7mm}
\begin{flushleft}
\footnotesize{Notes:   Table reports results from regressing log(rank - 0.5) on linear and quadratic log population using the largest 25, 50, and 75 observations. Following the methodology described in \cite{gabaix_ioannides_2004}, the left three columns report the value of the t-statistic on the quadratic term at which the cumulative distribution from Monte Carlo simulations equals the specified value. The remaining columns report the t-statistic on the quadratic coefficient for each of the delineations. The Monte Carlo simulations draw the specified number of observations from a Pareto distribution 100,000 times; the t-statistic is independent of the shape parameter, reflecting that the shape parameter linearly affects both the coefficient and standard error.  * denotes that we can reject that the observations are drawn from a Pareto distribution at the 0.10 level based on a two-tail test; ** that we can reject at the 0.05 level.}
\end{flushleft}

\end{table}


\newpage
%\thispagestyle{empty}

\begin{table}[H] % Metropolitan CBSA Pairs with Strong Cross Commuting
\vspace{0mm}  %
\caption{\label{table_append_metrocbsa_flows} \textbf{Metropolitan CBSA Pairs with Strong Cross Commuting}}
\vspace{-2mm}
\includegraphics[scale = 0.90, trim = 18mm 29mm 0mm 19mm, clip]{kbma_append_table_metrocbsa_pairwise_flows_2025_04_08a.pdf}

\vspace{-5mm}
\begin{flushleft}
\footnotesize{Notes: Outflow rates are measured by  workers commuting to the paired CBSA as a share of employed residents. Inflow rates are measured by workers commuting from the paired CBSA as a share of employment. The 64 pairs are all those for which the sum of the four inflow and outflow rates exceeds 0.25, the threshold value used for the KBMA parameterization. Delineations and flows are based on the 2000 decennial census.}
\end{flushleft}

\end{table}






\newpage
%\subsection{Supplemental Maps}
\renewcommand{\thefigure}{\mbox{C.\arabic{figure}}}
\setcounter{figure}{0}
\renewcommand{\thetable}{\mbox{C.\arabic{table}}}
\setcounter{table}{0}



\newpage
\begin{figure}[h] % Commuting Zones in the Vicinity of New York City
\caption{\label{map_czs_nyc} \textbf{Commuting Zones in the Vicinity of New York City}}
\includegraphics[scale = 1.10, trim = 40mm 82mm 0mm 60mm, clip]{append_map_czs_nyc_2025_03_18.png}
\vspace{-8mm}\begin{flushleft}
\footnotesize{Blue lines demarcate the borders of metropolitan CBSAs. The commuting zone delineations are based on commuting patterns in 2000 \citep{ers_2012}.}
\end{flushleft}
\end{figure}



\newpage
\begin{figure}[H] % KBMAs from NYC to Virginia Beach
%\hspace{-22mm}
\caption{\label{map_kbmas_nyc_va}\textbf{KBMAs from NYC to Virginia Beach}}
\includegraphics[scale = 0.875, trim = 15mm 15mm 0mm 25mm,clip]{supp_map_kbmas_nyc_to_va_2025_03_31.png}
\vspace{-7mm}\begin{flushleft}
\footnotesize{Blue lines demarcate the borders of metropolitan CBSAs. Gray lines demarcate tract borders.  Additional maps of KBMAs, kernel-based urban areas, and kernel-based metropolitan regions are available from the paper's \href{https://www.kansascityfed.org/research/research-working-papers/a-better-delineation-of-us-metropolitan-areas/}
{\underline{webpage}}.}
\end{flushleft}
\end{figure}



\newpage
\begin{figure}[H] % KBMAs in Northern California
%\hspace{-22mm}
\caption{\label{map_ncal}\textbf{KBMAs in Northern California}}
\includegraphics[scale = 0.875, trim = 15mm 15mm 0mm 25mm,clip]{supp_map_kbmas_north_california_2025_03_28.png}
\vspace{-7mm}\begin{flushleft}
\footnotesize{Blue lines demarcate the borders of metropolitan CBSAs. Gray lines demarcate tract borders.  Additional maps of KBMAs, kernel-based urban areas, and kernel-based metropolitan regions are available from the paper's \href{https://www.kansascityfed.org/research/research-working-papers/a-better-delineation-of-us-metropolitan-areas/}
{\underline{webpage}}.}
\end{flushleft}
\end{figure}


\newpage
\begin{figure}[H] % KBMAs in Florida
%\hspace{-22mm}
\caption{\label{map_florida}\textbf{KBMAs in Florida}}
\includegraphics[scale = 0.875, trim = 15mm 15mm 0mm 25mm,clip]{append_map_kbmas_florida_2025_04_08.png}
\vspace{-7mm}\begin{flushleft}
\footnotesize{Blue lines demarcate the borders of metropolitan CBSAs. Gray lines demarcate tract borders.  Additional maps of KBMAs, kernel-based urban areas, and kernel-based metropolitan regions are available from the paper's \href{https://www.kansascityfed.org/research/research-working-papers/a-better-delineation-of-us-metropolitan-areas/}
{\underline{webpage}}.}
\end{flushleft}
\end{figure}

\newpage
\begin{figure}[H] % KBMAs in Texas
%\hspace{-22mm}
\caption{\label{map_florida}\textbf{KBMAs in Texas}}
\includegraphics[scale = 0.875, trim = 15mm 15mm 0mm 25mm,clip]{append_map_kbmas_texas_2025_04_08.png}
\vspace{-7mm}\begin{flushleft}
\footnotesize{Blue lines demarcate the borders of metropolitan CBSAs. Gray lines demarcate tract borders.  Additional maps of KBMAs, kernel-based urban areas, and kernel-based metropolitan regions are available from the paper's \href{https://www.kansascityfed.org/research/research-working-papers/a-better-delineation-of-us-metropolitan-areas/}
{\underline{webpage}}.}
\end{flushleft}
\end{figure}



% vvvvvvvvvvvvvvvvvvvvvvvvvvvvvvvvvvvvvvvvvvvvvvvvvvvvvvvvvvvvvvvvvvvvvvvvvvvvvvvvvvvvvvvvvvvvvvvv
% TABLES ENUMERATING KBMAs

\newpage
%\subsection{Online Tables}
\renewcommand{\thefigure}{\mbox{D.\arabic{figure}}}
\setcounter{figure}{0}
\renewcommand{\thetable}{\mbox{D.\arabic{table}}}
\setcounter{table}{0}



\begin{table}[h] % enumeration of KBMAs (ranks 1 to 50)
\vspace{0mm}  %
\caption{\label{kbma_enumeration_p1} \textbf{Enumeration of KBMAs} (ranks 1 to 50 of 361)}
\includegraphics[page=1, scale = 1.01, trim = 18mm 60mm 0mm 19mm, clip]{enumeration_kbmas.2025_03_20b.pdf}
\vspace{-11mm}\begin{flushleft}
\footnotesize{``Max Dist'' is the distance between the tract centroids within a KBMA that are farthest from each other. Tables with more detailed variables for KBMAs and analogous tables for kernel-based metropolitan regions and kernel-based urban areas are available from the paper's \href{https://www.kansascityfed.org/research/research-working-papers/a-better-delineation-of-us-metropolitan-areas/}
{\underline{webpage}}.}
\end{flushleft}
\end{table}


\newpage
\addtocounter{table}{-1}
\begin{table}[h] % enumeration of KBMAs (ranks 51 to 100)
\vspace{0mm}  %
\caption{\label{kbma_enumeration_p1} \textbf{Enumeration of KBMAs} (ranks 51 to 100 of 361)}
\includegraphics[page=2, scale = 1.01, trim = 18mm 60mm 0mm 19mm, clip]{enumeration_kbmas.2025_03_20b.pdf}
\vspace{-11mm}\begin{flushleft}
\footnotesize{``Max Dist'' is the distance between the tract centroids within a KBMA that are farthest from each other. Tables with more detailed variables for KBMAs and analogous tables for kernel-based metropolitan regions and kernel-based urban areas are available from the paper's \href{https://www.kansascityfed.org/research/research-working-papers/a-better-delineation-of-us-metropolitan-areas/}
{\underline{webpage}}.}
\end{flushleft}
\end{table}


\newpage
\addtocounter{table}{-1}
\begin{table}[h] % enumeration of KBMAs (ranks 101 to 150)
\vspace{0mm}  %
\caption{\label{kbma_enumeration_p1} \textbf{Enumeration of KBMAs} (ranks 101 to 150 of 361)}
\includegraphics[page=3, scale = 1.01, trim = 18mm 60mm 0mm 19mm, clip]{enumeration_kbmas.2025_03_20b.pdf}
\vspace{-11mm}\begin{flushleft}
\footnotesize{``Max Dist'' is the distance between the tract centroids within a KBMA that are farthest from each other. Tables with more detailed variables for KBMAs and analogous tables for kernel-based metropolitan regions and kernel-based urban areas are available from the paper's \href{https://www.kansascityfed.org/research/research-working-papers/a-better-delineation-of-us-metropolitan-areas/}
{\underline{webpage}}.}
\end{flushleft}
\end{table}


\newpage
\addtocounter{table}{-1}
\begin{table}[h] % enumeration of KBMAs (ranks 151 to 200)
\vspace{0mm}  %
\caption{\label{kbma_enumeration_p1} \textbf{Enumeration of KBMAs} (ranks 151 to 200 of 361)}
\includegraphics[page=4, scale = 1.01, trim = 18mm 60mm 0mm 19mm, clip]{enumeration_kbmas.2025_03_20b.pdf}
\vspace{-11mm}\begin{flushleft}
\footnotesize{``Max Dist'' is the distance between the tract centroids within a KBMA that are farthest from each other. Tables with more detailed variables for KBMAs and analogous tables for kernel-based metropolitan regions and kernel-based urban areas are available from the paper's \href{https://www.kansascityfed.org/research/research-working-papers/a-better-delineation-of-us-metropolitan-areas/}
{\underline{webpage}}.}
\end{flushleft}
\end{table}


\newpage
\addtocounter{table}{-1}
\begin{table}[h] % enumeration of KBMAs (ranks 201 to 250)
\vspace{0mm}  %
\caption{\label{kbma_enumeration_p1} \textbf{Enumeration of KBMAs} (ranks 201 to 250 of 361)}
\includegraphics[page=5, scale = 1.01, trim = 18mm 60mm 0mm 19mm, clip]{enumeration_kbmas.2025_03_20b.pdf}
\vspace{-11mm}\begin{flushleft}
\footnotesize{``Max Dist'' is the distance between the tract centroids within a KBMA that are farthest from each other. Tables with more detailed variables for KBMAs and analogous tables for kernel-based metropolitan regions and kernel-based urban areas are available from the paper's \href{https://www.kansascityfed.org/research/research-working-papers/a-better-delineation-of-us-metropolitan-areas/}
{\underline{webpage}}.}
\end{flushleft}
\end{table}


\newpage
\addtocounter{table}{-1}
\begin{table}[h] % enumeration of KBMAs (ranks 251 to 300)
\vspace{0mm}  %
\caption{\label{kbma_enumeration_p1} \textbf{Enumeration of KBMAs} (ranks 251 to 300 of 361)}
\includegraphics[page=6, scale = 1.01, trim = 18mm 60mm 0mm 19mm, clip]{enumeration_kbmas.2025_03_20b.pdf}
\vspace{-11mm}\begin{flushleft}
\footnotesize{``Max Dist'' is the distance between the tract centroids within a KBMA that are farthest from each other. Tables with more detailed variables for KBMAs and analogous tables for kernel-based metropolitan regions and kernel-based urban areas are available from the paper's \href{https://www.kansascityfed.org/research/research-working-papers/a-better-delineation-of-us-metropolitan-areas/}
{\underline{webpage}}.}
\end{flushleft}
\end{table}


\newpage
\addtocounter{table}{-1}
\begin{table}[h] % enumeration of KBMAs (ranks 301 to 350)
\vspace{0mm}  %
\caption{\label{kbma_enumeration_p1} \textbf{Enumeration of KBMAs} (ranks 301 to 350 of 361)}
\includegraphics[page=7, scale = 1.01, trim = 18mm 60mm 0mm 19mm, clip]{enumeration_kbmas.2025_03_20b.pdf}
\vspace{-11mm}\begin{flushleft}
\footnotesize{``Max Dist'' is the distance between the tract centroids within a KBMA that are farthest from each other. Tables with more detailed variables for KBMAs and analogous tables for kernel-based metropolitan regions and kernel-based urban areas are available from the paper's \href{https://www.kansascityfed.org/research/research-working-papers/a-better-delineation-of-us-metropolitan-areas/}
{\underline{webpage}}.}
\end{flushleft}
\end{table}


\newpage
\addtocounter{table}{-1}
\begin{table}[h] % enumeration of KBMAs (ranks 351 to 361)
\vspace{0mm}  %
\caption{\label{kbma_enumeration_p1} \textbf{Enumeration of KBMAs} (ranks 351 to 361 of 361)}
\includegraphics[page=8, scale = 1.01, trim = 18mm 202mm 0mm 19mm, clip]{enumeration_kbmas.2025_03_20b.pdf}
\vspace{-11mm}\begin{flushleft}
\footnotesize{``Max Dist'' is the distance between the tract centroids within a KBMA that are farthest from each other. Tables with more detailed variables for KBMAs and analogous tables for kernel-based metropolitan regions and kernel-based urban areas are available from the paper's \href{https://www.kansascityfed.org/research/research-working-papers/a-better-delineation-of-us-metropolitan-areas/}
{\underline{webpage}}.}
\end{flushleft}
\end{table}



\begin{comment}

\newpage
\addtocounter{table}{-1}
\begin{table}[h] % enumeration of KBMAs (ranks 51 to 100)
\vspace{0mm}  %
\includegraphics[page=2, scale = 1.01, trim = 18mm 60mm 0mm 19mm, clip]{enumeration_kbmas.2025_03_20b.pdf}
\vspace{-7mm}
\caption{\label{kbma_enumeration_p1} \textbf{Enumeration of KBMAs} (ranks 51 to 100 of 361) {\normalfont \footnotesize \hspace{0.5mm} ``Max Dist'' is the distance between the tract centroids that are farthest from each other. Tables with more detailed variables for KBMAs and analogous tables for kernel-based metropolitan regions and kernel-based urban areas are available \href{https://www.kansascityfed.org/research/research-working-papers/a-better-delineation-of-us-metropolitan-areas/}{\underline{online}}.}}
\end{table}


\newpage
\addtocounter{table}{-1}
\begin{table}[h] % enumeration of KBMAs (ranks 101 to 150)
\vspace{0mm}  %
\includegraphics[page=3, scale = 1.01, trim = 18mm 60mm 0mm 19mm, clip]{enumeration_kbmas.2025_03_20b.pdf}
\vspace{-7mm}
\caption{\label{kbma_enumeration_p1} \textbf{Enumeration of KBMAs} (ranks 101 to 150 of 361) {\normalfont \footnotesize \hspace{0.5mm} ``Max Dist'' is the distance between the tract centroids that are farthest from each other. Tables with more detailed variables for KBMAs and analogous tables for kernel-based metropolitan regions and kernel-based urban areas are available \href{https://www.kansascityfed.org/research/research-working-papers/a-better-delineation-of-us-metropolitan-areas/}{\underline{online}}.}}
\end{table}



\newpage
\addtocounter{table}{-1}
\begin{table}[h] % enumeration of KBMAs (ranks 151 to 200)
\vspace{0mm}  %
\includegraphics[page=4, scale = 1.01, trim = 18mm 60mm 0mm 19mm, clip]{enumeration_kbmas.2025_03_20b.pdf}
\vspace{-7mm}
\caption{\label{kbma_enumeration_p1} \textbf{Enumeration of KBMAs} (ranks 151 to 200 of 361) {\normalfont \footnotesize \hspace{0.5mm} ``Max Dist'' is the distance between the tract centroids that are farthest from each other. Tables with more detailed variables for KBMAs and analogous tables for kernel-based metropolitan regions and kernel-based urban areas are available \href{https://www.kansascityfed.org/research/research-working-papers/a-better-delineation-of-us-metropolitan-areas/}{\underline{online}}.}}
\end{table}



\newpage
\addtocounter{table}{-1}
\begin{table}[h] % enumeration of KBMAs (ranks 201 to 250)
\vspace{0mm}  %
\includegraphics[page=5, scale = 1.01, trim = 18mm 60mm 0mm 19mm, clip]{enumeration_kbmas.2025_03_20b.pdf}
\vspace{-7mm}
\caption{\label{kbma_enumeration_p1} \textbf{Enumeration of KBMAs} (ranks 201 to 250 of 361) {\normalfont \footnotesize \hspace{0.5mm} ``Max Dist'' is the distance between the tract centroids that are farthest from each other. Tables with more detailed variables for KBMAs and analogous tables for kernel-based metropolitan regions and kernel-based urban areas are available \href{https://www.kansascityfed.org/research/research-working-papers/a-better-delineation-of-us-metropolitan-areas/}{\underline{online}}.}}
\end{table}



\newpage
\addtocounter{table}{-1}
\begin{table}[h] % enumeration of KBMAs (ranks 251 to 300)
\vspace{0mm}  %
\includegraphics[page=6, scale = 1.01, trim = 18mm 60mm 0mm 19mm, clip]{enumeration_kbmas.2025_03_20b.pdf}
\vspace{-7mm}
\caption{\label{kbma_enumeration_p1} \textbf{Enumeration of KBMAs} (ranks 251 to 300 of 361) {\normalfont \footnotesize \hspace{0.5mm} ``Max Dist'' is the distance between the tract centroids that are farthest from each other. Tables with more detailed variables for KBMAs and analogous tables for kernel-based metropolitan regions and kernel-based urban areas are available \href{https://www.kansascityfed.org/research/research-working-papers/a-better-delineation-of-us-metropolitan-areas/}{\underline{online}}.}}
\end{table}



\newpage
\addtocounter{table}{-1}
\begin{table}[h] % enumeration of KBMAs (ranks 301 to 350)
\vspace{0mm}  %
\includegraphics[page=7, scale = 1.01, trim = 18mm 60mm 0mm 19mm, clip]{enumeration_kbmas.2025_03_20b.pdf}
\vspace{-7mm}
\caption{\label{kbma_enumeration_p1} \textbf{Enumeration of KBMAs} (ranks 301 to 350 of 361) {\normalfont \footnotesize \hspace{0.5mm} ``Max Dist'' is the distance between the tract centroids that are farthest from each other. Tables with more detailed variables for KBMAs and analogous tables for kernel-based metropolitan regions and kernel-based urban areas are available \href{https://www.kansascityfed.org/research/research-working-papers/a-better-delineation-of-us-metropolitan-areas/}{\underline{online}}.}}
\end{table}


\newpage
\addtocounter{table}{-1}
\begin{table}[h] % enumeration of KBMAs (ranks 351 to 361)
\vspace{0mm}  %
\includegraphics[page=8, scale = 1.01, trim = 18mm 200mm 0mm 19mm, clip]{enumeration_kbmas.2025_03_20b.pdf}
\vspace{-7mm}
\caption{\label{kbma_enumeration_p1} \textbf{Enumeration of KBMAs} (ranks 351 to 361 of 361) {\normalfont \footnotesize \hspace{0.5mm} ``Max Dist'' is the distance between the tract centroids that are farthest from each other. Tables with more detailed variables for KBMAs and analogous tables for kernel-based metropolitan regions and kernel-based urban areas are available \href{https://www.kansascityfed.org/research/research-working-papers/a-better-delineation-of-us-metropolitan-areas/}{\underline{online}}.}}
\end{table}


% enumeration of KBMAs
\end{comment}


\end{document}


