\documentclass[12pt,letterpaper]{article}
\usepackage{graphicx}
\usepackage{epstopdf}
%\usepackage{natbib} %incompatible with biblatex
\usepackage{amsmath, amssymb,bbm}
\usepackage{color,hyperref}
\definecolor{darkblue}{rgb}{0.0,0.0,0.3}
\hypersetup{colorlinks,breaklinks,
            linkcolor=darkblue,urlcolor=darkblue,
            anchorcolor=darkblue,citecolor=darkblue}
       
\usepackage[left=1.0in,top=1.0in,bottom=1.0in,right=1.0in]{geometry}
\renewcommand{\baselinestretch}{1.25}
\usepackage{threeparttable}
\usepackage{booktabs}
\usepackage{listings}
\usepackage{subcaption}
%\captionsetup[subcaption]{labelformat=empty}
\usepackage{pdflscape}
%\usepackage{siunitx}
\newcommand\THead[1]{\multicolumn{1}{c}{#1}}

\usepackage{etoolbox}
\usepackage{hyperref}
\usepackage{sepfootnotes}
\usepackage{dsfont}
\usepackage{bbding}
\usepackage{xcolor}
\input{biblatex-aer}
%\usepackage[authordate]{biblatex-chicago}
\addbibresource{references.bib}

\makeatletter

% make numeric styles use name format
\patchcmd{\NAT@test}{\else \NAT@nm}{\else \NAT@nmfmt{\NAT@nm}}{}{}

% define \citepos just like \citet
\DeclareRobustCommand\citepos
  {\begingroup
   \let\NAT@nmfmt\NAT@posfmt% ...except with a different name format
   \NAT@swafalse\let\NAT@ctype\z@\NAT@partrue
   \@ifstar{\NAT@fulltrue\NAT@citetp}{\NAT@fullfalse\NAT@citetp}}

\let\NAT@orig@nmfmt\NAT@nmfmt
\def\NAT@posfmt#1{\NAT@orig@nmfmt{#1's}}

\makeatother

%redefine \citeauthor to link to bibliography
\DeclareCiteCommand{\citeauthor}
  {\boolfalse{citetracker}%
   \boolfalse{pagetracker}%
   \usebibmacro{prenote}}
  {\ifciteindex
     {\indexnames{labelname}}
     {}%
   \printtext[bibhyperref]{\printnames{labelname}}}
  {\multicitedelim}
  {\usebibmacro{postnote}}


\begin{document}
\protected\def\str#1{$^{#1}$}

%remove annoying "Underfull hbox" warnings/linting
\hbadness=99999

%siunitx setup
%\sisetup{
%    input-open-uncertainty  = ,
%    input-close-uncertainty = ,
%    table-align-text-pre    = false,
%    round-precision=2,
%    table-space-text-pre    = (,
%    table-space-text-post   = \str{***},
%}

\title{\vspace{-0.7in}\Large{\bf{Liquidity Shocks in Virtual Markets}}\thanks{I thank Elior Cohen, Francisco Scott, Jordan Pandolfo, Brent Bundick, and Lee Smith for their helpful comments and feedback. The views expressed herein are solely those of the author and do not necessarily reflect the views of the Federal Reserve Bank of Kansas City or the Federal Reserve System.  \newline}}%\newline Available: \href{http://dx.doi.org/10.18651/RWP2016-02}{http://dx.doi.org/10.18651/RWP2016-02} \vspace{0.15in} }}
\author{\vspace{5mm} Caleb Bray \\ \vspace{-0.8mm}}
\date{\normalsize{\hspace{0.1in}September 2025}}

\maketitle
\begin{abstract}
\noindent This paper serves as a case study to examine the effects of sudden market structure changes in virtual asset markets. The virtual market I examine behaves similarly to over-the-counter asset markets where liquidity shocks have a meaningful effect on asset price and choice. Using a difference-in-differences approach, I find that implementation of a 7-day trade ban on all cosmetic items in the video game Counter-Strike reduces quality adjusted prices by 4.2\% and trading volume by 34.4\% suggesting the creation of an illiquidity discount. This effect persists for at least two years after treatment and is consistent with empirical findings in the literature. \\
\end{abstract}
 


\newpage 
	 
\section{Introduction}
There is a broad literature covering transaction costs and illiquidity in ``over-the-counter'' asset markets inspired by \textcite{dgp_2005}. These papers highlight the meaningful effects of trade delays and market driven declines in liquidity on asset price and choice \parencites{dgp_2007}{krishna_2002}{lagos_rocheteau_2007}{amihud_et_al_2006}{geromichalos_et_al_2016}. These contributions analyze declines in liquidity due to market structure but are unable to observe the effect of direct and fixed constraints on liquidity. In this paper I leverage a novel dataset of heterogenous digital assets in the form of video game cosmetics (``skins'') to analyze the price effect of a 7-day trade delay enforced by the game's developer. I find a 4.2\% decline in quality adjusted selling prices that persists for at least 2 years post-implementation despite growing player counts.\\

The growing popularity of the video game Counter-Strike has created a large and unique market of digital assets. In March 2025, the market capitalization for digital cosmetics in the video game Counter-Strike reached an all-time high of \$4.3 billion \autocite{bloomberg2025-03-07}. As of August 2025, the game averages about 1 million concurrent players, playing the game at any given moment \parencite{steamcharts_cs}. Third parties also estimate that about 2-3 million trades are made between users every week \parencite{tradeTF} and around 32 million new Counter-Strike skins are generated every month via player payments to the developer to open ``cases'' which function as virtual slot machines \parencite{case-tracker}. As assets, Counter-Strike skins experience sizable returns, with average annual returns of 40\% and low exposure to traditional financial markets \parencite{dobrynskaya_2025}. There are considerable seasonal trends related to game updates, however.\\

The developer Valve fostered this market in combination with their expansive digital PC game distributor, Steam. In addition to purchasing PC games on Steam, users can list their in-game items on Steam's community marketplace, where other users can purchase them using real currency. This marketplace serves as a valuable source of high frequency data for emerging digital markets and exchanges controlled by a central figure. Using these data we can study the effects of various policy changes and leverage their relative exogeneity to identify treatment effects. In this paper, I examine an exogenous liquidity shock in the form of a 7-day trade restriction imposed on all Counter-Strike items to examine the change in asset prices. Following the example set in the hedonic pricing and housing literature \parencites{lancaster_1966}{rosen_1974}{rosen1986}{sirmans_2005}, I carry out a first stage regression to control for heterogeneity in skin quality. After constructing a quality adjusted price, I use an event study difference-in-differences (DID) approach to estimate the effect on both quantity and price of Counter-Strike skins sold on the Steam Community Market. I find a persistent negative effect on both selling prices and quantity sold for two years after the imposition of the 7-day trade ban. \\

\section{Institutional Details}
Skins in Counter-Strike hold significant monetary value despite being purely cosmetic items with no gameplay effects. These skins are created via opening cases that are randomly placed in users' Steam inventories after a completed match. Users may pay \$2.50 to buy a key from the developer to ``open'' the case, where the most likely outcome is receiving a skin worth a couple of cents, but there is an extremely slim chance (the rarest items having a 0.26\% chance) of unboxing something worth thousands of dollars. These skins have monetary value after being opened because they can be listed for sale on the Steam Community Market. Other Steam users can purchase these skins using real cash or their Steam account balances which in-turn transfers that amount to the seller's Steam account balance (minus a 15\% fee that goes to Valve). For most users, Steam account balance is largely equivalent to cash in hand as the consumers of Counter-Strike skins tend to also use their Steam account balances to buy PC video games on Steam. Additionally, prices listed on unsanctioned third party exchanges are typically close or equal to the listed prices on the Steam Community Market. \\

\begin{figure}[h]
\vspace{-10mm}
\textbf{\caption{Monthly Averages of Weekly Trade Offers and Player Counts \label{fig:trade_delta}}}\vspace{-5mm}
\begin{center}
\includegraphics[width=7.3in]{./figures/trade_delta_and_plyr_ct.pdf}
\end{center}
\footnotesize{Note: Steam trade offer data includes trade offers for all Steam items and is estimated from the difference in trade offer ID between trades. Top chart is sourced from \citeauthor{tradeTF} and bottom figure is sourced from \citeauthor{steamcharts_cs} \nocite{steamcharts_dota}.}
\end{figure}

To rein in scamming, phishing, speculation, third party gambling sites, and eSports skin betting, Valve instituted a 7-day trade ban for all Counter-Strike items on March 29th, 2018. This acted as an exogenous liquidity shock by restricting the trade and sale of all newly acquired Counter-Strike items for 7 days. This policy continues to be in effect, such that anytime a user acquires a new Counter-Strike skin via unboxing, trade, or purchase from the Steam Community Market, this item cannot be traded or sold for 7 days. This policy change also affected third party exchanges as items are listed and exchanged on these sites via users trading with automated bot accounts on Steam. \\

Crucially however, other games that had items listed on the Steam Community Market were unaffected at the time. This enabled me to use a DID approach to estimate the treatment effect on the treated of this overnight policy change using items from the game Dota 2 as an untreated group. Dota 2 is the only game with a comparably large skin market that behaves similarly to Counter-Strike. Dota 2 is another game developed by Valve, thus the items in Dota 2 are also purely cosmetic with no gameplay effects, and they are unlocked similarly by paying to acquire ``treasures'' that act like slot machines. Dota 2 and Counter-Strike also have similarly sized player bases and skin markets. \hyperref[fig:trade_delta]{Figure 1} shows the monthly averages for weekly trade offers on Steam and concurrent player counts for Counter-Strike and Dota 2. \\

\begin{table}[h]
\textbf{\caption{Counter-Strike and Dota 2 Items Summary}}
\label{tab:cs-dota2-compare}
\begin{center}
\begin{tabular}{lcccc}
\hline
                        & Counter-Strike    & Dota 2    & Combined &  \\
\toprule
Observations            & 2940          & 2675          & 5615  &  \\
Unique Items            & 60            & 56            & 116   &  \\
Oldest Origin Date      & 2013-08-14    & 2014-05-08    & 2013-08-14 &  \\
Newest Origin Date      & 2016-02-18    & 2016-10-05    & 2016-10-05 &  \\
Median Sale Price       & \$4.26        & \$1.90        & \$4.13    & \\ 
                        & (\$7.28)      & (\$3.61)      & (\$7.14) &  \\
Log Median Sale Price   & 0.517         & -0.064        & 0.484 & \\
                        & (1.379)       & (1.160)       & (1.253)   & \\
Volume Sold             & 20524.25      &  1356.84      & 11392.85  & \\ 
                        & (37685.37)    & (2201.17)     & (28938.78) & \\
Grade Rarity            & 0.154         & 0.110         & 0.152 & \\ 
                        & (0.109)       & (0.065)       & (0.107)   & \\
\bottomrule
\end{tabular}
\end{center}
\footnotesize{Note: Median Price and Grade Rarity are monthly averages weighted by volume sold with weighted standard deviation in parentheses below. Volume Sold is a monthly average with unweighted standard errors. Origin date and grade rarity are provided by \citeauthor{cs2db} and \citeauthor{backpackcs} \nocite{backpackdota} respectively.}
\end{table}
\section{Data and Methods}
To estimate the treatment effect of this exogenous shock, I leveraged a novel dataset of item sales on the Steam Community Market. These data are pulled from Steam's Web API via webscraping and provide daily median sale price and volume sold for items listed on the Steam Community Market. I take a subset of items from Dota 2 and Counter-Strike that is broadly representative of the market as a whole because the number of unique items is large, and some items are infrequently sold on the Steam Community Market. Additionally, Dota 2 faced the exact same 7-day trade ban on May 25, 2020, so I limit the dataset to cover 2 years before and after Counter-Strike's trade ban, a period covering March 29th, 2016 to March 29th, 2020. Lastly, I aggregate the data by month weighted by volume sold in order to increase statistical power, minimize noise, and join on monthly level average player counts. \hyperref[tab:cs-dota2-compare]{Table 1} shows the characteristics of the subset of items I use in my regressions. \\

\vspace{-5mm}
\begin{equation}
    \log(p_{it}) = \beta_0 + \beta_1 ItemAge_{it} + \beta_2 ItemAge_{it}^2 + \beta_3 GradeRarity_{i} + \beta_4 GradeRarity_{i}^2 + \varepsilon_{it} \label{eq:phat_reg}
\end{equation}

%Foot notes for below paragraph (line 125)
\sepfootnotecontent{1}{In both Counter-Strike and Dota 2, grades are assigned as rough proxies for their rarity. In Counter-Strike (in order of increasing rarity) these grades are: ``Mil-spec,'' ``Restricted,'' ``Classified,'' and ``Covert''. In Dota 2 these grades are: ``Rare,'' ``Mythical,'' ``Legendary,'' and ``Immortal''.}
\sepfootnotecontent{2}{One key assumption is that today's grade rarity applies to the 2016-2018 time period. This is reasonable because case opening odds for Counter-Strike and Dota 2 items have remained stable over time.}
\sepfootnotecontent{3}{Dota 2 items have a drop rarity separate from item grade, and a single item may appear in multiple different treasures with varying drop rarities. The chances of obtaining an item with higher drop rarity tier from a given treasure increases with subsequent treasure openings.}

As seen in \hyperref[tab:cs-dota2-compare]{Table 1}, there exists a wide variation within and between games in item attributes. Following the hedonic pricing literature \parencites{lancaster_1966}{rosen_1974}{rosen1986}{sirmans_2005}, I estimate the treatment effect on fitted quality-adjusted prices. \hyperref[eq:phat_reg]{Equation 1} shows the reduced form equation used to adjust prices for item \(i\) in period \(t\) weighted by volume sold. I add square terms to both item age (the number of months since the item was added to the game) and grade rarity to allow for nonlinearity in their relationships with actual price. In plotting both item age and grade rarity against actual log median price, both appear to fit reasonably well to a quadratic curve, but I do not add higher order terms in order to avoid overfitting. Intuitively, items tend to experience price appreciation in the medium-term as their age makes them rarer, but fall off in the longer term as newer, more aesthetically pleasing skins are released. Grade rarity\sepfootnote{1} is represented as the share of items of a certain ``grade'' that currently exist in the market and follows a decreasing concave down curve in relation to actual price\sepfootnote{2}. This relationship can be explained by the large chance one has to unbox lower tier items. In Counter-Strike the lowest two grade tiers have a 79.92\% and 15.98\% chance of being unboxed respectively. In Dota 2, these odds are further skewed as on average the lowest two grade of items have a 99\% chance of being unboxed on first time opening\sepfootnote{3}. \\

\begin{figure}[!ht]
\vspace{-20mm}
\textbf{\caption{Event Studies \label{fig:event_study}}}
\vspace{-5mm}
\begin{center}
\hspace{-6mm}
\makebox[0pt]{\includegraphics[height=4.5in]{./figures/did_mos_2yr_phat_iplot.pdf}}
\end{center}
\vspace{-16mm}
\begin{center}
\hspace{-6mm}
\makebox[0pt]{\includegraphics[height=4.5in]{./figures/did_mos_2yr_vol_iplot.pdf}}
\end{center}
\begin{center}
  \vspace{-8mm}
\footnotesize{Note: The last period, \(t=24\) is excluded and shown here as 0 due to collinearity.}
\end{center}
\end{figure}

I use this quality adjusted price to estimate the average treatment effect on the treated (ATT) in a simple DID setting, shown in \hyperref[eq:did]{Equation 2}. To control for endogenous swings in player counts affecting demand I include the natural log of monthly concurrent players for both Counter-Strike and Dota 2. I also use item- and month-level fixed effects to control for serial correlation. \\ 
\vspace{-5mm}
\begin{equation}
    \log(\hat{p}_{it}) = \beta_0 + \beta_1 \log(ConcurrentPlayers)_{it} + \delta Treated_i\times Post_t + \alpha_i + \gamma_t + \varepsilon_{it} \label{eq:did}
\end{equation}

To show the validity of my identification strategy, I conduct an event study regression as shown in \hyperref[eq:event_study]{Equation 3}. In this specification, I add dummy terms corresponding to a data point being \(k\) months before treatment period \(e\) (excluding \(k=-1\) to act as a reference period) interacted with treatment status to recover treatment effect dynamics. Thus, \(\delta_k\) is the average treatment effect of being \(k\) months before/after treatment at period \(e\). \hyperref[fig:event_study]{Figure 2} shows the results of the event study. \\

\vspace{-10mm}
\begin{equation}
\log(\hat{p}_{it}) = \beta_0 + \beta_1 \log(ConcurrentPlayers)_{it} + \sum_{k\neq-1}\delta_{k}\cdot\mathds{1}(k=t-e)_{t}\cdot Treated_{i} + \alpha_i + \gamma_t + \varepsilon_{it} \label{eq:event_study}
\end{equation}

\section{Results}
I find that when player count is fixed, and item and time fixed effects are added, there is \(\sim 4.2\%\) decrease in quality-adjusted prices of Counter-Strike items post trade ban implementation. This model is able to account for 99\% of the variation in quality adjusted prices, but only 13\% of the variation between items within a given period, likely due to the strong item fixed effects. As expected with heterogenous goods, my simple DID specification achieves both a more significant negative result and larger \(R^2\) when quality adjusted prices are regressed on \(Treated \times Post\). The results of both regressions can be found in Table \hyperref[tab:phat_reg]{2}. \\

The results are robust to quality differences in items due to the wide range of qualities captured in the data set as shown in \hyperref[tab:cs-dota2-compare]{Table 1} and the relatively high \(R^2\) of my quality adjustment regression. The quality adjustment regression from \hyperref[eq:phat_reg]{Equation 1} performs well despite the many unquantifiable characteristics belonging to skins in both Dota 2 and Counter-Strike, with an adjusted \(R^2\) of 0.45. As predicted, Item Age, Grade Rarity, and their square terms are strongly statistically significant.\\

When volume sold is regressed on my DID term and concurrent players, the DID term has a negative coefficient of \(-15703.3\) as expected, but the estimated effect of concurrent players is negative as well. Similarly, in my regression of quality adjusted price on the DID term and concurrent players, I observed a similarly negative coefficient on \(\log(ConcurrentPlayers)\). This is unexpected as I would expect more players playing the game to push prices up as demand for skins increases. The player count term seems to be capturing a moving ``game effect'' and the negative value may be attached to the higher prices Counter-Strike items sell for despite having lower player counts than Dota 2 over the majority of the 2-year window I analyze. Given the fluctuations in player counts over this time period, the interpretation of this coefficient is ambiguous, as in 2020 Counter-Strike slips above Dota 2 in concurrent player counts as seen in \hyperref[fig:trade_delta]{Figure 1}. \\

\begin{table}[ht!]
  \renewcommand{\arraystretch}{1.25}%
  \centering
  \caption{Quality Adjustment and DID Regression Results}
  \begin{tabular}[t]{lc}
    \hline
     & \(\log(p_{it})\) \\ \toprule
    \(\beta_0\)         & $1.595^{***}$ \\
                        & $(0.075)$ \\
    $ItemAge$           & $0.029^{***}$ \\ 
                        & $(0.003)$ \\
    $Item Age^2$        & $-0.001^{***}$ \\
                        & $(0.000)$ \\
    $GradeRarity$       & $-19.43^{***}$ \\ 
                        & $( 0.894)$ \\
    $Grade Rarity^2$    & $38.37^{***}$ \\
                        & $(2.623)$ \\ \hline
    N                   & 5615 \\
    Adjusted \(R^2\)    & 0.452 \\ 
    \bottomrule
  \end{tabular}\label{tab:phat_reg}
  \quad
  \begin{tabular}[t]{lcc}
    \hline
                                        & \(\log(\hat{p}_{it})\) & \(\log(Volume \: Sold)\) \\ \toprule
    $\log(ConcurrentPlayers)$           & $-0.050^{***}$         & $-0.422^{***}$ \\ 
                                        & $(0.012)$              & $(0.120)$ \\
    $Treated\times Post$                & $-0.043^{***}$         & $-0.326^{***}$ \\
                                        & $(0.010)$              & $(0.081)$ \\ \hline
    N                                   & 5579                   & 5615 \\
    Adjusted \(R^2\)                    & 0.997                  & 0.956 \\
    Within \(R^2\)                      & 0.130                  & 0.055 \\ \hline
    Item and Month FEs                  & \Checkmark             & \Checkmark  \\
    \bottomrule
  \end{tabular}\label{tab:did_regs}
\vspace{5mm}

{\raggedright \footnotesize{Note: $^{*}\, p<0.5$; $^{**}\, p<0.01$; $^{***}\, p<0.001$. Standard errors are shown in parentheses below their respective estimates. Both regressions with price as a dependent variable are weighted by Volume Sold.} \par}
\end{table}

There are also possible violations of the ``Stable Unit Treatment Value Assumption'' (SUTVA) in my identification strategy. Items from both games are listed on the Steam Community Marketplace, so Dota 2 items such as the expensive ``Dragonclaw Hook'' became valuable intermediates in Counter-Strike skin trading. Prices for Dragonclaw Hooks peaked shortly after the implementation of Counter-Strike's trade ban. As such, there are likely spillovers to Dota 2 items as users purchased Dota 2 items in order to get around the 7-day trade ban. Additionally, there are likely pre-trends I was unable to control for as seen in \hyperref[fig:event_study]{Figure 2} which are most likely attached to major game updates that affect the in-game appearances of items and draw new players. However, the consistent negative treatment effect estimates for 2 years after treatment do suggest that there is a nonnegligible negative effect overall. \\

The results are otherwise intuitive. Quality adjusted prices and quantity sold decrease as Counter-Strike items become less valuable as assets. Relative to Dota 2 items, Counter-Strike items appear to have taken on an ``illiquidity discount''. However, the items retained significant aesthetic value as their price level remained above Dota 2 items. This shock appeared as a one-time downward shift in price level for Counter-Strike items that gradually faded away as the market for Counter-Strike items continued to grow. Dota 2 items temporarily became more valuable as they served as liquid intermediates for Counter-Strike skin trading, but their prices continued downward over time. \\ 

\section{Conclusion}
This paper serves as a case study to analyze the effects of policy-based market structure changes on asset price and choice. I argue that virtual marketplaces such as the Steam Community Marketplace can be valuable sources of data to explore the effects of sudden transaction cost increases in ``Over the Counter'' financial markets where liquidity shocks greatly affect asset prices. Using a simple DID identification strategy comparing Counter-Strike items to Dota 2 items, I find that the 7-day trade ban imposed on Counter-Strike items caused quality adjusted prices to decline by \(\sim 4.2\%\) and decreased the monthly volume sold by -15703.3. This effect manifested as a downward shift in quality adjusted prices that attenuated over time as Counter-Strike items retained their aesthetic value but lost some of their investment value. Counter-Strike items attained an illiquidity discount relative to Dota 2 items due to this overnight change in market structure. Lastly, avenues for further research include: using archived Steam Community Market listings to explore the effect of market structure changes on bid-ask spreads, exploring the role of unsanctioned third party market makers (such as \textcolor{blue}{\href{https://skinport.com/}{Skinport}}), and estimating the value of limited use but guaranteed Steam account balances versus riskier cash payouts from selling skins on different marketplaces. \\

%\bibliographystyle{aea}
\printbibliography
\clearpage

\end{document}
