\documentclass[12pt,letterpaper]{article}
\usepackage{graphicx}
\usepackage{epstopdf}
\usepackage{natbib}
\usepackage{amsmath, amssymb,bbm}
\usepackage{color,hyperref}
\definecolor{darkblue}{rgb}{0.0,0.0,0.3}
\hypersetup{colorlinks,breaklinks,
            linkcolor=darkblue,urlcolor=darkblue,
            anchorcolor=darkblue,citecolor=darkblue}
       
\usepackage[left=1.0in,top=1.0in,bottom=1.0in,right=1.0in]{geometry}
\renewcommand{\baselinestretch}{1.25}
\usepackage{threeparttable}
\usepackage{booktabs}
\usepackage{pdflscape}
%\usepackage{siunitx}
\newcommand\THead[1]{\multicolumn{1}{c}{#1}}

\usepackage{etoolbox}
\usepackage{hyperref}
\usepackage{sepfootnotes}
\usepackage{dsfont}
\usepackage{bbding}

\makeatletter

% make numeric styles use name format
\patchcmd{\NAT@test}{\else \NAT@nm}{\else \NAT@nmfmt{\NAT@nm}}{}{}

% define \citepos just like \citet
\DeclareRobustCommand\citepos
  {\begingroup
   \let\NAT@nmfmt\NAT@posfmt% ...except with a different name format
   \NAT@swafalse\let\NAT@ctype\z@\NAT@partrue
   \@ifstar{\NAT@fulltrue\NAT@citetp}{\NAT@fullfalse\NAT@citetp}}

\let\NAT@orig@nmfmt\NAT@nmfmt
\def\NAT@posfmt#1{\NAT@orig@nmfmt{#1's}}

\makeatother

\begin{document}
\protected\def\str#1{$^{#1}$}

%remove annoying "Underfull hbox" warnings/linting
\hbadness=99999

%siunitx setup
%\sisetup{
%    input-open-uncertainty  = ,
%    input-close-uncertainty = ,
%    table-align-text-pre    = false,
%    round-precision=2,
%    table-space-text-pre    = (,
%    table-space-text-post   = \str{***},
%}

\title{\vspace{-0.7in}\Large{\bf{Liquidity Shocks in Virtual Skin Markets}}\thanks{I thank Elior Cohen, Francisco Scott, Jordan Pandolfo, and Brent Bundick for their helpful comments and feedback. The views expressed herein are solely those of the author and do not necessarily reflect the views of the Federal Reserve Bank of Kansas City or the Federal Reserve System.  \newline}}%\newline Available: \href{http://dx.doi.org/10.18651/RWP2016-02}{http://dx.doi.org/10.18651/RWP2016-02} \vspace{0.15in} }}
\author{\vspace{5mm} Caleb Bray \\ \vspace{-0.8mm}}
\date{\normalsize{\hspace{0.1in}September 2025}}

\maketitle
\begin{abstract}
\noindent TEXT HERE  \\
\end{abstract}
 

%\noindent\textbf{JEL Classification:} E32, E52   \\
%\noindent\textbf{Keywords:} Monetary Policy Uncertainty, Forward Guidance, Proxy VAR \\
\newpage 
	 
\section{Introduction}
In May 2025, the market capitalization for digital cosmetics in the video game Counter Strike reached an all-time high of \$5 billion (CITE PRICE EMPIRE OR NEWS SOURCE). The market for cosmetics in Counter-Strike is immense and player-driven. As of August 2025, the game averages about 1 million concurrent players, playing the game at any given moment. Third parties also estimate that about 2-3 million trades are made between users every week (tradeTF CITE) and around 32 million new Counter Strike cosmetics are generated every month via player payments to the developer to open ``cases'' which function as virtual slot machines (CITE CSGO CASE TRACKER). \\

The developer Valve fostered this market in combination with their expansive digital PC game distributor, Steam. In addition to purchasing PC games on Steam, users can list their in-game items, also called ``skins,'' on Steam's community marketplace, where other users can purchase them using real currency. This marketplace serves as a valuable source of high frequency data for emerging digital markets and exchanges controlled by a central figure. Using these data we can study the effects of various policy changes and leverage their relative exogeneity to parse out their effects. In this paper, I leverage an exogenous liquidity shock in the form of a 7-day trade restriction imposed on all Counter-Strike items to examine the change in market participants' behavior. After constructing a quality adjusted price, I use an event study difference-in-differences (DID) approach to estimate the effect on both quantity and price of Counter-Strike skins sold on the Steam Community Market. I find a persistent positive effect on selling prices and negative effect on quantity sold for one year after the imposition of the 7-day trade ban. \\

\section{Institutional Details}
Skins in Counter-Strike hold significant monetary value despite being purely cosmetic items that only change the appearance of a user's weapons in game. These skins are created via opening crates that are randomly placed in users' Steam inventories after a completed match. Users pay \$2.50 to buy a key from the developer in order to essentially take a spin of a virtual slot machine where the most likely outcome is receiving a skin worth a couple of cents, but there is an extremely slim chance (the rarest items having a 0.26\% chance) of unboxing something worth hundreds or thousands of dollars. These skins have monetary value after being opened because they can be listed for sale on the Steam Community Market. Other Steam users can purchase these skins using real cash or their Steam account balances which in-turn transfers that amount to the seller's Steam account balance (minus a 15\% fee that goes to Valve). Steam account balance is largely equivalent to cash in hand as the consumers of Counter-Strike skins tend to also use their Steam account balances to buy PC video games on Steam. Additionally, prices listed on unsanctioned third party exchanges are typically close or equal to the listed prices on the Steam Community Market. \\

\begin{figure}[h]
\vspace{-10mm}
\textbf{\caption{Monthly Averages of Weekly Trade Offers and Player Counts \label{fig:trade_delta}}}
\begin{center}
\includegraphics[width=7.3in]{./figures/trade_delta_and_plyr_ct.pdf}
\end{center}
\footnotesize{Note: Steam trade offer data includes trade offers for all Steam items and is estimated from the difference in trade offer ID between trades. Top chart is sourced from Trade.TF and bottom figure is sourced from Steam Charts.}
\end{figure}

To rein in scamming, phishing, speculation, third party gambling sites, and eSports skin betting, Valve instituted a 7-day trade ban for all Counter-Strike items on March 29th, 2018. This acted as an exogenous liquidity shock by restricting the trade and sale of all newly acquired Counter-Strike items for 7 days. This policy continues to be in effect, such that anytime a user acquires a new Counter-Strike skin via unboxing, trade, or purchase from the Steam Community Market, this item cannot be traded or sold for 7 days. This policy change also affected third party exchanges as items are listed and exchanged on these sites via users trading with automated bot accounts on Steam. \\

Crucially however, other games that had items listed on the Steam Community Market were unaffected at the time. This enabled me to use a DID approach to estimate the treatment effect of this overnight policy change using items from the game Dota 2 as an untreated group. Dota 2 is the only game with a comparably large skin market that behaves similarly to Counter-Strike. Dota 2 is another game developed by Valve, thus the items in Dota 2 are also purely cosmetic with no gameplay effects, and they are unlocked similarly by paying to acquire ``treasures'' that act like slot machines. Dota 2 and Counter-Strike also have similarly sized player bases and skin markets. \hyperref[fig:trade_delta]{Figure 1}  shows the monthly averages for weekly trade offers on Steam and concurrent player counts for Counter-Strike and Dota 2. \\

%MAYBE MAKE THIS INTO TWO PART TABLE WITH DAILY VS MONTHLY? -- meh probably not
\vspace{-5mm}
\begin{table}[h]
\textbf{\caption{Counter-Strike and Dota 2 Items Summary}}
\label{tab:cs-dota2-compare}
\begin{center}
\begin{tabular}{lcccc}
                        & Counter-Strike    & Dota 2    & Total &  \\
\hline
Observations            & 2940          & 2675          & 5615  &  \\
Unique Items            & 60            & 56            & 116   &  \\
Oldest Origin Date      & 2013-08-14    & 2014-05-08    & 2013-08-14 &  \\
Newest Origin Date      & 2016-02-18    & 2016-10-05    & 2016-10-05 &  \\
Median Sale Price       & \$4.26        & \$1.90        & \$4.13    & \\ 
                        & (\$7.28)      & (\$3.61)      & (\$7.14) &  \\
Log Median Sale Price   & 0.517         & -0.064        & 0.484 & \\
                        & (1.379)       & (1.160)       & (1.253)   & \\
Volume Sold             & 20524.25      &  1356.84      & 11392.85  & \\ 
                        & (37685.37)    & (2201.17)     & (28938.78) & \\
Grade Rarity            & 0.154         & 0.110         & 0.152 & \\
                        & (0.109)       & (0.065)       & (0.107)   & \\
\hline
\end{tabular}
\end{center}
\footnotesize{Note: Median Price and Grade Rarity are monthly averages weighted by volume sold with weighted standard deviation in parentheses below. Volume Sold is a monthly average. Origin date and grade rarity are provided by csgoskins.gg and backpack.tf respectively.}
\end{table}
\section{Data and Methods}
To estimate the treatment effect of this exogenous shock, I leveraged a novel dataset of item sales on the Steam Community Market. These data are pulled from Steam's Web API via webscraping and provide daily median sale price and volume sold for items listed on the Steam Community Market. I take a subset of items from Dota 2 and Counter-Strike that is broadly representative of the market as a whole because the number of unique items is large, and some items are infrequently sold on the Steam Community Market. Additionally, Dota 2 faced the exact same 7-day trade ban on May 25, 2020, so I limit the dataset to cover 2 years before and after Counter-Strike's trade ban, a period covering March 29th, 2016 to March 29th, 2020. Lastly, I aggregate the data by month weighted by volume sold in order to minimize noise and join on monthly level average player counts. \hyperref[tab:cs-dota2-compare]{Table 1} shows the characteristics of the subset of items I use in my regressions.

\vspace{-5mm}
\begin{equation}
    \log(p_{it}) = \beta_0 + \beta_1 ItemAge_{it} + \beta_2 ItemAge_{it}^2 + \beta_3 GradeRarity_{i} + \beta_4 GradeRarity_{i}^2 + \varepsilon_{it} \label{eq:phat_reg}
\end{equation}

%Foot notes for below paragraph (line 125)
\sepfootnotecontent{1}{In both Counter-Strike and Dota 2, these grades are developer assigned and rough proxies for their rarity.  In Counter-Strike (in order of increasing rarity) these grades are: ``Mil-spec,'' ``Restricted,'' ``Classified,'' and ``Covert.'' In Dota 2 the grades are: ``Rare,'' ``Mythical,'' ``Legendary,'' and ``Immortal.''}
\sepfootnotecontent{2}{One key assumption is that today's grade rarity applies to the 2016-2018 time period. This is reasonable because case opening odds for Counter-Strike and Dota 2 items have remained stable over time.}
\sepfootnotecontent{3}{Dota 2 items have a drop rarity separate from item grade, and a single item may appear in multiple different treasures with varying drop rarities. The chances of obtaining an item with higher drop rarity tier from a given treasure increases with subsequent treasure openings.}

As seen in \hyperref[tab:cs-dota2-compare]{Table 1}, there exists a wide variation within and between games in item attributes. Following the hedonic pricing literature (CITES HERE ROSEN), I estimate the treatment effect on fitted quality-adjusted prices. \hyperref[eq:phat_reg]{Equation 1} shows the reduced form equation used to adjust prices for item \(i\) in period \(t\) weighted by volume sold. I add square terms to both item age (the number of months since the item was added to the game) and grade rarity to allow for nonlinearity in their relationships with actual price. In plotting both item age and grade rarity against actual log median price, both appear to fit reasonably well to a quadratic curve, but I do not add higher order terms in order to avoid overfitting. Intuitively, items tend to experience price appreciation in the medium-term as their age makes them rarer, but fall off in the longer term as newer, more aesthetically pleasing skins are released. Grade rarity\sepfootnote{1} is represented as the share of items of a certain ``grade'' that currently exist in the market and follows a decreasing concave down curve in relation to actual price\sepfootnote{2}. This relationship can be explained by the large chance one has to unbox lower tier items. In Counter-Strike the lowest two grade tiers have a 79.92\% and 15.98\% chance of being unboxed respectively. In Dota 2, these odds are further skewed as on average the lowest two grade of items have a 99\% chance of being unboxed on first time opening\sepfootnote{3}. \\

\begin{figure}[ht]
\vspace{-20mm}
\textbf{\caption{Event Studies \label{fig:event_study}}}
\vspace{-5mm}
\begin{center}
\hspace{-5mm}
\makebox[0pt]{\includegraphics[height=4.5in]{./figures/did_mos_2yr_phat_iplot.pdf}}
\end{center}
\vspace{-15mm}
\begin{center}
\hspace{-5mm}
\makebox[0pt]{\includegraphics[height=4.5in]{./figures/did_mos_2yr_vol_iplot.pdf}}
\end{center}
\begin{center}
\footnotesize{Note: The last period, \(t=24\) is excluded and shown here as 0 due to collinearity.}
\end{center}
\end{figure}

I use this quality adjusted price to estimate the average treatment effect on the treated (ATT) in a simple DID setting, shown in \hyperref[eq:did]{Equation 2}. To control for endogenous swings in player counts affecting demand I include the natural log of monthly concurrent players for both Counter-Strike and Dota 2. I also use item- and month-level fixed effects to control for serial correlation. \\ 
\vspace{-5mm}
\begin{equation}
    \log(\hat{p}_{it}) = \beta_0 + \beta_1 \log(ConcurrentPlayers)_{it} + \delta Treated_i\times Post_t + \alpha_i + \gamma_t + \varepsilon_{it} \label{eq:did}
\end{equation}

To show the validity of my identification strategy, I conduct an event study regression as shown in \hyperref[eq:event_study]{Equation 3}. In this specification, I add dummy terms for each period (excluding period -1 to act as a reference) interacted with treatment status to recover treatment effect dynamics. Thus, \(\delta_k\) is the average treatment effect of being \(k\) months before/after treatment at period \(e\). \hyperref[fig:event_study]{Figure 2} shows the results of the event study. \\


\vspace{-15mm}
\begin{equation}
\log(\hat{p}_{it}) = \beta_0 + \beta_1 \log(ConcurrentPlayers)_{it} + \sum_{k\neq-1}\delta_{k}\cdot\mathds{1}(k=t-e)_{t}\cdot Treated_{i} + \alpha_i + \gamma_t + \varepsilon_{it} \label{eq:event_study}
\end{equation}

\section{Results}
I find that when player count is fixed, and item and time fixed effects are added, there is \(\sim 4.2\%\) decrease in quality-adjusted prices of Counter-Strike items post trade ban implementation. This model is able to account for 99\% of the variation in quality adjusted prices, but only 13\% of the variation between items within a given period, likely due to the strong item fixed effects. As expected with heterogenous goods, my simple DID specification achieves both a more significant negative result and larger \(R^2\) when quality adjusted prices are regressed on \(Treated \times Post\). The results of both regressions can be found in Tables \hyperref[tab:phat_reg]{2} and \hyperref[tab:did_regs]{3}. \\

The results are robust to quality differences in items due to the wide range of qualities captured in the data set as shown in \hyperref[tab:cs-dota2-compare]{Table 1} and the relatively high \(R^2\) of my quality adjustment regression. The quality adjustment regression from \hyperref[eq:phat_reg]{Equation 1} performs well despite the many unquantifiable characteristics belonging to skins in both Dota 2 and Counter-Strike, with an adjusted \(R^2\) of 0.45. As predicted, Item Age, Grade Rarity, and their square terms are strongly statistically significant.\\


\begin{table}[ht]
\begin{minipage}[b]{0.56\linewidth}
\centering
\caption{Quality Adjustment Results}\vspace{5mm}
\begin{tabular}{lc}
    \hline
     & \(\log(p_{it})\) \\ \hline \hline
    \(\beta_0\)         & $1.595^{***}$ \\
                        & $(0.075)$ \\
    $ItemAge$           & $0.029^{***}$ \\ 
                        & $(0.003)$ \\
    $Item Age^2$        & $-0.001^{***}$ \\
                        & $(0.000)$ \\
    $GradeRarity$       & $-19.43^{***}$ \\ 
                        & $( 0.894)$ \\
    $Grade Rarity^2$    & $38.37^{***}$ \\
                        & $(2.623)$ \\ \hline
    N        & 5615 \\
    Adjusted \(R^2\)    & 0.452 \\ \hline \hline
   \end{tabular}
    \label{tab:phat_reg}
\end{minipage}\hfill
\begin{minipage}[b]{0.4\linewidth}
\centering
\caption{DID Results}\vspace{5mm}
\begin{tabular}{lcc}
    \hline
     & \(\log(\hat{p}_{it})\) & Volume Sold \\ \hline \hline
    $\log(ConcurrentPlayers)$           & $-0.050^{***}$ & $-12555.1^{***}$ \\ 
                                        & $(0.012)$      & $(2124.41)$ \\
    $Treated_i\times Post_t$            & $-0.043^{***}$ & $-18421.7^{***}$ \\
                                        & $(0.010)$      & $(4120.30)$ \\ \hline
    N                                   & 5579           & 5579 \\
    Adjusted \(R^2\)                    & 0.997          & 0.742 \\
    Within \(R^2\)                      & 0.130          & 0.080 \\ \hline
    Item and Month Level FEs  & \Checkmark     & \Checkmark  \\ \hline \hline
   \end{tabular}
    \label{tab:did_regs}
\end{minipage}\vspace{5mm}

\footnotesize{Note: $^{*}\, p<0.5$; $^{**}\, p<0.01$; $^{***}\, p<0.001$}
\end{table}



% \begin{table}[h]
% \textbf{\caption{Regression Results}}
% \label{tab:regs}
% \begin{center}
% \begin{tabular}{lllll}
%                    & Item & Dota 2 & Total &  \\

% \hline

% \hline
% \end{tabular}
% \end{center}
% \footnotesize{Note: Median Price and Grade Rarity are monthly averages weighted by volume sold with weighted standard deviation in parentheses below. Volume Sold is a monthly average. Origin date and grade rarity are provided by csgoskins.gg and backpack.tf respectively.}


\end{document}
